\documentclass{ximera}
%% You can put user macros here
%% However, you cannot make new environments

\graphicspath{{./}{firstExample/}{secondExample/}}

\usepackage{tikz}
\usepackage{tkz-euclide}
\usetkzobj{all}
\pgfplotsset{compat=1.7} % prevents compile error.

\tikzstyle geometryDiagrams=[ultra thick,color=blue!50!black]


\author{Anna Davis \and Paul Zachlin} \title{VEC-0100: Linear Independence} \license{CC-BY 4.0}

\begin{document}

\begin{abstract}
 We define linear independence of a set of vectors, and explore this concept algebraically and geometrically.
\end{abstract}
\maketitle

\section*{VEC-0100: Linear Independence}
In this module we discuss the notion of \dfn{linear independence} of a set of vectors $\{\vec{v}_1, \vec{v}_2,\ldots ,\vec{v}_p\}$.  We consider the equation \begin{align}\label{eq:defoflinind}a_1\vec{v}_1+a_2\vec{v}_2+\ldots +a_p\vec{v}_p=\vec{0}\end{align}
 
Observe that equation (\ref{eq:defoflinind}) is consistent, for the equation can be solved by letting  $a_1=a_2=\ldots =a_p=0$.  This solution is known as the \dfn{trivial solution}.

\begin{definition}[Linear Independence]\label{def:linearindependence}
Let $\vec{v}_1, \vec{v}_2,\ldots ,\vec{v}_p$ be vectors of $\RR^n$.  We say that the set $\{\vec{v}_1, \vec{v}_2,\ldots ,\vec{v}_p\}$ is \dfn{linearly independent} if the only solution to 
$$a_1\vec{v}_1+a_2\vec{v}_2+\ldots +a_p\vec{v}_p=\vec{0}$$
is the \dfn{trivial solution} $a_1=a_2=\ldots =a_p=0$.

If, in addition to the trivial solution, a \dfn{non-trivial solution} (not all $a_1, a_2,\ldots ,a_p$ are zero) exists, then we say that the set $\{\vec{v}_1, \vec{v}_2,\ldots ,\vec{v}_p\}$ is \dfn{linearly dependent}.
\end{definition}

If the {\it set} of vectors $\{\vec{v}_1, \vec{v}_2,\ldots ,\vec{v}_p\}$ is linearly independent (dependent), we often say that the {\it vectors} $\vec{v}_1, \vec{v}_2,\ldots ,\vec{v}_p$ are linearly independent (dependent).

% \begin{definition}[Linear Independence]\label{def:linearindependence}
% Let $\vec{v}_1, \vec{v}_2,\ldots ,\vec{v}_p$ be vectors of $\RR^n$.  We say that $\vec{v}_1, \vec{v}_2,\ldots ,\vec{v}_p$ are \dfn{linearly dependent} (or the set $\{\vec{v}_1, \vec{v}_2,\ldots ,\vec{v}_p\}$ is linearly dependent) if there exist scalars $a_1, a_2, \ldots , a_p$, not all zero, such that
% \begin{align}\label{eq:defoflinind}a_1\vec{v}_1+a_2\vec{v}_2+\ldots +a_p\vec{v}_p=\vec{0}\end{align}
% Otherwise we say that $\vec{v}_1, \vec{v}_2,\ldots ,\vec{v}_p$ are \dfn{linearly independent} (or the set $\{\vec{v}_1, \vec{v}_2,\ldots ,\vec{v}_p\}$ is linearly independent).
% \end{definition}

% Note that given {\it any} set of vectors in $\RR^n$, equation (\ref{eq:defoflinind}) will always have at least the \dfn{trivial solution} $a_1=a_2=\ldots =a_p=0$.  If the trivial solution is the only possible solution, then we say that the vectors $\vec{v}_1, \vec{v}_2,\ldots ,\vec{v}_p$ are linearly independent.  If, in addition to the trivial solution, a \dfn{non-trivial solution} (not all $a_1, a_2,\ldots ,a_p$ are zero) exists, then we say that vectors $\vec{v}_1, \vec{v}_2,\ldots ,\vec{v}_p$ are linearly dependent.

\begin{example}\label{ex:linind}Determine whether the vectors in each part are linearly independent.

\begin{enumerate}
\item \label{item:linindpart1}
$$\begin{bmatrix}2\\-3\end{bmatrix}, \begin{bmatrix}0\\3\end{bmatrix},\begin{bmatrix}1\\-1\end{bmatrix},\begin{bmatrix}1\\-2\end{bmatrix}$$

\item \label{item:linindpart2} $$\begin{bmatrix}2\\1\\4\end{bmatrix},\begin{bmatrix}-3\\1\\1\end{bmatrix}$$
\end{enumerate}
\begin{explanation} \ref{item:linindpart1}
We will solve the vector equation
\begin{align}\label{eq:linrelationpart1}a_1\begin{bmatrix}2\\-3\end{bmatrix}+a_2 \begin{bmatrix}0\\3\end{bmatrix}+a_3\begin{bmatrix}1\\-1\end{bmatrix}+a_4\begin{bmatrix}1\\-2\end{bmatrix}=\vec{0}\end{align}
 
 Clearly $a_1=a_2=a_3=a_4=0$ is a solution to the equation.  The question is whether another solution exists.
 
The vector equation translates into the following system:

$$\begin{array}{ccccccccc}
      2a_1 & &&+&a_3&+&a_4&= &0 \\
        -3a_1& +&3a_2&-&a_3&-&2a_4&= &0 \\
      \end{array}$$
  Writing the system in augmented matrix form and applying elementary row operations gives us the following reduced row-echelon form:
  $$\left[\begin{array}{cccc|c}  
 2&0&1&1&0\\-3&3&-1&-2&0
 \end{array}\right]\rightsquigarrow\left[\begin{array}{cccc|c}  
 1&0&1/2&1/2&0\\0&1&1/6&-1/6&0
 \end{array}\right]$$
 This shows that (\ref{eq:linrelationpart1}) has infinitely many solutions:  
 $$a_1=-\frac{1}{2}s-\frac{1}{2}t,\quad a_2=-\frac{1}{6}s+\frac{1}{6}t,\quad a_3=s,\quad a_4=t$$
 Letting $t=s=6$, we obtain the following:
 
 $$-6\begin{bmatrix}2\\-3\end{bmatrix}+0 \begin{bmatrix}0\\3\end{bmatrix}+6\begin{bmatrix}1\\-1\end{bmatrix}+6\begin{bmatrix}1\\-2\end{bmatrix}=\vec{0}$$
 We conclude that the vectors are linearly dependent.
 
 \ref{item:linindpart2} We need to solve the equation
 
 $$a_1\begin{bmatrix}2\\1\\4\end{bmatrix}+a_2\begin{bmatrix}-3\\1\\1\end{bmatrix}=\vec{0}$$
 Converting the equation to augmented matrix form and performing row reduction gives us
 $$\left[\begin{array}{cc|c}  
 2&-3&0\\1&1&0\\4&1&0
 \end{array}\right]\rightsquigarrow\left[\begin{array}{cc|c}  
 1&0&0\\0&1&0\\0&0&0
 \end{array}\right]$$
 This shows that $a_1=a_2=0$ is the only solution.  Therefore the two vectors are linearly independent.
\end{explanation}
\end{example}

In Part \ref{item:linindpart1} of Example \ref{ex:linind} we found that the vectors $\begin{bmatrix}2\\-3\end{bmatrix}, \begin{bmatrix}0\\3\end{bmatrix},\begin{bmatrix}1\\-1\end{bmatrix},\begin{bmatrix}1\\-2\end{bmatrix}$ are linearly dependent because

\begin{equation}\label{eq:nontrivrel}
-6\begin{bmatrix}2\\-3\end{bmatrix}+0 \begin{bmatrix}0\\3\end{bmatrix}+6\begin{bmatrix}1\\-1\end{bmatrix}+6\begin{bmatrix}1\\-2\end{bmatrix}=\vec{0}\end{equation}

Observe that (\ref{eq:nontrivrel}) allows us to solve for one of the vectors and express it as a linear combination of the others. For example, 
$$\begin{bmatrix}2\\-3\end{bmatrix}=0 \begin{bmatrix}0\\3\end{bmatrix}+\begin{bmatrix}1\\-1\end{bmatrix}+\begin{bmatrix}1\\-2\end{bmatrix}$$
This would not be possible if a nontrivial solution to the equation
$$a_1\begin{bmatrix}2\\-3\end{bmatrix}+a_2 \begin{bmatrix}0\\3\end{bmatrix}+a_3\begin{bmatrix}1\\-1\end{bmatrix}+a_4\begin{bmatrix}1\\-2\end{bmatrix}=\vec{0}$$
did not exist.

\begin{theorem}\label{th:lindeplincombofother}
A subset of $\RR^n$ containing two or more vectors is linearly dependent if and only if one of the vectors can be expressed as a linear combination of the others.
\end{theorem}
\begin{proof}
See Practice Problem \ref{prob:lindeplincombofother}.
\end{proof}

\subsection*{Geometry of Linearly Dependent and Linearly Independent Sets}

Theorem \ref{th:lindeplincombofother} gives us a convenient way of looking at linear dependence/independence geometrically.  When looking at two or more vectors, we ask ``can one of the vectors be written as a linear combination of the others?"  If the answer is ``yes", then the vectors are linearly dependent.
\subsubsection*{A Set of Two Vectors}
Two vectors are linearly dependent if and only if one is a scalar multiple of the other.  Two nonzero linearly dependent vectors may look like this:
\begin{center}
\begin{tikzpicture}[scale=0.8]
% \draw[<->] (-0.5,0)--(2.5,0);
 % \draw[<->] (0,-0.5)--(0,2.5);
    \draw[line width=1pt,blue,-stealth](0,0)--(2,2);
  \draw[line width=1pt,red,-stealth](0,0)--(1,1);
 \end{tikzpicture}
\end{center}
or like this:
\begin{center}
\begin{tikzpicture}[scale=0.8]
%\draw[<->] (-2.5,0)--(2.5,0);
 % \draw[<->] (0,-2.5)--(0,2.5);
    \draw[line width=1pt,blue,-stealth](0,0)--(1,2);
  \draw[line width=1pt,red,-stealth](0,0)--(-0.5,-1);
 \end{tikzpicture}
\end{center}
Two linearly independent vectors will look like this:
\begin{center}
\begin{tikzpicture}[scale=0.8]
%\draw[<->] (-2.5,0)--(2.5,0);
 % \draw[<->] (0,-2.5)--(0,2.5);
    \draw[line width=1pt,blue,-stealth](0,0)--(1,1);
  \draw[line width=1pt,red,-stealth](0,0)--(0.5,-1);
 \end{tikzpicture}
\end{center}
\subsubsection*{A Set of Three Vectors}
Given a set of three nonzero vectors, we have the following possibilities: 
\begin{itemize}
\item (Linearly Dependent Vectors)
The three vectors are scalar multiples of each other.
\begin{center}
\begin{tikzpicture}[scale=0.8]
% \draw[<->] (-0.5,0)--(2.5,0);
 % \draw[<->] (0,-0.5)--(0,2.5);
    \draw[line width=1pt,blue,-stealth](0,0)--(2,2);
  \draw[line width=1pt,red,-stealth](0,0)--(1,1);
  \draw[line width=1pt,-stealth](0,0)--(0.5,0.5);
 \end{tikzpicture}
\end{center}
\item (Linearly Dependent Vectors) Two of the vectors are scalar multiples of each other.
\begin{center}
\begin{tikzpicture}[scale=0.8]
    \draw[line width=1pt,blue,-stealth](0,0)--(2,2);
  \draw[line width=1pt,red,-stealth](0,0)--(1,1);
  \draw[line width=1pt,-stealth](0,0)--(1,0);
 \end{tikzpicture}
\end{center}
\item (Linearly Dependent Vectors) One vector can be viewed as the diagonal of a parallelogram determined by scalar multiples of the other two vectors.  All three vectors lie in the same plane.
\begin{center}
\begin{tikzpicture}[scale=0.8]
  \filldraw[blue, opacity=0.3](0,0)--(-2,2)--(2,4)--(4,2)--cycle;
\draw[line width=1pt,red,-stealth](0,0)--(2,1);
\draw[line width=1pt,red,-stealth, dashed](0,0)--(4,2);
  \draw[line width=1pt,blue,-stealth](0,0)--(-2,2);
  \draw[line width=1pt,-stealth](0,0)--(2,4); 
\end{tikzpicture}
\end{center}
\item (Linearly Independent Vectors)
A set of three vectors is linearly independent if the vectors do not lie in the same plane.  For example, vectors $\vec{i}$, $\vec{j}$ and $\vec{k}$ are linearly independent.
\end{itemize}

\section*{Practice Problems}

\begin{problem} Are the given vectors linearly independent?

\begin{problem}\label{prob:linindmultchoice1}
$$\begin{bmatrix}-1\\0\end{bmatrix}, \begin{bmatrix}2\\3\end{bmatrix},\begin{bmatrix}4\\-1\end{bmatrix}$$

\begin{multipleChoice}
 \choice{Yes}
  \choice[correct]{No }
 \end{multipleChoice}
\end{problem}

\begin{problem}\label{prob:linindmultchoice2}
$$\begin{bmatrix}1\\0\\5\end{bmatrix}, \begin{bmatrix}2\\2\\3\end{bmatrix},\begin{bmatrix}-1\\0\\1\end{bmatrix}$$

\begin{multipleChoice}
 \choice[correct]{Yes}
  \choice{No }
 \end{multipleChoice}

\end{problem}

\begin{problem}\label{prob:linindmultchoice3}
$$\begin{bmatrix}3\\0\\5\end{bmatrix}, \begin{bmatrix}2\\0\\2\end{bmatrix},\begin{bmatrix}-1\\0\\-5\end{bmatrix}$$

\begin{multipleChoice}
 \choice{Yes}
  \choice[correct]{No }
 \end{multipleChoice}
\end{problem}

\begin{problem}\label{prob:linindmultchoice4}
$$\begin{bmatrix}3\\1\\4\\1\end{bmatrix}, \begin{bmatrix}-2\\1\\1\\1\end{bmatrix}$$

\begin{multipleChoice}
 \choice[correct]{Yes}
  \choice{No }
 \end{multipleChoice}
\end{problem}

\end{problem}

\begin{problem} True or False?
\begin{problem}\label{prob:TFlinind1}
Any set containing the zero vector is linearly dependent.
\begin{multipleChoice}
 \choice[correct]{TRUE}
  \choice{FALSE}
 \end{multipleChoice}
 \end{problem}
\begin{problem}\label{prob:TFlinind2}
A set containing exactly one nonzero vector is linearly dependent.
\begin{multipleChoice}
 \choice{TRUE}
  \choice[correct]{FALSE}
 \end{multipleChoice}
\end{problem}

\end{problem}

\begin{problem}
Each problem below provides information about vectors $\vec{v}_1, \vec{v}_2, \vec{v}_3$.  If possible, determine whether the vectors are linearly dependent or independent.

\begin{problem}\label{prob:linindmultchoice5}
$$0\vec{v}_1+ 0\vec{v}_2+ 0\vec{v}_3=\vec{0}$$
\begin{multipleChoice}
 \choice{The vectors are linearly independent}
  \choice{The vectors are linearly dependent }
  \choice[correct]{There is not enough information given to make a determination }
 \end{multipleChoice}
\end{problem}

\begin{problem}\label{prob:linindmultchoice6}
$$3\vec{v}_1+ 4\vec{v}_2- \vec{v}_3=\vec{0}$$
\begin{multipleChoice}
 \choice{The vectors are linearly independent}
  \choice[correct]{The vectors are linearly dependent }
  \choice{There is not enough information given to make a determination }
 \end{multipleChoice}
\end{problem}

\begin{problem}\label{prob:linindmultchoice7}
$$2\vec{v}_1+ 0\vec{v}_2+ 0\vec{v}_3=\vec{0}$$
\begin{multipleChoice}
 \choice{The vectors are linearly independent}
  \choice[correct]{The vectors are linearly dependent }
  \choice{There is not enough information given to make a determination }
 \end{multipleChoice}
\end{problem}

\end{problem}

\begin{problem}\label{prob:lindeplincombofother}
Prove Theorem \ref{th:lindeplincombofother}.
\begin{hint}
This is an ``if and only if" statement.  First, assume that the set is linearly dependent and show that one vector can be expressed as a linear combination of the others.  Second, assume that one vector can be expressed as a linear combination of the others, then shown that the set is linearly dependent.
\end{hint}
\end{problem}

\begin{problem}
Each diagram below shows a collection of vectors.  Are the vectors linearly dependent or independent?

\begin{problem}\label{prob:linindmultchoice8}
\begin{multipleChoice}
 \choice{The vectors are linearly independent}
  \choice[correct]{The vectors are linearly dependent }
  \choice{There is not enough information given to make a determination }
 \end{multipleChoice}
\begin{center}
\tdplotsetmaincoords{70}{130}
\begin{tikzpicture}
	\draw[->](-2,0,0)--(3,0,0) node[below left]{$y$};
    \draw[->](0,-2,0)--(0,3,0) node[below left]{$z$};
    \draw[->](0,0,-2)--(0,0,3) node[below left]{$x$};
    \draw[->, line width=2pt,blue, -stealth](0,0,0)--(0,2,1);
    \draw[->, line width=2pt,red, -stealth](0,0,0)--(1,2,0);
    \draw[->, line width=2pt, -stealth](0,0,0)--(0.5,1,0);
    
\end{tikzpicture}
\end{center}
\end{problem}

\begin{problem}\label{prob:linindmultchoice9}
\begin{multipleChoice}
 \choice{The vectors are linearly independent}
  \choice[correct]{The vectors are linearly dependent }
  \choice{There is not enough information given to make a determination }
 \end{multipleChoice}
\begin{center}
\begin{tikzpicture}[scale=0.8]
  \draw[<->] (-2,0)--(2,0);
  \draw[<->] (0,-2)--(0,2);
  
  \draw[line width=2pt,blue,-stealth](0,0)--(1.5,1.5);
  \draw[line width=2pt,-stealth](0,0)--(1,-1);
  \draw[line width=2pt,red,-stealth](0,0)--(0,1.5);

 \end{tikzpicture}
\end{center}
\end{problem}

\begin{problem}\label{prob:linindmultchoice10}
\begin{multipleChoice}
 \choice[correct]{The vectors are linearly independent}
  \choice{The vectors are linearly dependent }
  \choice{There is not enough information given to make a determination }
 \end{multipleChoice}
\begin{center}
\begin{tikzpicture}[scale=0.8]
  \draw[<->] (-2.5,0)--(2.5,0);
  \draw[<->] (0,-1)--(0,2.5);
  
  \draw[line width=2pt,blue,-stealth](0,0)--(2,2);
  \draw[line width=2pt,red,-stealth](0,0)--(-1,1);

 \end{tikzpicture}
\end{center}
\end{problem}


\begin{problem}\label{prob:linindmultchoice11}
\begin{multipleChoice}
 \choice{The vectors are linearly independent}
  \choice[correct]{The vectors are linearly dependent }
  \choice{There is not enough information given to make a determination }
 \end{multipleChoice}
\begin{center}
\tdplotsetmaincoords{70}{130}
\begin{tikzpicture}
	\draw[->](-2,0,0)--(2,0,0) node[below left]{$y$};
    \draw[->](0,-2,0)--(0,2,0) node[below left]{$z$};
    \draw[->](0,0,-2)--(0,0,2) node[below left]{$x$};
    \draw[->, line width=2pt,blue, -stealth](0,0,0)--(0,-1,0);
    \draw[->, line width=2pt,red, -stealth](0,0,0)--(1,0,0);
    \draw[->, line width=2pt,orange, -stealth](0,0,0)--(0,1,0);
    
\end{tikzpicture}
\end{center}
\end{problem}

\end{problem}
\end{document} 