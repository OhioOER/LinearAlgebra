\documentclass{ximera}
%% You can put user macros here
%% However, you cannot make new environments

\graphicspath{{./}{firstExample/}{secondExample/}}

\usepackage{tikz}
\usepackage{tkz-euclide}
\usetkzobj{all}
\pgfplotsset{compat=1.7} % prevents compile error.

\tikzstyle geometryDiagrams=[ultra thick,color=blue!50!black]


\author{Anna Davis \and Paul Zachlin \and Paul Bender} \title{VSP-0035: Bases and Dimension} \license{CC-BY-NC-SA}



\begin{document}
\begin{abstract}
We discuss existence of bases of $\RR^n$ and subspaces of $\RR^n$, and define dimension.
\end{abstract}
\maketitle

\section*{VSP-0035: Bases and Dimension}
Recall that a \dfn{basis} of a subspace $V$ of $\RR^n$ is a subset of $V$ that is linearly independent and spans $V$.  A basis allows us to uniquely express every element of $V$ as a linear combination of the elements of the basis.  Several questions may come to mind at this time.  Does every subspace of $\RR^n$ have a basis?  We know that bases are not unique.  If there is more than one basis, what, if anything, do they have in common?

\subsection*{Exploring Dimension}
\begin{exploration}\label{init:dimensionintro}
How would you describe 
$$V=\mbox{span}\left(\begin{bmatrix}1\\-2\\3\end{bmatrix}, \begin{bmatrix}-2\\4\\-6\end{bmatrix}\right)?$$
If you answered that $V$ is a line in $\RR^3$, you are correct.  While the two vectors span the line, it is not necessary to have both of them in the spanning set to describe the line. 

What is the minimum number of vectors needed to span a line?  

Answer: $\answer{1}$.

Observe also that the vectors in the given spanning set are not linearly independent, so they do not form a basis for $V$.  How many vectors would a basis for $V$ have?

Answer: $\answer{1}$.

Now consider another subspace of $\RR^3$:
$$W=\mbox{span}\left(\begin{bmatrix}1\\0\\2\end{bmatrix}, \begin{bmatrix}0\\-3\\0\end{bmatrix}\right)$$
Geometrically, $W$ is a plane in $\RR^3$.  Note that the vectors in the spanning set are linearly independent.  Can we remove one of the vectors and have the remaining vector span the plane?  

What is the minimum number of vectors needed to span a plane?

Answer: $\answer{2}$.

How many vectors would a basis for a plane have?

Answer: $\answer{2}$.
\end{exploration}

Our observations in Exploration \ref{init:dimensionintro} hint at the idea of dimension.  We know that a line is a one-dimensional object, a plane is a two-dimensional object, and the space we reside in is three-dimensional.  

Based on our observations in Exploration \ref{init:dimensionintro}, it makes sense for us to define dimension of a vector space (or a subspace) as the minimum number of vectors required to span the space (subspace).  In other words, we would like to define dimension as the number of elements in a basis.

We have to proceed carefully because we don't want the dimension to depend on our choice of a basis.  So, before we state our definition, we need to make sure that every basis for a given vector space (or subspace) has the same number of elements.

\begin{theorem}\label{th:dimwelldefined}
Suppose $\mathcal{B}=\{\vec{v}_1, \vec{v}_2,\ldots ,\vec{v}_t\}$ and $\mathcal{C}=\{\vec{w}_1, \vec{w}_2,\ldots ,\vec{w}_s\}$ be two bases of $\RR^n$ (or a subspace $V$ of $\RR^n$).  Then $s=t$.
\end{theorem}
\begin{proof}
Suppose $s\neq t$.  Without loss of generality, assume that $t<s$.  Because $\mathcal{B}$ spans $V$, every $\vec{w}_i$ of $\mathcal{C}$ can be written as a linear combination of elements of $\mathcal{B}$:
$$\vec{w}_i=a_{1i}\vec{v}_1+a_{2i}\vec{v}_{2}+\ldots +a_{ti}\vec{v}_t$$

Consider the vector equation
\begin{align}\label{eq:sizeofbases}
b_1\vec{w}_1+b_2\vec{w}_2+\ldots +b_s\vec{w}_s=\vec{0}
\end{align}

By substitution, we have:
\begin{align*}
    &b_1\vec{w}_1+b_2\vec{w}_2+\ldots +b_s\vec{w}_s=\\ \\
     =&b_1(a_{11}\vec{v}_1+a_{21}\vec{v}_{2}+\ldots +a_{t1}\vec{v}_t)+b_2(a_{12}\vec{v}_1+a_{22}\vec{v}_{2}+\ldots +a_{t2}\vec{v}_t)+\ldots\\
     &+b_s(a_{1s}\vec{v}_1+a_{2s}\vec{v}_{2}+\ldots +a_{ts}\vec{v}_t)\\ \\
     =&(b_1a_{11}+b_2a_{12}+\ldots +b_sa_{1s})\vec{v}_1
+(b_1a_{21}+b_2a_{22}+\ldots +b_sa_{2s})\vec{v}_2+
\ldots \\
&+(b_1a_{t1}+b_2a_{t2}+\ldots +b_sa_{ts})\vec{v}_t\\ \\
=&\vec{0}
\end{align*}

Because $\vec{v}_j$'s are linearly independent, we must have
$$b_1a_{j1}+b_2a_{j2}+\ldots +b_sa_{js}=0$$
For all $1\leq j\leq t$.
This gives us a system of $t$ equations and $s$ unknowns.  We can write the system as a matrix equation.
$$\begin{bmatrix}a_{11}&a_{12}&\ldots &a_{1s}\\a_{21}&a_{22}&\ldots &a_{2s}\\\vdots&\vdots&\ddots&\vdots\\a_{t1}&a_{t2}&\ldots&a_{ts}\end{bmatrix}\begin{bmatrix}b_1\\b_2\\\vdots\\b_s\end{bmatrix}=\vec{0}$$

Recall our assumption that $s<t$.  By Theorem \ref{th:rankandsolutions} of SYS-0030 we know that the system has infinitely many solutions.  This shows that equation (\ref{eq:sizeofbases}) has a nontrivial solution.  But this shows that $\{\vec{w}_1, \vec{w}_2,\ldots ,\vec{w}_s\}$ is linearly dependent and contradicts our assumption that $\mathcal{C}$ is a basis of $V$.  We conclude that $s=t$.
\end{proof}


\begin{definition}\label{def:dimension}
Let $V$ be a subspace of $\RR^n$.  The \dfn{dimension} of $V$ is the number, $m$, of elements in any basis of $V$.  We write
$$\mbox{dim}(V)=m$$
\end{definition}

\begin{example}\label{ex:basisofrn}
We know that vectors $\vec{e}_1, \ldots ,\vec{e}_n$ form a basis of $\RR^n$.  Therefore $\mbox{dim}(\RR^n)=n$.
\end{example}

The following section will guarantee that dimension is defined for every subspace of $\RR^n$.

\subsection*{Every Subspace of $\RR^n$ has a Basis}
%We know that $\RR^n$ has a standard basis with $n$ elements.  Therefore all bases of $\RR^n$ have $n$ elements.  Now we turn our attention to subspaces of $\RR^n$.  

\begin{lemma}\label{lemma:atmostnlinindinrn}
If a linearly independent subset of $\RR^n$ contains $m$ vectors, then $m\leq n$.
\end{lemma}
\begin{proof}
See Practice Problem \ref{prob:atmostnlinindinrnproof}.
\end{proof}

\begin{lemma}\label{lemma:expandinglinindset}
Let $\{\vec{v}_1,\ldots ,\vec{v}_k\}$ be a linearly independent subset of $\RR^n$.  If $\vec{u}$ is not in $\mbox{span}(\vec{v}_1,\ldots ,\vec{v}_k)$, then $\{\vec{u},\vec{v}_1,\ldots ,\vec{v}_k\}$ is linearly independent.
\end{lemma}
\begin{proof}
Consider the equation
\begin{align}\label{eq:expandinglinindset}a\vec{u}+a_1\vec{v}_1+\ldots +a_k\vec{v}_k=\vec{0}\end{align}
We need to show that $a=a_1=\ldots =a_k=0$.  Suppose $a\neq 0$, then $\vec{u}=\frac{-a_1}{a}\vec{v}_1+\ldots +\frac{-a_k}{a}\vec{v}_k$.  But this contradicts the assumption that $\vec{u}$ is not in the span of $\vec{v}_1,\ldots ,\vec{v}_k$.  So, $a=0$.  But $a_1=\ldots =a_k=0$ because $\vec{v}_1,\ldots ,\vec{v}_k$ are linearly independent.

This means that (\ref{eq:expandinglinindset}) has only the trivial solution, and $\{\vec{u},\vec{v}_1,\ldots ,\vec{v}_k\}$ is linearly independent.
\end{proof}

\begin{theorem}\label{th:expandtobasis}
Let $V$ be a subspace of $\RR^n$.  Any linearly independent subset of $V$ can be expanded to a basis of $V$.
\end{theorem}
\begin{proof}
Suppose that $I=\{\vec{v}_1,\ldots ,\vec{v}_k\}$ is a linearly independent subset of $V$. If $\mbox{span}(I) = V$ then $I$ is already a basis of $V$. If $\mbox{span}(I) \neq V$, choose $\vec{u}_1$ in $V$ such that $\vec{u}_1$ is not in $\mbox{span}(I)$. The set $\{\vec{u}_1, \vec{v}_1,\ldots ,\vec{v}_k\}$ is linearly independent by Lemma \ref{lemma:expandinglinindset}. 

If $\mbox{span}(\vec{u}_1, \vec{v}_1,\ldots ,\vec{v}_k) = V$ we are done; otherwise choose $\vec{u}_{2} \in V$ such that $\vec{u}_{2}$ is not in $\mbox{span}(\vec{u}_1, \vec{v}_1,\ldots ,\vec{v}_k)$. Then $\{\vec{u}_1,\vec{u}_2, \vec{v}_1,\ldots ,\vec{v}_k\}$ is linearly independent, and the process continues. We claim that a basis of $V$ will be reached eventually. If no basis of $V$ is ever reached, the process creates arbitrarily large independent sets in $\RR^n$. But this is impossible by Lemma \ref{lemma:atmostnlinindinrn}.
\end{proof}

\section*{Practice Problems}
\begin{problem}\label{prob:atmostnlinindinrnproof}
Prove Lemma \ref{lemma:atmostnlinindinrn}.
\begin{hint}
Look at the proof of Theorem \ref{th:dimwelldefined}.
\end{hint}
\end{problem}

\begin{problem}
For each given set $S$ of vectors, find $\mbox{dim}(\mbox{span}(S))$.
\begin{hint}
Use Theorem \ref{th:linindandrank} of VEC-0110.
\end{hint}
\begin{problem}\label{prob:finddimension1}
$$S=\left\{\begin{bmatrix}1\\1\\0\\1\end{bmatrix}, \begin{bmatrix}0\\1\\1\\1\end{bmatrix}, \begin{bmatrix}1\\0\\1\\1\end{bmatrix}, \begin{bmatrix}1\\1\\0\\1\end{bmatrix} \right\}$$
Answer: $\mbox{dim}(\mbox{span}(S))=\answer{3}$
\end{problem}

\begin{problem}\label{prob:finddimension2}
$$S=\left\{\begin{bmatrix}3\\-2\\1\\1\end{bmatrix}, \begin{bmatrix}2\\3\\3\\-2\end{bmatrix}, \begin{bmatrix}1\\-5\\-2\\3\end{bmatrix}\right\}$$
Answer: $\mbox{dim}(\mbox{span}(S))=\answer{2}$
\end{problem}

\begin{problem}\label{prob:finddimension3}
$$S=\left\{\begin{bmatrix}1\\1\\-3\end{bmatrix}, \begin{bmatrix}-3\\2\\1\end{bmatrix}, \begin{bmatrix}5\\-2\\4\end{bmatrix}\right\}$$
Answer: $\mbox{dim}(\mbox{span}(S))=\answer{3}$
\end{problem}


\end{problem}

\begin{problem}\label{prob:matrixtimesbasisvectors} 
Let $\mathcal{B}=\{\vec{v}_1, \vec{v}_2, \vec{v}_3\}$ be a basis of $\RR^3$.  Suppose $A$ is a nonsingular $3\times 3 $ matrix.  Show that $\mathcal{C}=\{A\vec{v}_1, A\vec{v}_2, A\vec{v}_3\}$ is also a basis of $\RR^3$. 
\begin{hint}
To show that $\mathcal{C}$ spans $\RR^3$, express $A^{-1}\vec{v}$ as a linear combination of $\vec{v}_1$, $\vec{v}_2$ and $\vec{v}_3$.
\end{hint}

\end{problem}

\section*{Exercise Source}
Practice Problem \ref{prob:matrixtimesbasisvectors} is adopted from Keith Nicholson's Example 5.2.12  \href{https://open.umn.edu/opentextbooks/textbooks/linear-algebra-with-applications}{\it Linear Algebra with Applications}. (CC-BY-NC-SA)

W. Keith Nicholson, {\it Linear Algebra with Applications}, Lyryx 2018, Open Edition, p 277.

\end{document}
