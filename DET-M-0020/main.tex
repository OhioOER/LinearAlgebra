\documentclass{ximera}
%% You can put user macros here
%% However, you cannot make new environments

\graphicspath{{./}{firstExample/}{secondExample/}}

\usepackage{tikz}
\usepackage{tkz-euclide}
\usetkzobj{all}
\pgfplotsset{compat=1.7} % prevents compile error.

\tikzstyle geometryDiagrams=[ultra thick,color=blue!50!black]


\author{Anna Davis \and Paul Zachlin \and Rosemarie Emanuele \and Paul Bender} \title{DET-0020: Definition of the Determinant -- Expansion Along the First Column} \license{CC-BY 4.0}

\begin{document}

\begin{abstract}
 We define the determinant of a square matrix in terms of cofactor expansion along the first column, and show that this definition is equivalent to the definition in terms of cofactor expansion along the first row.
\end{abstract}
\maketitle

\section*{DET-0020: Definition of the Determinant -- Expansion Along the First Column}
In DET-0010 we described the determinant as a function that assigns a scalar to every square matrix.  The value of the function is given by cofactor expansion along the first row of the matrix.  In this module we will mimic this process, but expand along the first column instead of the first row.  Surprisingly, our new approach to will yield the same result as the original definition.  We will conclude this module by proving that the two expansions will produce the same result for any $n\times n$ matrix. This will allow us to state an alternative definition of the determinant in terms of cofactor expansion along the first column.  


We will begin by revisiting Examples \ref{ex:threebythreedet1} and \ref{ex:expansiontoprow} from DET-0010.      

\begin{exploration}\label{init:expansionfirstcol1}
Let 
$$A=\begin{bmatrix}3&-2&1\\5&-1&2\\1&4&1\end{bmatrix}$$
In Example \ref{ex:threebythreedet1} we found that $\det{A}=0$.  Let's try to mimic what we did earlier, but instead of doing cofactor expansion along the first row, we will do the expansion along the fist column.  We will form each minor matrix by picking an entry from the first column, deleting the first column and deleting the row of the chosen entry.  We will also be following the same alternating sign pattern as before.
\begin{align*}
&(3)\begin{vmatrix}-1&2\\4&1\end{vmatrix}-(5)\begin{vmatrix}-2&1\\4&1\end{vmatrix}+(1)\begin{vmatrix}-2&1\\-1&2\end{vmatrix}\\
&=(3)(-1-8)-(5)(-2-4)+(1)(-4+1)\\
&=-27+30-3\\
&=0\\
&=\det{A}
\end{align*}
\end{exploration}
Let's go through this process again for a larger matrix.

\begin{exploration}\label{init:expansionfirstcol2}
Let
$$A=\begin{bmatrix}4&-1&2&1\\3&0&1&-2\\
2&1&5&1\\-2&1&3&-1\end{bmatrix}$$
In Example \ref{ex:expansiontoprow} we found that $\det{A}=-58$.  We will now try to expand along the fist column.  

When computing determinants of the four $3\times 3$ matrices below, try different approaches.  You might want to expand along the first row for some of them, and along the first column for others.  Looking for where zeros are located will help you decide what to try.
\begin{align*}
&4\begin{vmatrix}0&1&-2\\1&5&1\\1&3&-1\end{vmatrix}-3\begin{vmatrix}-1&2&1\\1&5&1\\1&3&-1\end{vmatrix}+2\begin{vmatrix}-1&2&1\\0&1&-2\\1&3&-1\end{vmatrix}-(-2)\begin{vmatrix}-1&2&1\\0&1&-2\\1&5&1\end{vmatrix}\\
&=4(\answer{6})-3(\answer{10})+2(\answer{-10})+2(\answer{-16})\\
&=\answer{-58}\\
&=\det{A}
\end{align*}

\end{exploration}

According to our current definition (Definition \ref{def:toprowexpansion} of DET-0010), we compute the determinant by doing cofactor expansion along the first row, as follows:

  Let $A=\begin{bmatrix}a_{ij}\end{bmatrix}$ be an $n\times n$ matrix.  Define the \dfn{determinant} of $A$ by
\begin{align*}\det{A}=(-1)^{1+1}a_{11}\det{A_{11}}&+(-1)^{1+2}a_{12}\det{A_{12}}+\ldots \\
\ldots &+(-1)^{1+j}a_{1j}\det{A_{1j}}+\ldots \\
\ldots &+(-1)^{1+n}a_{1n}\det{A_{1n}}\\
=\sum_{j=1}^n(-1)^{1+j}a_{1j}\det{A_{1j}}
\end{align*}
or
$$\det{A}=a_{11}C_{11}+a_{12}C_{12}+\ldots +a_{1n}C_{1n}=\sum_{j=1}^n a_{1j}C_{1j}$$


This definition uses \dfn{minor matrix} ($A_{1j}$) and \dfn{cofactor} ($C_{1j}$) notation.  Let's take a look at how this notation can accommodate for expansion along the first column.

Let $A$ be an $n\times n$ matrix. 
$$A=\begin{bmatrix}a_{11} & a_{12} & \dots  & a_{1n}  \\
    a_{21} & a_{22} &\dots  & a_{2n}  \\
   \vdots & \vdots &  & \vdots \\
   a_{i1} & a_{i2} & \dots  & a_{in}\\
   \vdots & \vdots &  & \vdots  \\
   a_{n1} & a_{n2} & \dots  & a_{nn}\end{bmatrix}$$
   Define $A_{i1}$ to be an $(n-1)\times (n-1)$ matrix obtained from $A$ by deleting the first column and the $i^{th}$ row of $A$.  We say that $A_{i1}$ is the \dfn{$(i, 1)$-minor} of $A$.
\begin{center}
\begin{tikzpicture}
  \matrix (m)[
    matrix of math nodes,
    nodes in empty cells,
    left delimiter={[},right delimiter={]},minimum width=width("a22")] {
    a_{11} & a_{12} & \dots  & a_{1n}  \\
    a_{21} & a_{22} &\dots  & a_{2n}  \\
   \vdots & \vdots &  & \vdots \\
   a_{i1} & a_{i2} & \dots  & a_{in}\\
   \vdots & \vdots &  & \vdots  \\
   a_{n1} & a_{n2} & \dots  & a_{nn}\\
  } ;
\draw (m-1-2.north west) rectangle (m-3-4.south east);
\draw (m-6-2.south west) rectangle (m-5-4.north east);
 \end{tikzpicture}
 \end{center} 
Define $C_{i1}=(-1)^{i+1}\det{A_{i1}}$ to be the
 \dfn{$(i,1)$-cofactor of $A$}.

We are now ready to propose an alternative definition of the determinant in terms of cofactor expansion along the first column.  Keep in mind that at this point we have not proved that the two definitions will always produce the same result.  We will prove that the two definitions are equivalent in the next section.

\begin{definition}\label{def:firstcolexpansion1}  Let $A=\begin{bmatrix}a_{ij}\end{bmatrix}$ be an $n\times n$ matrix.  Define the \dfn{determinant} of $A$ by
\begin{align*}\det{A}=(-1)^{1+1}a_{11}\det{A_{11}}&+(-1)^{2+1}a_{21}\det{A_{21}}+\ldots \\
\ldots &+(-1)^{i+1}a_{i1}\det{A_{i1}}+\ldots \\
\ldots &+(-1)^{n+1}a_{n1}\det{A_{n1}}\\
=\sum_{i=1}^n(-1)^{i+1}a_{i1}\det{A_{i1}}
\end{align*}
or
$$\det{A}=a_{11}C_{11}+a_{21}C_{21}+\ldots +a_{n1}C_{n1}=\sum_{i=1}^n a_{i1}C_{i1}$$
\end{definition}

\subsection*{Proof of Definition Equivalence}
We will now show that cofactor expansion along the first row produces the same result as cofactor expansion along the first column.
\begin{theorem}\label{th:rowcolexpequivalence}
Let $A=\begin{bmatrix}a_{ij}\end{bmatrix}$ be an $n\times n$ matrix.  Then
$$\sum_{j=1}^n(-1)^{1+j}a_{1j}\det{A_{1j}}=\sum_{i=1}^n(-1)^{i+1}a_{i1}\det{A_{i1}}$$
\end{theorem}
\begin{proof}
We will proceed by induction on $n$.  Clearly, the result holds for $n=1$.  Just for practice you should also verify the equality for $n=2, 3$. (See Practice Problem \ref{prob:extrainductionsteps}.)  We will assume that the result holds for $(n-1)\times (n-1)$ matrices and show that it must hold for $n\times n$ matrices.

You will find the following matrix a useful reference as we proceed.

$$A=\begin{bmatrix}a_{11} & a_{12} & \dots &  a_{1j} &  \dots & a_{1n}  \\
    a_{21} & a_{22} & \dots &  a_{2j} &  \dots & a_{2n}  \\
   \vdots & \vdots &  & \vdots &   & \vdots  \\
  a_{i1} & a_{i2} & \ldots &  a_{ij} &  \ldots & a_{in}\\
  \vdots & \vdots &  & \vdots &   & \vdots  \\
   a_{n1} & a_{n2} & \dots &  a_{nj} &  \dots & a_{nn}\end{bmatrix}$$
   For convenience, we will refer to the Right Hand Side (RHS) and the Left Hand Side (LHS) of the equality we are trying to prove.
   $$\text{LHS}=\sum_{j=1}^n(-1)^{1+j}a_{1j}\det{A_{1j}}\overset{?}{=}\sum_{i=1}^n(-1)^{i+1}a_{i1}\det{A_{i1}}=\text{RHS}$$
   
   Note that the first term $a_{11}(-1)^{1+1}\det{A_{11}}$ is the same for LHS and RHS, so we will only need to consider $i,j\geq 2$.
   
We will start by analyzing RHS.  Consider an arbitrary entry $a_{i1}$ of the fist column.  This entry will only appear in the term $a_{i1}(-1)^{i+1}\det{A_{i1}}$. We will find $\det{A_{i1}}$ by cofactor expansion along the first row.  As we proceed, we have to pay special attention to the subscripts.  Because the first column of $A$ was removed, the $j^{th}$ column of $A$ contains the $(j-1)$ column of $A_{i1}$.  
\begin{align*}
&a_{i1}(-1)^{i+1}\det{A_{i1}}=\\
=&a_{i1}(-1)^{i+1}\Big(a_{12}(-1)^{1+1}\det{(A_{i1})_{11}}+\ldots +a_{1j}(-1)^{1+(j-1)}\det{(A_{i1})_{1(j-1)}}+\ldots \Big)
\end{align*}
Note that the entry $a_{1j}$ will only appear in the term $$a_{1j}(-1)^{1+(j-1)}\det{(A_{i1})_{1(j-1)}}$$
So, after we distribute $a_{i1}(-1)^{i+1}$, RHS will contain only one term of the form
$$a_{i1}a_{1j}(-1)^{p+q}\det{(A_{st})_{pq}}$$
We will perform a similar analysis on LHS.  Consider an arbitrary entry $a_{1j}$ of the fist row.  This entry will only appear in the term $a_{1j}(-1)^{1+j}\big(\det{A_{1j}}\big)$. Invoking the induction hypothesis, we will find $\det{A_{1j}}$ by cofactor expansion along the first column.
\begin{align*}
&a_{1j}(-1)^{1+j}\det{A_{1j}}=\\
=&a_{1j}(-1)^{1+j}\Big(a_{21}(-1)^{1+1}\det{(A_{1j})_{11}}+\ldots +a_{i1}(-1)^{(i-1)+1}\det{(A_{1j})_{(i-1)1}}+\ldots \Big)
\end{align*}
The entry $a_{i1}$ will only appear in the term $$a_{i1}(-1)^{(i-1)+1}\det{(A_{1j})_{(i-1)1}}$$
So, after we distribute $a_{1j}(-1)^{1+j}$, LHS will contain only one term of the form
$$a_{i1}a_{1j}(-1)^{p+q}\det{(A_{st})_{pq}}$$
But RHS also has only one term of this form.  We now need to show that these two terms are equal.  The two terms are
\begin{align*}a_{i1}(-1)^{i+1}a_{1j}(-1)^{1+(j-1)}\det{(A_{i1})_{1(j-1)}}=a_{i1}a_{1j}(-1)^{i+j+1}\det{(A_{i1})_{1(j-1)}}\end{align*}
and 
\begin{align*}a_{1j}(-1)^{1+j}a_{i1}(-1)^{(i-1)+1}\det{(A_{1j})_{(i-1)1}}=a_{i1}a_{1j}(-1)^{i+j+1}\det{(A_{1j})_{(i-1)1}}\end{align*}
Observe that $(A_{i1})_{1(j-1)}$ and $(A_{1j})_{(i-1)1}$ are the same matrix because both were obtained from matrix $A$ by deleting the first and the $i^{th}$ rows of $A$, and the first and the $j^{th}$ columns of $A$.  Therefore $\det{(A_{i1})_{1(j-1)}}=\det{(A_{1j})_{(i-1)1}}$.


We conclude that the terms of LHS and RHS match.  This establishes the desired equality.
\end{proof}

Now we know that cofactor expansion along the first row and cofactor expansion along the first column produce the same result, so either expansion can be used to find the determinant.  

\section*{Practice Problems}

\begin{problem} Compute the determinant of each matrix by cofactor expansion along the first row and by cofactor expansion along the first column.  Compare the results.
\begin{problem}\label{prob:rowcolexpansionequiv1}
$$A=\begin{bmatrix}-3&4&-1\\1&3&-2\\0&4&1\end{bmatrix}$$
Answer:
$$\det{A}=\answer{-41}$$
\end{problem}

\begin{problem}\label{prob:rowcolexpansionequiv2}
$$B=\begin{bmatrix}1&2&3\\4&5&6\\7&8&9\end{bmatrix}$$
Answer:
$$\text{det}(B)=\answer{0}$$
\end{problem}
\end{problem}

\begin{problem}\label{prob:det26} Let 
$$A=\begin{bmatrix}x&2&3\\0&8&7\\0&5&6\end{bmatrix}$$
Find $x$ such that $\det{A}=-26$.

Answer: $$x=\answer{-2}$$
\end{problem}

\begin{problem}\label{prob:dettranspose}
Let $$A=\begin{bmatrix}2&-1&0\\-3&1&2\\1&4&-1\end{bmatrix}$$
\begin{enumerate}
\item
Find $\det{A}$ and $\det{A^T}$.

Answer:
$$\det{A}=\answer{-17}\quad\text{and}\quad\det{A^T}=\answer{-17}$$
\item Formulate a conjecture about the relationship between the determinant of a matrix and the determinant of its transpose.
\item Prove your conjecture.
\end{enumerate}
\end{problem}

\begin{problem}\label{prob:detoftrimat}
A matrix is called \dfn{upper triangular} if all of its entries below the main diagonal are 0.  A matrix is called \dfn{lower triangular} if all of its entries above the main diagonal are 0. Upper and lower triangular matrices are collectively referred to as \dfn{triangular} matrices.

Use mathematical induction to prove that the determinant of a triangular matrix is equal to the product of its diagonal entries.

(Note: We will make use of this property extensively in Module DET-0030.)
\end{problem}

\begin{problem}\label{prob:extrainductionsteps}
Prove that Definition \ref{def:toprowexpansion} of DET-0010 and Definition \ref{def:firstcolexpansion1} give the same result for the determinant of $2\times 2$ and $3\times 3$ matrices. (This problem is referenced in Theorem \ref{th:rowcolexpequivalence}.)
\end{problem}
\end{document} 