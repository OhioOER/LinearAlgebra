\documentclass{ximera}
%% You can put user macros here
%% However, you cannot make new environments

\graphicspath{{./}{firstExample/}{secondExample/}}

\usepackage{tikz}
\usepackage{tkz-euclide}
\usetkzobj{all}
\pgfplotsset{compat=1.7} % prevents compile error.

\tikzstyle geometryDiagrams=[ultra thick,color=blue!50!black]


\author{Anna Davis \and Paul Zachlin \and Paul Bender} \title{LTR-0022: Linear Transformations of Abstract Vector Spaces} \license{CC-BY 4.0}

\begin{document}
\begin{abstract}
We define linear transformation for abstract vector spaces, and  illustrate the definition with examples. 
\end{abstract}
\maketitle

\section*{LTR-0022: Linear Transformations of Abstract Vector Spaces}

Recall that a transformation $T:\mathbb{R}^n\rightarrow \mathbb{R}^m$ is called a \dfn{linear transformation} if the following are true for all vectors ${\bf u}$ and ${\bf v}$ in $\mathbb{R}^n$, and scalars $k$.
\begin{equation*}
T(k{\bf u})= kT({\bf u})
\end{equation*}
\begin{equation*}
T({\bf u}+{\bf v})= T({\bf u})+T({\bf v})
\end{equation*}

We generalize this definition as follows.

\begin{definition}\label{def:lintransgeneral}
Let $V$ and $W$ be vector spaces. A transformation $T:V\rightarrow W$ is called a \dfn{linear transformation} if the following are true for all vectors ${\bf u}$ and ${\bf v}$ in $V$, and scalars $k$.
\begin{equation*}
T(k{\bf u})= kT({\bf u})
\end{equation*}
\begin{equation*}
T({\bf u}+{\bf v})= T({\bf u})+T({\bf v})
\end{equation*}
\end{definition}

\begin{example}\label{ex:abstvectsplintransM22}
Recall that $\mathbb{M}_{n,n}$ is the set of all $n\times n$ matrices.  In Example \ref{ex:setofmatricesvectorspace} of VSP-0050 we demonstrated that $\mathbb{M}_{n,n}$ together with operations of matrix addition and scalar multiplication is a vector space.

Let $T_Q:\mathbb{M}_{n,n}\rightarrow \mathbb{M}_{n,n}$ be a transformation defined by $T_Q(A)=QA$, where $Q$ is fixed $n\times n$ matrix.  Show that $T_Q$ is a linear transformation.
\begin{explanation}
We verify the linearity properties using properties of matrix-matrix and matrix-scalar multiplication.  (See Theorem \ref{th:propertiesofmatrixmultiplication} of MAT-0020.)  For $A$ and $B$ in $\mathbb{M}_{n,n}$ and a scalar $k$ we have:
$$T_Q(kA)=Q(kA)=k(QA)=kT_Q(A)$$
$$T_Q(A+B)=Q(A+B)=QA+QB=T_Q(A)+T_Q(B)$$
\end{explanation}
\end{example}

\begin{example}\label{ex:abstvecsplintrans2}
Recall that $\mathbb{P}^3$ is the set of polynomials of degree $3$ or less than $3$.  In Example \ref{ex:pnisavectorspace} of VSP-0050 we showed that $\mathbb{P}^3$ together with operations of polynomial addition and scalar multiplication is a vector space. 

Suppose $T:\RR^3\rightarrow\mathbb{P}^3$ is a linear transformation such that 
$$T(\vec{i})=1+x-2x^2+x^3$$
$$T(\vec{j})=x+2x^3$$
$$T(\vec{k})=3+x^3$$
Find the image of $\begin{bmatrix}1\\-2\\1\end{bmatrix}$ under $T$.
\begin{explanation}
\begin{align*}
    T\left(\begin{bmatrix}1\\-2\\1\end{bmatrix}\right)&=T(\vec{i}-2\vec{j}+\vec{k})=T(\vec{i})-2T(\vec{j})+T(\vec{k})\\
    &=(1+x-2x^2+x^3)-2(x+2x^3)+(3+x^3)\\
    &=4-x-2x^2-2x^3
\end{align*}
\end{explanation}
\end{example}

\begin{example}\label{ex:nonlinabstvectsp}
Let $T:\mathbb{M}_{3,3}\rightarrow \RR$ be a transformation such that $T(A)=\mbox{rank}(A)$.  Show that $T$ is not linear.
\begin{explanation}
To show that $T$ is not linear it suffices to find two matrices $A$ and $B$ such that $T(A+B)\neq T(A)+T(B)$.  Observe that if we pick $A$ and $B$ so that each has rank $3$ we would have $T(A)+T(B)=\mbox{rank}(A)+\mbox{rank}(B)=6$ while $T(A+B)=\mbox{rank}(A+B)\leq 3$.  Clearly  $T(A+B)\neq T(A)+T(B)$.  This argument is sufficient, but if we want to have a specific example, we can find one.
Let $$A=\begin{bmatrix}1&0&0\\0&1&0\\0&0&1\end{bmatrix} \quad\text{and}\quad B=\begin{bmatrix}-1&0&0\\0&1&0\\0&0&-1\end{bmatrix}$$
Then
$$T(A)=3\quad\text{and}\quad T(B)=3$$
$$T(A+B)=T\left(\begin{bmatrix}0&0&0\\0&2&0\\0&0&0\end{bmatrix}\right)=1$$
Thus, $1=T(A+B)\neq T(A)+T(B)=6$.
\end{explanation}
\end{example}

\subsection*{Coordinate Vectors}
Transformations that map vectors to their coordinate vectors will prove to be of great importance.  In this section we will prove that such transformations are linear and give several examples.

Recall that if $V$ is a vector space, and  $\mathcal{B}=\{\vec{v}_1, \ldots ,\vec{v}_n\}$ is a basis for $V$ then any vector $\vec{v}$ of $V$ can be written as a unique linear combination of the elements of $\mathcal{B}$.  In other words, $\vec{v}=a_1\vec{v}_1+\ldots +a_n\vec{v}_n$ for some scalars $a_1, \ldots ,a_n$.  The vector in $\RR^n$ whose components are the coefficients $a_1, \ldots ,a_n$  is said to be the \dfn{coordinate vector for $\vec{v}$ with respect to $\mathcal{B}$} and is denoted by $[\vec{v}]_{\mathcal{B}}$.  (Definition \ref{def:coordvector} of VSP-0060.)  

It turns out that the transformation $T:V\rightarrow \RR^n$ defined by $T(\vec{v})=[\vec{v}]_{\mathcal{B}}$ is linear.  Before we prove linearity of $T$, consider the following examples.

\begin{example}\label{ex:abstvectsplintranscoordvect1}
Consider $\mathbb{M}_{2,2}$.  Let $\mathcal{B}=\left\{\begin{bmatrix}1&0\\0&0\end{bmatrix}, \begin{bmatrix}0&1\\0&0\end{bmatrix}, \begin{bmatrix}0&0\\1&0\end{bmatrix}, \begin{bmatrix}0&0\\0&1\end{bmatrix}\right\}$ be a basis for $\mathbb{M}_{2,2}$.  (You should do a quick mental check that $\mathcal{B}$ is a legitimate basis.)  Define $T:\mathbb{M}_{2,2}\rightarrow \RR^4$ by $T(A)=[A]_{\mathcal{B}}$.  Find $T\left(\begin{bmatrix}-2&3\\1&-5\end{bmatrix}\right)$.
\begin{explanation}
We need to find the coordinate vector for $\begin{bmatrix}-2&3\\1&-5\end{bmatrix}$ with respect to $\mathcal{B}$.
$$\begin{bmatrix}-2&3\\1&-5\end{bmatrix}=-2\begin{bmatrix}1&0\\0&0\end{bmatrix}+ 3\begin{bmatrix}0&1\\0&0\end{bmatrix}+ \begin{bmatrix}0&0\\1&0\end{bmatrix}+ (-5)\begin{bmatrix}0&0\\0&1\end{bmatrix}$$
This gives us:
$$T\left(\begin{bmatrix}-2&3\\1&-5\end{bmatrix}\right)=\left[\begin{bmatrix}-2&3\\1&-5\end{bmatrix}\right]_{\mathcal{B}}=\begin{bmatrix}-2\\3\\1\\-5\end{bmatrix}$$
\end{explanation}
\end{example}

\begin{example}\label{ex:abstvectsplintranspoly}
Recall that $\mathbb{P}^2$ is the set of polynomials of degree $2$ or less than $2$.  In Example \ref{ex:deg_le_2vectorspace} of VSP-0050 we showed that $\mathbb{P}^2$ is a vector space. 
\begin{enumerate}
    \item \label{item:lintranspolycoordvect1} It is easy to verify that $\mathcal{B}_1=\{1, x, x^{2}\}$ is a basis for $\mathbb{P}^2$.  If $T:\mathbb{P}^2\rightarrow \RR^3$ is given by $T(p)=[p]_{\mathcal{B}_1}$, find $T(2x^2-3x)$.
    \item \label{item:lintranspolycoordvect2}
In Practice Problem \ref{prob:linindabstractvsp1} of VSP-0060, you demonstrated that $\mathcal{B}_2=\{1 + x, 1 - x, x + x^{2}\}$ is also a basis for $\mathbb{P}^2$.  If $T:\mathbb{P}^2\rightarrow \RR^3$ is given by $T(p)=[p]_{\mathcal{B}_2}$, find $T(2x^2-3x)$.
\end{enumerate}
\begin{explanation}
\ref{item:lintranspolycoordvect1}  We express $2x^2-3x$ as a linear combination of elements of $\mathcal{B}_1$.
$$2x^2-3x=0\cdot 1+ (-3)x+2x^2$$
Therefore $$[2x^2-3x]_{\mathcal{B}_1}=\begin{bmatrix}0\\-3\\2\end{bmatrix}$$
Note that it is important to keep the basis elements in the same order in which they are listed, as the order of components of the coordinate vector depends on the order of the basis elements.  We conclude that
$$T(2x^2-3x)=\begin{bmatrix}0\\-3\\2\end{bmatrix}$$

\ref{item:lintranspolycoordvect2} Our goal is to express $2x^2-3x$ as a linear combination of the elements of $\mathcal{B}_2$.  Thus, we need to find coefficients $a$, $b$ and $c$ such that
$$2x^2-3x=a(1+x)+b(1-x)+c(x+x^2)=(a+b)+(a-b+c)x+cx^2$$
This gives us a system of linear equations:
$$\begin{array}{ccccccc}
      a & +&b&&&= &0 \\
	 a& -&b&+&c&=&-3\\
     & &&&c&=&2
    \end{array}$$
    Solving the system yields $a=-\frac{5}{2}$, $b=\frac{5}{2}$ and $c=2$.  Thus
    $$T(2x^2-3x)=[2x^2-3x]_{\mathcal{B}_2}=\begin{bmatrix}-5/2\\5/2\\2\end{bmatrix}$$
\end{explanation}
\end{example}

\begin{theorem}\label{th:coordvectmappinglinear}
Let $V$ be an $n$-dimensional vector space, and let $\mathcal{B}$ be a basis for $V$.  Then  $T:V\rightarrow \RR^n$ given by $T(\vec{v})=[\vec{v}]_{\mathcal{B}}$ is a linear transformation.
\end{theorem}
\begin{proof}
First observe that Theorem \ref{th:uniquerep} of VSP-0060 guarantees that there is only one way to represent each element of $V$ as a linear combination of elements of $\mathcal{B}$.  Thus each element of $V$ maps to exactly one element of $\RR^n$, as long as the order in which elements of $\mathcal{B}$ appear is taken into account.  (The order of elements of $\mathcal{B}$ is important as it determines the order of components of the coordinate vectors.)  This proves that $T$ is a function, or a transformation.  We will now prove that $T$ is linear.

Let $\vec{v}$ be an element of $V$.  We will first show that $T(k\vec{v})=kT(\vec{v})$.  Suppose $\mathcal{B}=\{\vec{v}_1, \ldots ,\vec{v}_n\}$, then $\vec{v}$ can be written as a unique linear combination:
$$\vec{v}=a_1\vec{v}_1+ \ldots +a_n\vec{v}_n$$
We have:
\begin{align*}
    T(k\vec{v})&=T(k(a_1\vec{v}_1+ \ldots +a_n\vec{v}_n))\\
    &=T((ka_1)\vec{v}_1+ \ldots +(ka_n)\vec{v}_n)\\
    &=\begin{bmatrix}ka_1\\\vdots\\ka_n\end{bmatrix}=k\begin{bmatrix}a_1\\\vdots\\a_n\end{bmatrix}=kT(\vec{v})
\end{align*}
We leave it to the reader to verify that $T(\vec{v}+\vec{w})=T(\vec{v})+T(\vec{w})$.  (See Practice Problem \ref{prob:completeproofoflin}.)
\end{proof}

%\begin{center}
%\begin{tikzpicture}[scale=1]
%  \filldraw[orange](-0.25,3.5)--(0.25,3.5)--(1.5,0)--(-1.5,0)--cycle;
%  \filldraw[orange] (0,0) ellipse (2cm and 1cm);
%  \filldraw[orange] (0,3.5) ellipse (0.25cm and 0.15cm);
%\end{tikzpicture}

%UNDER CONSTRUCTION -- COMING SOON
%\end{center}

\section*{Practice Problems}
\begin{problem}\label{prob:lintransP2toM22}
Suppose $T:\mathbb{P}^2\rightarrow\mathbb{M}_{2,2}$ is a linear transformation such that 
$$T(1)=\begin{bmatrix}1&0\\0&1\end{bmatrix},\quad T(x)=\begin{bmatrix}1&1\\0&1\end{bmatrix},\quad T(x^2)=\begin{bmatrix}1&1\\1&1\end{bmatrix}$$
Find $T(4-x+3x^2)$.

Answer:$$T(4-x+3x^2)=\begin{bmatrix}\answer{6}&\answer{2}\\\answer{3}&\answer{6}\end{bmatrix}$$

\end{problem}

\begin{problem}
Define $T:\mathbb{M}_{3,3}\rightarrow \RR$ by $T(A)=\mbox{tr}(A)$.  (Recall that $\mbox{tr}(A)$ denotes the \dfn{trace} of $A$, which is the sum of the main diagonal entries of $A$.)

\begin{problem}\label{prob:tracelintrans1}
Find $T\left(\begin{bmatrix}1&2&3\\4&5&6\\7&8&9\end{bmatrix}\right)$

Answer: $$T\left(\begin{bmatrix}1&2&3\\4&5&6\\7&8&9\end{bmatrix}\right)=\answer{15}$$
\end{problem}

\begin{problem}\label{prob:tracelintrans2}
Is $T$ a linear transformation?  If so, prove it.  If not, give a counterexample.
\end{problem}
\end{problem}

\begin{problem}
Define $T:\RR^2\rightarrow\mathbb{M}_{2,2}$ by $T\left(\begin{bmatrix}a\\b\end{bmatrix}\right)=\begin{bmatrix}a&1\\1&b\end{bmatrix}$.

\begin{problem}\label{prob:lintransr2toM22part1}
Find $T\left(\begin{bmatrix}2\\-1\end{bmatrix}\right)$.

Answer: $$T\left(\begin{bmatrix}2\\-1\end{bmatrix}\right)=\begin{bmatrix}\answer{2}&\answer{1}\\\answer{1}&\answer{-1}\end{bmatrix}$$
\end{problem}
\begin{problem}\label{prob:lintransr2toM22part2}
Is $T$ a linear transformation?  If so, prove it.  If not, give a counterexample.
\end{problem}
\end{problem}

\begin{problem}  This problem requires the knowledge of how to compute a $3\times 3$ determinant. (For a quick reminder, see Definition \ref{def:threedetcrossprod} of VEC-0080.)
Define $T:\mathbb{M}_{3,3}\rightarrow \RR$ by $T(A)=\det(A)$.  

\begin{problem}\label{prob:detlintrans1}
Find $T\left(\begin{bmatrix}1&2&3\\4&5&6\\7&8&9\end{bmatrix}\right)$

Answer: $$T\left(\begin{bmatrix}1&2&3\\4&5&6\\7&8&9\end{bmatrix}\right)=\answer{0}$$
\end{problem}

\begin{problem}\label{prob:detlintrans2}
Is $T$ a linear transformation?  If so, prove it.  If not, give a counterexample.
\end{problem}
\end{problem}

\begin{problem}
Define $T:\mathbb{P}^3\rightarrow\mathbb{P}^2$ by $T(p(x))=p'(x)$.  (In other words, $T$ maps a polynomial to its derivative.)

\begin{problem}\label{prob:lintransderivative1}
Find $T(4x^3-2x^2+x+6)$.

Answer: $$T(4x^3-2x^2+x+6)=\answer{12}x^2-\answer{4}x+\answer{1}$$
\end{problem}

\begin{problem}\label{prob:lintransderivative2}
Is $T$ a linear transformation?  If so, prove it.  If not, give a counterexample.
\end{problem}
\end{problem}

\begin{problem}
Define $T:\mathbb{P}^2\rightarrow\mathbb{P}^3$ by $T(p(x))=xp(x)$.  

\begin{problem}\label{prob:lintransmultbyx1}
Find $T(-x^2+2x-4)$.

Answer: $$T(-x^2+2x-4)=\answer{-x^3+2x^2-4x}$$
\end{problem}

\begin{problem}\label{prob:lintransmultbyx2}
Is $T$ a linear transformation?  If so, prove it.  If not, give a counterexample.
\end{problem}
\end{problem}

\begin{problem}\label{prob:symmMatLinTrans}
Recall that the set $V$ of all symmetric $2\times 2$ matrices is a subspace of $\mathbb{M}_{2,2}$.  In Example \ref{ex:symmetricmatsubspace} of VSP-0060 we demonstrated that $\mathcal{B} = \left\{
\begin{bmatrix}
1 & 0 \\
0 & 0
\end{bmatrix}, \begin{bmatrix}
0 & 0 \\
0 & 1
\end{bmatrix}, \begin{bmatrix}
0 & 1 \\
1 & 0
\end{bmatrix}
\right\}$ is a basis for $V$.  Define $T:V\rightarrow \RR^3$ by $T(A)=[A]_{\mathcal{B}}$.  Find $T(I_2)$ and $T\left(\begin{bmatrix}2&-3\\-3&1\end{bmatrix}\right)$.

Answer:
$$T(I_2)=\begin{bmatrix}\answer{1}\\\answer{1}\\\answer{0}\end{bmatrix}$$
$$T\left(\begin{bmatrix}2&-3\\-3&1\end{bmatrix}\right)=\begin{bmatrix}\answer{2}\\\answer{1}\\\answer{-3}\end{bmatrix}$$
\end{problem}

\begin{problem}\label{prob:coordvector}
Let $V$ be a subspace of $\RR^3$ with a basis $\mathcal{B}=\left\{\begin{bmatrix}2\\1\\-1\end{bmatrix}, \begin{bmatrix}0\\3\\2\end{bmatrix}\right\}$.  Find the coordinate vector, $[\vec{v}]_{\mathcal{B}}$, for $\vec{v}=\begin{bmatrix}4\\-1\\-4\end{bmatrix}$.
$$[\vec{v}]_{\mathcal{B}}=\begin{bmatrix}\answer{2}\\\answer{-1}\end{bmatrix}$$
\end{problem}

\begin{problem}\label{prob:switchbasisorder}
If the order of the basis elements in Problem \ref{prob:coordvector} was switched to form a new basis
$$\mathcal{B}'=\left\{\begin{bmatrix}0\\3\\2\end{bmatrix}, \begin{bmatrix}2\\1\\-1\end{bmatrix} \right\}$$
How would this affect the coordinate vector?

$$[\vec{v}]_{\mathcal{B}'}=\begin{bmatrix}\answer{-1}\\\answer{2}\end{bmatrix}$$
\end{problem}



\begin{problem}\label{prob:polylintranscoordvect} In Practice Problem \ref{prob:linindabstractvsp123} of VSP-0060 you demonstrated that
$\mathcal{B}=\{x^{2}, x + 1, 1 - x - x^{2}\}$ is a basis for $\mathbb{P}^2$.  Define $T:\mathbb{P}^2\rightarrow \RR^3$ by $T(p(x))=[p(x)]_{\mathcal{B}}$.  Find $T(0)$, $T(x+1)$ and $T(x^2-3x+1)$.

Answer:
$$T(0)=\begin{bmatrix}\answer{0}\\\answer{0}\\\answer{0}\end{bmatrix}$$
$$T(x+1)=\begin{bmatrix}\answer{0}\\\answer{1}\\\answer{0}\end{bmatrix}$$
$$T(x^2-3x+1)=\begin{bmatrix}\answer{3}\\\answer{-1}\\\answer{2}\end{bmatrix}$$
\end{problem}

\begin{problem}\label{prob:completeproofoflin}
Complete the proof of Theorem \ref{th:coordvectmappinglinear}.
\end{problem}


\end{document}
