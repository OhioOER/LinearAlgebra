\documentclass{ximera}

\author{Anna Davis \and Paul Zachlin \and Rosemarie Emanuele} \title{Matrix Addition and Scalar Multiplication} \license{CC-BY 4.0}

%% You can put user macros here
%% However, you cannot make new environments

\graphicspath{{./}{firstExample/}{secondExample/}}

\usepackage{tikz}
\usepackage{tkz-euclide}
\usetkzobj{all}
\pgfplotsset{compat=1.7} % prevents compile error.

\tikzstyle geometryDiagrams=[ultra thick,color=blue!50!black]


\begin{document}

\begin{abstract}
  We introduce matrices, define matrix addition and scalar and prove properties of those operations.
\end{abstract}
\maketitle



\section*{Introduction to Matrices}

{\color{red} This section was adopted from Kuttler matricesMatrixArithmetic.tex need hyperlink ref.}

A \dfn{matrix} is a rectangular array of numbers. The plural form of \dfn{matrix} is \dfn{matrices}. For example, here is a matrix.

$$M=\begin{bmatrix}
1 & 2 & 3 & 4 \\
5 & 2 & 8 & 7 \\
6 & -9 & 1 & 2
\end{bmatrix}$$

The size, or dimension, of a matrix is defined as $m\times n$ where $m$ is the
number of rows and $n$ is the number of columns. The above matrix is a 
$3\times 4$ matrix because there are three rows and four columns.  

The familiar \dfn{column vector} in $\RR^n$ is an an $n\times 1$ matrix.  Sometimes we find it convenient to talk about \dfn{row vectors}.  A row vector in $\RR^n$ is a $1\times n$ matrix.  

The individual \dfn{entries} in the matrix are identified according to their position. The $( i, j)$-entry of a matrix is the entry 
in the $i^{th}$ row and $j^{th}$ column. For example, in matrix $M$ above,  $8$ is called the $(2,3)$-entry because it is in the second row and the third column. 

We denote the entry in the $i^{th}$ row  and the $j^{th}$ column of matrix $A$ by $a_{ij}$, and write $A$ in terms of its entries
as $$A= \begin{bmatrix} a_{ij} \end{bmatrix}=\begin{bmatrix}
           a_{11} & a_{12}&\dots&a_{1j}&\dots&a_{1n}\\
           a_{21}&a_{22} &\dots&a_{2j}&\dots &a_{2n}\\
		\vdots & \vdots&&\vdots&&\vdots\\
        a_{i1}&a_{i2}&\dots &a_{ij}&\dots &a_{in}\\
        \vdots & \vdots&&\vdots&&\vdots\\
		a_{m1}&a_{m2}&\dots &a_{mj}&\dots &a_{mn}
         \end{bmatrix}$$. 

We denote the $j^{th}$ column of a matrix $A$ 
by $A_{j}$ and write
$$A=\begin{bmatrix}|&|&&|\\A_1& A_2 &\ldots & A_n\\|&|&&|\end{bmatrix}\quad\text{or}\quad A=\begin{bmatrix}A_1& A_2 &\ldots & A_n\end{bmatrix}$$

A matrix $B=\begin{bmatrix}b_{ij}\end{bmatrix}$ which has size $n \times n$ is called a \dfn{square matrix}. 
In other words, $B$ is a square matrix if it has the same number of rows and columns.  In a square matrix, entries of the form $b_{ii}$ are said to lie on the \dfn{main diagonal}.  For example, if $$B=\begin{bmatrix}1&2&3\\4&5&6\\7&4&9\end{bmatrix}$$
then the main diagonal consists of entries $b_{11}=1$, $b_{22}=5$ and $b_{33}=9$.

There are various operations which are done on matrices of appropriate 
sizes. Matrices can be added to and subtracted from other matrices,
multiplied by a scalar, and multiplied by other matrices. We will
never divide a matrix by another matrix, but we will see later how matrix inverses play a similar role. 

In doing arithmetic with matrices, we often define the action by what
happens in terms of the entries (or components) of the
matrices. Before looking at these operations in depth, consider a few
general definitions.

\begin{definition}[The Zero Matrix]\label{def:zeromatrix}
The \dfn{$m\times n$ zero matrix} is the $m\times n$ matrix
having every entry equal to zero. The zero matrix is
denoted by ${\bf 0}$.
\end{definition}

\begin{definition}[Equality of Matrices]\label{def:equalityofmatrices}
 Let $A=\begin{bmatrix} a_{ij}\end{bmatrix}$ and $B=\begin{bmatrix} b_{ij}\end{bmatrix}$ be two $m \times n$ matrices. Then $A=B$ means
that $a_{ij}=b_{ij}$ for all $1\leq i\leq m$ and 
$1\leq j\leq n$.
\end{definition}

\section*{Addition of Matrices}

{\color{red} This section was adopted from Kuttler matricesMatrixArithmeticAddition.tex hyperlink ref needed}

Given two matrices of the same dimensions, we can add them together by adding their corresponding entries.
\begin{definition}[Addition of Matrices]\label{def:additionofmatrices}
Let $A=\begin{bmatrix} a_{ij}\end{bmatrix} $ and $B=\begin{bmatrix} b_{ij}\end{bmatrix}$ be two
$m\times n$ matrices. Then $A+B$  is an $m \times n$
matrix  given by 
$$A+B=\begin{bmatrix}a_{ij}+b_{ij}\end{bmatrix}$$

\end{definition}

\begin{example}\label{ex:samesizematrixaddition}
Find the sum of $A$ and $B$, if possible.
\begin{equation*}
A = \begin{bmatrix}
1 & 2 & 3 \\
1 & 0 & 4
\end{bmatrix},
B = \begin{bmatrix}
5 & 2 & 3 \\
-6 & 2 & 1
\end{bmatrix}
\end{equation*}
\begin{explanation}
Notice that both $A$ and $B$ are of size $2 \times 3$. 
Since $A$ and $B$ are of the same size, addition is possible. 
\begin{align*}
A + B &= \begin{bmatrix}
1 & 2 & 3 \\
1 & 0 & 4
\end{bmatrix}
+
\begin{bmatrix}
5 & 2 & 3 \\
-6 & 2 & 1
\end{bmatrix}\\
&=
\begin{bmatrix}
1+5 & 2+2 & 3+3 \\
1+ -6 & 0+2 & 4+1
\end{bmatrix}\\
&=
\begin{bmatrix}
6 & 4 & 6 \\
-5 & 2 & 5
\end{bmatrix}
\end{align*}
\end{explanation}
\end{example}

Going forward, whenever we write $A+B$ it will be assumed that the two matrices are of equal size and addition is possible.

\begin{theorem}[Properties of Matrix Addition]\label{th:propertiesofaddition}
Let $A,B$ and $C$ be matrices. Then, the following properties  hold. 

\begin{enumerate}
\item\label{item:mataddcomm} Commutative Law of Addition

$$A+B=B+A$$  


\item \label{item:mataddass} Associative Law of Addition

$$\left( A+B\right) +C=A+\left( B+C\right) $$


\item\label{item:mataddid} Additive Identity
\begin{center}
There exists a zero matrix such that
\end{center}
$$A+{\bf 0}=A$$


\item\label{item:mataddinv} Additive Inverse
\begin{center}
There exists a matrix, $-A$, such that
\end{center}
$$A+\left( -A\right) ={\bf 0} $$

\end{enumerate}
\end{theorem}
We will prove Properties \ref{item:mataddcomm} and \ref{item:mataddinv}.  The remaining properties are left as exercises.
\begin{proof}[Proof of Property~\ref{item:mataddcomm}:]
The $(i,j)$-entry of $A+B$ is given by

$$a_{ij}+b_{ij}$$

The $(i,j)$-entry of $B+A$ is given by
$$b_{ij}+a_{ij}$$

Since $a_{ij}+b_{ij}=b_{ij}+a_{ij}$, for all $i$, $j$, we conclude that $A+B=B+A$.
\end{proof}

\begin{proof}[Proof of Property~\ref{item:mataddinv}:]
Let $-A$ be defined by
$$-A=\begin{bmatrix}-a_{ij}\end{bmatrix}$$
Then $A+(-A)={\bf 0}$.
\end{proof}

You will recognize the zero matrix of Theorem \ref{th:propertiesofaddition}\ref{item:mataddid} as the zero matrix of Definition \ref{def:zeromatrix}. 

\section*{Scalar Multiplication of Matrices}

{\color{red} This section was adopted from Kuttler matricesMatrixArithmeticScalarMultiplication.tex hyperlink ref needed}

When a matrix is multiplied by a scalar, the new matrix is obtained by multiplying every entry of the original matrix
by the given scalar. 

\begin{definition}[Scalar Multiplication of Matrices]\label{def:scalarmultofmatrices}
If $A=\begin{bmatrix} a_{ij}\end{bmatrix} $ and $k$ is a scalar,
then $kA=\begin{bmatrix} ka_{ij}\end{bmatrix}$. 
\end{definition}

\begin{example}\label{ex:effectofscalarmult}
Find $7A$ if

$$A=\begin{bmatrix}
2 & 0 \\
1 & -4

\end{bmatrix}$$



\begin{explanation}
By Definition \ref{def:scalarmultofmatrices}, we multiply each entry of $A$ by $7$.
Therefore,
$$
7A = 
7\begin{bmatrix}

2 & 0 \\
1 & -4

\end{bmatrix} =
\begin{bmatrix}

7(2) & 7(0) \\
7(1) & 7(-4)

\end{bmatrix} =
\begin{bmatrix}

14 & 0 \\
7 & -28

\end{bmatrix}
$$
\end{explanation}
\end{example}


\begin{theorem}[Properties of Scalar Multiplication]\label{th:propertiesscalarmult}
Let $A, B$ be matrices, and $k, p$ be scalars. Then, the following properties properties of scalar multiplication hold.
\begin{enumerate}
\item\label{item:scalardistmatadd} Distributive Law over Matrix Addition
\begin{equation*}
k \left( A+B\right) =k A+ kB  
\end{equation*}

\item \label{item:matdistscalaradd}Distributive Law over Scalar Addition
\begin{equation*}
\left( k +p \right) A= k A+p A
\end{equation*}

\item \label{item:scalarmatmultass}Associative Law for Scalar Multiplication
\begin{equation*}
k \left( p A\right) = \left( k p \right) A 
\end{equation*}

\item\label{item:matmult1} Multiplication by $1$
\begin{equation*}
1A=A  
\end{equation*}
\end{enumerate}

\end{theorem}

The proof of this theorem is similar to the proof of Theorem \ref{th:propertiesofaddition} and is left as an exercise. 


\section*{Practice Problems}

\begin{problem} If 
$$A=\begin{bmatrix}2&-1\\3&4\end{bmatrix}\quad\text{and}\quad B=\begin{bmatrix}3&0\\-2&1\end{bmatrix}$$
then $$2(A+B)=\begin{bmatrix}\answer{10}&\answer{-2}\\\answer{2}&\answer{10}\end{bmatrix}$$

$$3A-2B=\begin{bmatrix}\answer{0}&\answer{-3}\\\answer{13}&\answer{10}\end{bmatrix}$$
\end{problem}





\begin{problem}
Prove properties \ref{item:mataddass} and  \ref{item:mataddid} of Theorem \ref{th:propertiesofaddition}.
\end{problem}

\begin{problem}
Prove Theorem \ref{th:propertiesscalarmult}.
\end{problem}


\end{document} 
