\documentclass{ximera}
%% You can put user macros here
%% However, you cannot make new environments

\graphicspath{{./}{firstExample/}{secondExample/}}

\usepackage{tikz}
\usepackage{tkz-euclide}
\usetkzobj{all}
\pgfplotsset{compat=1.7} % prevents compile error.

\tikzstyle geometryDiagrams=[ultra thick,color=blue!50!black]


\author{Paul Zachlin \and Anna Davis} \title{VSP-0060: Bases and Dimension for Abstract Vector Spaces} \license{CC-BY 4.0}
\begin{document}

\begin{abstract}
We revisit the definitions of linear independence, bases, and dimension in the context of abstract vector spaces.
\end{abstract}
\maketitle


\section*{VSP-0060: Bases and Dimension for Abstract Vector Spaces}
When working with $\RR^n$ and subspaces of $\RR^n$ we developed several fundamental ideas including \dfn{span}, \dfn{linear independence}, \dfn{bases} and \dfn{dimension}.  We will find that these concepts generalize easily to abstract vector spaces and that analogous results hold in these new settings.
\subsection*{Linear Independence}
\begin{definition}[Linear Independence]\label{def:linearindependenceabstract}
Let $V$ be a vector space.  Let $\vec{v}_1, \vec{v}_2,\ldots ,\vec{v}_p$ be vectors of $V$.  We say that the set $\{\vec{v}_1, \vec{v}_2,\ldots ,\vec{v}_p\}$ is \dfn{linearly independent} if the only solution to 
$$a_1\vec{v}_1+a_2\vec{v}_2+\ldots +a_p\vec{v}_p=\vec{0}$$
is the \dfn{trivial solution} $a_1=a_2=\ldots =a_p=0$.

If, in addition to the trivial solution, a \dfn{non-trivial solution} (not all $a_1, a_2,\ldots ,a_p$ are zero) exists, then we say that the set $\{\vec{v}_1, \vec{v}_2,\ldots ,\vec{v}_p\}$ is \dfn{linearly dependent}.
\end{definition}

\begin{example}\label{ex:polyindset} 
Show that $P=\{1 + x, 3x + x^{2}, 2 + x - x^{2}\}$ is independent in $\mathbb{P}^{2}$.

\begin{explanation}
Consider the linear combination equation
\begin{align*}
a(1 + x) + b(3x + x^2) + c(2 + x - x^2) &= 0\\
a+ax+3bx+bx^2+2c+cx-cx^2&=0\\
(a+2c)+(a+3b+c)x+(b-c)x^2&=0
\end{align*}
The constant term, as well as the coefficients in front of $x$ and $x^2$, must be equal to $0$.  This gives us the following system of equations.
\begin{equation*}
\begin{array}{rlrlrcr}
	a & + &      & + & 2c & = & 0 \\
	a & + & 3b & + &  c & = & 0 \\
	    &   &  b & - &  c & = & 0 \\
\end{array}
\end{equation*}
The only solution is $a = b = c = 0$.  We conclude that $P$ is linearly independent in $\mathbb{P}^2$.
\end{explanation}
\end{example}

\subsection*{Bases and Dimension}

Recall that our motivation for defining a basis of a subspace of $\RR^n$ was to have a collection of vectors such that every vector of the subspace can be expressed as a unique linear combination of the vectors in that collection.  Definition \ref{def:basis} we gave in VSP-M-0030 generalizes to abstract vector spaces as follows.

\begin{definition}\label{def:basisabstract}
Let $V$ be a vector space.  A set $\mathcal{B}$ of vectors of $V$ is called a \dfn{basis} of $V$  provided that $\mathcal{B}$
\begin{enumerate}
\item \label{item:defbasis1abstract}
Spans $V$
\item \label{item:defbasis2abstract}
Is linearly independent
\end{enumerate}
\end{definition}

\begin{theorem}\label{th:uniquerep}
Let $V$ be a vector space, and let $\mathcal{B}=\{\vec{v}_1, \vec{v}_2,\ldots,\vec{v}_n\}$ be a basis for $V$.  Then every element $\vec{v}$ of $V$ has a unique representation as linear combination of the elements of $\mathcal{B}$.
\end{theorem}
\begin{proof}
By the definition of a basis, we know that $\vec{v}$ can be written as a linear combination of $\vec{v}_1, \vec{v}_2,\ldots,\vec{v}_n$.  Suppose there are two such representations.  Then,
$$\vec{v}=a_1\vec{v}_1+ a_2\vec{v}_2+\ldots+a_n\vec{v}_n$$
$$\vec{v}=b_1\vec{v}_1+ b_2\vec{v}_2+\ldots+b_n\vec{v}_n$$
But then we have:
\begin{align*}
a_1\vec{v}_1+ a_2\vec{v}_2+\ldots+a_n\vec{v}_n&=b_1\vec{v}_1+ b_2\vec{v}_2+\ldots+b_n\vec{v}_n\\
a_1\vec{v}_1+ a_2\vec{v}_2+\ldots+a_n\vec{v}_n-(b_1\vec{v}_1+ b_2\vec{v}_2+\ldots+b_n\vec{v}_n)&=\vec{0}\\
(a_1-b_1)\vec{v}_1+ (a_2-b_2)\vec{v}_2+\ldots+(a_n-b_n)\vec{v}_n&=\vec{0}
\end{align*}
Because $\vec{v}_1, \vec{v}_2,\ldots,\vec{v}_n$ are linearly independent, we have $a_i-b_i=0$ for $1\leq i\leq n$. Consequently $a_i=b_i$ for $1\leq i\leq n$.
\end{proof}

In VSP-M-0035 we defined the dimension of a subspace of $\RR^n$ to be the number of elements in a basis. (Definition \ref{def:dimension})  We will adopt this definition for abstract vector spaces.  As before, to ensure that \dfn{dimension} is well-defined we need to establish that this definition is independent of our choice of a basis.  The proof of the following theorem is identical to the proof of its counterpart in $\RR^n$.  (Theorem \ref{th:dimwelldefined})

\begin{theorem}\label{th:dimwelldefinedabstract}
Let $V$ be a vector space.  Suppose $\mathcal{B}=\{\vec{v}_1, \vec{v}_2,\ldots ,\vec{v}_t\}$ and $\mathcal{C}=\{\vec{w}_1, \vec{w}_2,\ldots ,\vec{w}_s\}$ be two bases of $V$.  Then $s=t$.
\end{theorem}
Now we can state the definition.
\begin{definition}\label{def:dimensionabstract}
Let $V$ be a subspace of $\RR^n$.  The \dfn{dimension} of $V$ is the number, $m$, of elements in any basis of $V$.  We write
$$\mbox{dim}(V)=m$$
\end{definition}

In our discussion to this point we have always assumed that a basis is nonempty and hence that the dimension of the space is at least $1$. However, the zero space $\{\vec{0}\}$ has {\it no} basis (see Practice Problem \ref{prob:nobasis}).  To accommodate for this, we will say that the zero vector space $\{\vec{0}\}$ is defined to have dimension $0$:
\begin{equation*}
\mbox{dim }\{\vec{0}\} = 0
\end{equation*}

Our insistence that $\mbox{dim}\{\vec{0}\} = 0$ amounts to saying that the empty set of vectors is a basis of $\{\vec{0}\}$. Thus the statement that ``the dimension of a vector space is the number of vectors in any basis'' holds even for the zero space. [Nicholson]

\begin{example}\label{ex:dimofM}
Recall that the vector space $\mathbb{M}_{m,n}$ consists of all $m\times n$ matrices. (See Example \ref{ex:MLexamplesofvectspaces} of VSP-M-0050).  Find a basis and the dimension of $\mathbb{M}_{m,n}$.
\begin{explanation}
Let $\mathcal{B}$ consist of $m\times n$ matrices 
 with exactly one entry equal to $1$ and all other entries equal to $0$. It is clear that every $m\times n$ matrix can be written as a linear combination of elements of $\mathcal{B}$.  It is also easy to see that the elements of $\mathcal{B}$ are linearly independent.  Thus $\mathcal{B}$ is a basis for $\mathbb{M}_{m,n}$.  The set $\mathcal{B}$ contains $mn$ elements, so $\mbox{dim}(\mathbb{M}_{m,n})=mn$.
\end{explanation}
\end{example}

\begin{example}\label{ex:dimofP}
Recall that $\mathbb{P}^n$ is the set of all polynomials of degree $n$ or less.  (See Example \ref{ex:pnisavectorspace} of VSP-M-0050) Show that $\mbox{dim}( \mathbb{P}^{n}) = n + 1$ and that $\{1, x, x^{2}, \dots, x^{n}\}$ is a basis of $\mathbb{P}^{n}$.

\begin{explanation}
Each polynomial $p(x) = a_{0} + a_{1}x + \ldots + a_{n}x^{n}$ in $\mathbb{P}^{n}$ is clearly a linear combination of $1, x, \dots, x^{n}$, so $\mathbb{P}^{n} = \mbox{span}\{1, x, \dots, x^{n}\}$. 

Suppose $a_{0}1 + a_{1}x + \dots + a_{n}x^{n} = 0$, then $a_{0} = a_{1} = \ldots = a_{n} = 0$. So $\{1, x, \dots, x^{n}\}$ is linearly independent and hence is a basis containing $n + 1$ vectors. Thus, $\mbox{dim}(\mathbb{P}^{n}) = n + 1$.
\end{explanation}
\end{example}

\begin{example}\label{ex:CAbasis}
Consider the subset
\begin{equation*}
C_A = \{X \mbox{ in } \mathbb{M}_{2,2} : AX = XA \}
\end{equation*}
of $\mathbb{M}_{2,2}$. 

It was shown in Example \ref{ex:centralizerofA} of VSP-M-0050 that $C_A$ is a subspace for any choice of the matrix $A$.

Let $A = 
\begin{bmatrix}
1 & 1 \\
0 & 0
\end{bmatrix}$.
Show that $\mbox{dim}(C_A) = 2$ and find a basis of $C_A$.

\begin{explanation}
 Suppose $X = 
\begin{bmatrix}
a & b \\
c & d
\end{bmatrix}$
 is in $C_A$.  
 Then
 $$\begin{bmatrix}1&1\\0&0\end{bmatrix}\begin{bmatrix}a&b\\c&d\end{bmatrix}=\begin{bmatrix}a&b\\c&d\end{bmatrix}\begin{bmatrix}1&1\\0&0\end{bmatrix}$$
 $$\begin{bmatrix}a+c&b+d\\0&0\end{bmatrix}=\begin{bmatrix}a&a\\c&c\end{bmatrix}$$
 This gives us two relationships:  
 $$b+d=a\quad\text{and}\quad c=0$$
 We can now express a generic element $X$ of $C_A$ as
 $$X=\begin{bmatrix}a&b\\c&d\end{bmatrix}=\begin{bmatrix}b+d&b\\0&d\end{bmatrix}=\begin{bmatrix}b&b\\0&0\end{bmatrix}+\begin{bmatrix}d&0\\0&d\end{bmatrix}=b\begin{bmatrix}1&1\\0&0\end{bmatrix}+d\begin{bmatrix}1&0\\0&1\end{bmatrix}$$
 
Let 
$$\mathcal{B}=\left\{\begin{bmatrix}1&1\\0&0\end{bmatrix},\begin{bmatrix}1&0\\0&1\end{bmatrix}\right\}$$

The set $\mathcal{B}$ is linearly independent. (See Practice Problem \ref{prob:CABlinind}) Every element $X$ of $C_A$ can be written as a linear combination of elements of $\mathcal{B}$.  Thus $C_A=\mbox{span}\mathcal{B}$
 So $\mathcal{B}$ is a basis of $C_A$ and $\mbox{dim}(C_A) = 2$.
\end{explanation}
\end{example}

\begin{example}\label{ex:symmetricmatsubspace} 
In Practice Problem \ref{prob:symmetricsubspace} of VSP-M-0050 you demonstrated that the set of all symmetric $n\times n$ matrices is a subspace of $\mathbb{M}_{n,n}$.

Let $V$ be a subspace of $\mathbb{M}_{2,2}$ consisting of all $2\times 2$ symmetric matrices.  Find the dimension of $V$.

\begin{explanation}
A matrix $A$ is symmetric if $A^{T} = A$. In other words, a matrix $A$ is symmetric when entries directly across the main diagonal are equal, so each $2 \times 2$ symmetric matrix has the form
$$
\begin{bmatrix}
a & c \\
c & b
\end{bmatrix}
= a\begin{bmatrix}
1 & 0 \\
0 & 0
\end{bmatrix}
+ b\begin{bmatrix}
0 & 0 \\
0 & 1
\end{bmatrix}
+ c\begin{bmatrix}
0 & 1 \\
1 & 0
\end{bmatrix}
$$
Hence the set 
$\mathcal{B} = \left\{
\begin{bmatrix}
1 & 0 \\
0 & 0
\end{bmatrix}, \begin{bmatrix}
0 & 0 \\
0 & 1
\end{bmatrix}, \begin{bmatrix}
0 & 1 \\
1 & 0
\end{bmatrix}
\right\}$
 spans $V$. The reader can verify that $\mathcal{B}$ is linearly independent. Thus $\mathcal{B}$ is a basis of $V$, so $\mbox{dim}(V) = 3$.
\end{explanation}
\end{example}

\subsection*{Finite-Dimensional Vector Spaces}
Our definition of dimension of a vector space depends on the vector space having a basis.  In this section we will establish that any vector space spanned by finitely many vectors has a basis.

\begin{definition}
A vector space is said to be \dfn{finite-dimensional} if it is spanned by finitely many vectors.
\end{definition}

Given a finite-dimensional vector space $V$ we will find a basis for $V$ by starting with a linearly independent subset of $V$ and expanding it to a basis.  The following results are more general versions of Lemmas \ref{lemma:atmostnlinindinrn} and \ref{lemma:expandinglinindset}, and Theorem \ref{th:dimwelldefined} of VSP-M-0035.  The proofs are identical and we will omit them.
\begin{lemma}\label{lemma:atmostnlinindinabstract}
Let $V$ be a vector space spanned by $n$ vectors.  If a linearly independent subset $S$ of $V$ contains $m$ vectors, then $m\leq n$.
\end{lemma}

\begin{lemma}\label{lemma:expandinglinindsetabstract}
Let $V$ be a vector space.  Let $\{\vec{v}_1,\ldots ,\vec{v}_k\}$ be a linearly independent subset of $V$.  If $\vec{u}$ is not in $\mbox{span}(\vec{v}_1,\ldots ,\vec{v}_k)$, then $\{\vec{u},\vec{v}_1,\ldots ,\vec{v}_k\}$ is linearly independent.
\end{lemma}

\begin{theorem}\label{th:expandtobasisabstract}
Let $V$ be a finite-dimensional vector space.  Any linearly independent subset of $V$ can be expanded to a basis of $V$.
\end{theorem}

\subsection*{Coordinate Vectors}

Recall that in the context of $\RR^n$ (and subspaces of $\RR^n$) the requirement that elements of a basis be linearly independent guarantees that every element of the vector space has a {\it unique} representation in terms of the elements of the basis.  (See Theorem \ref{th:linindbasis} of VSP-M-0030)  We proved the same property for abstract vector spaces in Theorem \ref{th:uniquerep}.

Uniqueness of representation in terms of the elements of a basis allows us to associate every element of a vector space $V$ with a unique \dfn{coordinate vector} with respect to a given basis.  Coordinate vectors were first introduced in VSP-M-0030.  We now give a formal definition.

\begin{definition}\label{def:coordvector}
Let $V$ be a vector space, and let $\mathcal{B}=\{\vec{v}_1, \ldots ,\vec{v}_n\}$ be a basis for $V$.  If $\vec{v}=a_1\vec{v}_1+\ldots +a_n\vec{v}_n$, then the vector in $\RR^n$ whose components are the coefficients $a_1, \ldots ,a_n$  is said to be the \dfn{coordinate vector for $\vec{v}$ with respect to $\mathcal{B}$}.  We denote the coordinate vector by $[\vec{v}]_{\mathcal{B}}$ and write:
$$[\vec{v}]_{\mathcal{B}}=\begin{bmatrix}a_1\\\vdots \\a_n\end{bmatrix}$$
\end{definition}

\begin{example}
The coordinate vector for $p(x)=4-3x^2+5x^3$ in $\mathbb{P}^4$ with respect to $\mathcal{B}=\{1, x, x^2, x^3, x^4\}$ is 
$$[p(x)]_{\mathcal{B}}=\begin{bmatrix}4\\0\\-3\\5\\0\end{bmatrix}$$
\end{example}

\begin{example}
Show that the set $\mathcal{B}=\{x, 1+x, x+x^2\}$ is a basis for $\mathbb{P}^2$, and find the coordinate vector for $p(x)=4-x+3x^2$ with respect to $\mathcal{B}$.
\begin{explanation}
We will begin by showing that the elements of $\mathcal{B}$ are linearly independent.  Suppose 
$$ax+b(1+x)+c(x+x^2)=0$$
then
$$b+(a+b+c)x+cx^2=0$$
This gives us the following system of equations:
$$\begin{array}{ccccccc}
     & &b&&&=&0\\
     a & +&b&+&c&= &0 \\
	 & &&&c&=&0
     \end{array}$$
This gives us a unique solution $a=b=c=0$.  We conclude that $\mathcal{B}$ is linearly independent.

Next, we need to show that $\mathcal{B}$ spans $\mathbb{P}^2$.  To this end, we will consider a generic element $p(x)=\alpha+\beta x+\gamma x^2$ of $\mathbb{P}^2$ and attempt to express it as a linear combination of the elements of $\mathcal{B}$.
$$ax+b(1+x)+c(x+x^2)=\alpha+\beta x+\gamma x^2$$
then
$$b+(a+b+c)x+cx^2=\alpha+\beta x+\gamma x^2$$
Setting the coefficients of like terms equal to each other gives us
$$\begin{array}{ccccccc}
     & &b&&&=&\alpha\\
     a & +&b&+&c&= &\beta \\
	 & &&&c&=&\gamma
     \end{array}$$
Solving this linear system of $a$, $b$ and $c$ gives us
$$a=\beta-\alpha-\gamma,\quad b=\alpha,\quad c=\gamma$$
(You need to verify this)  This shows that every element of $\mathbb{P}^2$ can be written as a linear combination of elements of $\mathcal{B}$.  Therefore $\mathcal{B}$ is a basis for $\mathbb{P}^2$.

To find the coordinate vector for $p(x)=4-x+3x^2$ with respect to $\mathcal{B}$ we need to express $p(x)$ as a linear combination of the elements of $\mathcal{B}$.  Fortunately, we have already done all the necessary work.  For $p(x)$, $\alpha=4$, $\beta=-1$ and $\gamma=3$.  This gives us the coefficients of the linear combination: $a=\beta-\alpha-\gamma=-8$, $b=\alpha=4$, $c=\gamma=3$.  We now write $p(x)$ as a linear combination
$$p(x)=-8(x)+4(1+x)+3(x+x^2)$$
The coordinate vector for $p(x)$ with respect to $\mathcal{B}$ is
$$[p(x)]_{\mathcal{B}}=\begin{bmatrix}-8\\4\\3\end{bmatrix}$$
\end{explanation}
\end{example}

\begin{example}
Recall that the set $V$ of all symmetric $2\times 2$ matrices is a subspace of $\mathbb{M}_{2,2}$.  In Example \ref{ex:symmetricmatsubspace} we demonstrated that $\mathcal{B} = \left\{
\begin{bmatrix}
1 & 0 \\
0 & 0
\end{bmatrix}, \begin{bmatrix}
0 & 0 \\
0 & 1
\end{bmatrix}, \begin{bmatrix}
0 & 1 \\
1 & 0
\end{bmatrix}
\right\}$ is a basis for $V$.  Find the coordinate vector with respect to $\mathcal{B}$ for 
$$A=\begin{bmatrix}2&-3\\-3&1\end{bmatrix}$$
\begin{explanation}
We write $A$ as a linear combination of the elements of $\mathcal{B}$.
$$A=\begin{bmatrix}2&-3\\-3&1\end{bmatrix}=2\begin{bmatrix}1&0\\0&0\end{bmatrix}+\begin{bmatrix}
0 & 0 \\
0 & 1
\end{bmatrix}-3\begin{bmatrix}
0 & 1 \\
1 & 0
\end{bmatrix}$$
Thus, the coordinate vector is
$$[A]_{\mathcal{B}}=\begin{bmatrix}2\\1\\-3\end{bmatrix}$$
\end{explanation}
\end{example}


Coordinate vectors will play a vital role in establishing one of the most fundamental results in linear algebra, that all $n$-dimensional vector spaces have the same structure as $\RR^n$.  In Example \ref{ex:p2isor3} of LTR-M-0060, for instance, we will show that $\mathbb{P}^2$ is essentially the same as $\RR^3$.  


\section*{Practice Problems}
\begin{problem}\label{prob:CABlinind}
Prove that set $\mathcal{B}=\left\{\begin{bmatrix}1&1\\0&0\end{bmatrix},\begin{bmatrix}1&0\\0&1\end{bmatrix}\right\}$ of Example \ref{ex:CAbasis} is linearly independent.
\end{problem}

\begin{problem}\label{prob:linindabstractvsp}
{\color{red}Nicholson} Show that each of the following sets of vectors is linearly independent.

\begin{problem}\label{prob:linindabstractvsp1}
$$\{1 + x, 1 - x, x + x^{2}\}$$ in $\mathbb{P}^{2}$
\end{problem}

\begin{problem}\label{prob:linindabstractvsp2}
$$\{x^{2}, x + 1, 1 - x - x^{2}\}$$ in $\mathbb{P}^{2}$
\end{problem}

\begin{problem}\label{prob:linindabstractvsp3}
$$
\left\{
\begin{bmatrix}
1 & 1 \\
0 & 0
\end{bmatrix}
, 
\begin{bmatrix}
1 & 0 \\
1 & 0
\end{bmatrix}
, 
\begin{bmatrix}
0 & 0 \\
1 & -1
\end{bmatrix}
,\
\begin{bmatrix}
0 & 1 \\
0 & 1
\end{bmatrix}
\right\}$$ 

in $\mathbb{M}_{2,2}$
\end{problem}

\end{problem}

\begin{problem}
Show that each set in Practice Problem \ref{prob:linindabstractvsp} is a basis for its respective vector space.
\end{problem}

\begin{problem}
Find the coordinate vector for $p(x)=6-2x+4x^2$ with respect to the given basis $\mathcal{B}$.

\begin{problem}
$$\mathcal{B}=\{1 + x, 1 - x, x + x^{2}\}$$
Answer:
$$[p(x)]_{\mathcal{B}}=\begin{bmatrix}\answer{0}\\\answer{6}\\\answer{4}\end{bmatrix}$$
\end{problem}

\begin{problem}
$$\mathcal{B}=\{x^{2}, x + 1, 1 - x - x^{2}\}$$
Answer:
$$[p(x)]_{\mathcal{B}}=\begin{bmatrix}\answer{8}\\\answer{2}\\\answer{4}\end{bmatrix}$$
\end{problem}
\end{problem}
\begin{problem}
Find the coordinate vector for $A=\begin{bmatrix}4&-3\\1&2\end{bmatrix}$ with respect to the basis
$$\mathcal{B}=
\left\{
\begin{bmatrix}
1 & 1 \\
0 & 0
\end{bmatrix}
, 
\begin{bmatrix}
1 & 0 \\
1 & 0
\end{bmatrix}
, 
\begin{bmatrix}
0 & 0 \\
1 & -1
\end{bmatrix}
,\
\begin{bmatrix}
0 & 1 \\
0 & 1
\end{bmatrix}
\right\}$$ 
Answer:
$$[A]_{\mathcal{B}}=\begin{bmatrix}\answer{-1}\\\answer{5}\\\answer{-4}\\\answer{-2}\end{bmatrix}$$
\end{problem}

\begin{problem}
Let $V$ be a vector space of dimension $3$.  Suppose $S=\{\vec{v}_1, \vec{v}_2, \vec{v}_3\}$ is linearly independent in $V$.  Show that $S$ is a basis for $V$.
\end{problem}

\section*{Text Source} The discussion of the zero space was adapted from Section 6.3 of Keith Nicholson's \href{https://open.umn.edu/opentextbooks/textbooks/linear-algebra-with-applications}{\it Linear Algebra with Applications}.

W. Keith Nicholson, {\it Linear Algebra with Applications}, Lyryx 2018, Open Edition, p. 349 

\section*{Example Source}
Examples \ref{ex:polyindset} and \ref{ex:CAbasis} were adapted from Examples 6.3.1 and 6.3.10 of Keith Nicholson's \href{https://open.umn.edu/opentextbooks/textbooks/linear-algebra-with-applications}{\it Linear Algebra with Applications}.

W. Keith Nicholson, {\it Linear Algebra with Applications}, Lyryx 2018, Open Edition, p. 346, 350

\section*{Exercise Source}
Practice Problems \ref{prob:linindabstractvsp1}, \ref{prob:linindabstractvsp2} and \ref{prob:linindabstractvsp3} are Exercises 6.3(a)(b)(c) from Keith Nicholson's \href{https://open.umn.edu/opentextbooks/textbooks/linear-algebra-with-applications}{\it Linear Algebra with Applications}.

W. Keith Nicholson, {\it Linear Algebra with Applications}, Lyryx 2018, Open Edition, p. 351


\end{document} 