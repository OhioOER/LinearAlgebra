\documentclass{ximera}

\author{Anna Davis \and Paul Zachlin} \title{Linear Systems as Matrix and Linear Combination Equations} \license{CC-BY 4.0}

%% You can put user macros here
%% However, you cannot make new environments

\graphicspath{{./}{firstExample/}{secondExample/}}

\usepackage{tikz}
\usepackage{tkz-euclide}
\usetkzobj{all}
\pgfplotsset{compat=1.7} % prevents compile error.

\tikzstyle geometryDiagrams=[ultra thick,color=blue!50!black]


\begin{document}

\begin{abstract}
  We interpret linear systems as matrix equations and as equations involving linear combinations of vectors.
\end{abstract}
\maketitle


\section*{A Linear System as a Matrix Equation}

\begin{initprob}
Consider the linear system
$$\begin{array}{ccccccccc}
      3x_1 &- &2x_2&+&4x_3&+&x_4&= &5 \\
	 -x_1& &&+&5x_3&-&2x_4&=&1\\
     2x_1& +&x_2&-&x_3&+&3x_4&=&-4   
    \end{array}$$
Let's construct the coefficient matrix $A$ and multiply it by $\vec{x}=\begin{bmatrix}x_1\\x_2\\x_3\\x_4\end{bmatrix}$ on the right.  
$$A\vec{x}=\begin{bmatrix}3&-2&4&1\\-1&0&5&-2\\2&1&-1&3\end{bmatrix}\begin{bmatrix}x_1\\x_2\\x_3\\x_4\end{bmatrix}=\begin{bmatrix}3x_1-2x_2+4x_3+x_4\\-x_1+5x_3-2x_4\\2x_1+x_2-x_3+3x_4\end{bmatrix}$$
Observe that each component of the product vector corresponds to one of the equations in the system.  Let $\vec{b}=\begin{bmatrix}5\\1\\-4\end{bmatrix}$.  Then 
$$A\vec{x}=\vec{b}$$
$$\begin{bmatrix}3&-2&4&1\\-1&0&5&-2\\2&1&-1&3\end{bmatrix}\begin{bmatrix}x_1\\x_2\\x_3\\x_4\end{bmatrix}=\begin{bmatrix}5\\1\\-4\end{bmatrix}$$
is a matrix equation that corresponds to our system of equations.

\end{initprob}


\begin{general} A system of linear equations

$$\begin{array}{ccccccccc}
      a_{11}x_1 &+ &a_{12}x_2&+&\ldots&+&a_{1n}x_n&= &b_1 \\
	 a_{21}x_1 &+ &a_{22}x_2&+&\ldots&+&a_{2n}x_n&= &b_2 \\
     &&&&\vdots&&&& \\
     a_{m1}x_1 &+ &a_{m2}x_2&+&\ldots&+&a_{mn}x_n&= &b_m
    \end{array}$$

Can be written as a matrix equation as follows:

$$
 \begin{bmatrix}
           a_{11} & a_{12}&\dots&a_{1n}\\
           a_{21}&a_{22} &\dots &a_{2n}\\
		\vdots & \vdots&\ddots &\vdots\\
		a_{m1}&\dots &\dots &a_{mn}
         \end{bmatrix}
		\begin{bmatrix}
           x_1\\
           x_2\\
		\vdots \\
		x_n
         \end{bmatrix} = \begin{bmatrix}
           b_1\\
           b_2\\
		\vdots \\
		b_m
        \end{bmatrix}
$$
Solving this matrix equation (or showing that a solution does not exist) amounts to finding the reduced row-echelon form  of the augmented matrix

$$\left[\begin{array}{cccc|c}  
 a_{11} & a_{12}&\dots&a_{1n}&b_1\\
           a_{21}&a_{22} &\dots &a_{2n}&b_2\\
		\vdots & \vdots&\ddots &\vdots&\vdots\\
		a_{m1}&\dots &\dots &a_{mn}&b_m
 \end{array}\right]$$

\end{general}

\begin{example} Given a linear system
$$\begin{array}{ccccc}
      x& +&2y&=&0\\
      -x & +&y&= &-3\\
      & &y&=&-1\\
      x& &&=&2
    \end{array}$$
    \begin{enumerate}
    \item\label{ex:linsysmatrix1a}
    Write the system as a matrix equation
    \item\label{ex:linsysmatrix1b}
    Solve the system and the matrix equation
    \end{enumerate}
    \begin{explanation}
    \ref{ex:linsysmatrix1a} 
    The matrix equation that corresponds to the system is
    $$\begin{bmatrix}\answer{1}&\answer{2}\\\answer{-1}&\answer{1}\\\answer{0}&\answer{1}\\\answer{1}&\answer{0}\end{bmatrix}\begin{bmatrix}x\\y\end{bmatrix}=\begin{bmatrix}\answer{0}\\\answer{-3}\\\answer{-1}\\\answer{2}\end{bmatrix}$$
    
    \ref{ex:linsysmatrix1b}
    The augmented matrix that corresponds to the original system and its reduced row-echelon form are
    
    $$\left[\begin{array}{cc|c}  
 1&2&0\\-1&1&-3\\0&1&-1\\1&0&2
 \end{array}\right]\begin{array}{c}
 \\
 \rightsquigarrow\\
 \\
 \end{array}\left[\begin{array}{cc|c}  
 1&0&2\\0&1&-1\\0&0&0\\0&0&0
 \end{array}\right]$$
 
 This shows that the ordered pair $(2, -1)$ is a solution to the system.  We conclude that $\vec{x}=\begin{bmatrix}x\\y\end{bmatrix}=\begin{bmatrix}2\\-1\end{bmatrix}$ is a solution to the matrix equation in \ref{ex:linsysmatrix1a}.  A quick verification confirms this
 $$\begin{bmatrix}1&2\\-1&1\\0&1\\1&0\end{bmatrix}\begin{bmatrix}2\\-1\end{bmatrix}=\begin{bmatrix}0\\-3\\-1\\2\end{bmatrix}$$
    \end{explanation}
\end{example}

\begin{example}
Let $$A=\begin{bmatrix}2&1&-1&2\\1&1&0&3\end{bmatrix}\quad\text{and}\quad\vec{b}=\begin{bmatrix}0\\-2\end{bmatrix} $$
Solve $A\vec{x}=\vec{b}$.
\begin{explanation}
We write the equation $A\vec{x}=\vec{b}$ in augmented matrix form and apply elementary row operations to find its reduced row-echelon form.
$$\left[\begin{array}{cccc|c}  
 2&1&-1&2&0\\1&1&0&3&-2
 \end{array}\right]\begin{array}{c}
 \\
 \rightsquigarrow\\
 \\
 \end{array}\left[\begin{array}{cccc|c}  
 1&0&-1&-1&2\\0&1&1&4&-4
 \end{array}\right]$$
 
One way to obtain a solution is to convert this to a system of equations.  It is not necessary to write the system down, but it helps to think about it as you write out your solution vector. 

$$\begin{array}{ccccccccc}
      x_1 & &&-&x_3&-&x_4&= &2 \\
	 & &x_2&+&x_3&+&4x_4&=&-4
        \end{array}$$
        
        We see that $x_1$ and $x_2$ are leading variables because they correspond to leading 1s in the reduced row-echelon form , while $x_3$ and $x_4$ are free variables.  We start by assigning parameters $s$ and $t$ to $x_3$ and $x_4$, respectively, then solve for $x_1$ and $x_2$.
\begin{align*}x_1&=2+s+t\\
        x_2&=-4-s-4t\\
        x_3&=s\\
        x_4&=t
        \end{align*}
       We can now write the solution vector as follows
       \begin{equation}\label{eq:generalvsparticular}\vec{x}=\begin{bmatrix}2+s+t\\-4-s-4t\\s\\t\end{bmatrix}=\begin{bmatrix}2\\-4\\0\\0\end{bmatrix}+\begin{bmatrix}1\\-1\\1\\0\end{bmatrix}s+\begin{bmatrix}1\\-4\\0\\1\end{bmatrix}t\end{equation}
\end{explanation}
\end{example}

The solution given in (\ref{eq:generalvsparticular}) is an example of a \dfn{general solution} because it accounts for all of the solutions to the system.  Letting $s$ and $t$ take on specific values produces \dfn{particular solutions}.  For example, $\begin{bmatrix}2\\-1\\1\\-1\end{bmatrix}$ is a particular solution that corresponds to $s=1$, $t=-1$.
\section*{A Linear System as a Linear Combination Equation}

Recall that the product of a matrix and a vector can be interpreted as a linear combination of the columns of the matrix.  For example,
$$\begin{bmatrix}1&2&3&4\\5&6&7&8\end{bmatrix}\begin{bmatrix}x_1\\x_2\\x_3\\x_4\end{bmatrix}=x_1\begin{bmatrix}1\\5\end{bmatrix}+x_2\begin{bmatrix}2\\6\end{bmatrix}+x_3\begin{bmatrix}3\\7\end{bmatrix}+x_4\begin{bmatrix}4\\8\end{bmatrix}$$

\begin{example}\label{ex:linearcombofcols2}For each given matrix $A$ and vector $\vec{b}$, determine whether $\vec{b}$ is a linear combination of the columns of $A$.  If possible, express $\vec{b}$ as a linear combination of the columns of $A$.

  \begin{enumerate}
\item \label{item:linearcombofcols2a}
$$A=\begin{bmatrix}
3&1&-2\\
1&0&3\\
-2&1&1
\end{bmatrix},\,\,\,\vec{b}=\begin{bmatrix} -2\\7\\-1\end{bmatrix}$$

\item \label{item:linearcombofcols2b}
$$A=\begin{bmatrix}
1&-1&3\\
2&-1&1\\
0&1&-5
\end{bmatrix},\,\,\,\vec{b}=\begin{bmatrix} 4\\-1\\2\end{bmatrix}$$
\end{enumerate}

\begin{explanation} \ref{item:linearcombofcols2a}
 We are looking for $x_1, x_2, x_3$ such that
$$x_1\begin{bmatrix}3\\1\\-2\end{bmatrix}+x_2\begin{bmatrix}1\\0\\1\end{bmatrix}+x_3\begin{bmatrix}-2\\3\\1\end{bmatrix}=\begin{bmatrix}-2\\7\\-1\end{bmatrix}$$
Solving this equation amounts to finding $\vec{x}=\begin{bmatrix}x_1\\x_2\\x_3\end{bmatrix}$ such that $A\vec{x}=\vec{b}$.  The augmented matrix corresponding to this equation, together with its reduced row-echelon form are
$$\left[\begin{array}{ccc|c}  
 3&1&-2&-2\\1&0&3&7\\-2&1&1&-1
 \end{array}\right]\begin{array}{c}
 \\
 \rightsquigarrow\\
 \\
 \end{array}\left[\begin{array}{ccc|c}  
 1&0&0&1\\0&1&0&-1\\0&0&1&2
 \end{array}\right]$$

So, $\vec{x}=\begin{bmatrix} 1\\-1\\2\end{bmatrix}$ is a solution to the matrix equation.  We conclude that $\vec{b}$ is a linear combination of the columns of $A$, and write
$$\vec{b}=\begin{bmatrix} -2\\7\\-1\end{bmatrix}=\begin{bmatrix} 3\\1\\-2\end{bmatrix}-\begin{bmatrix} 1\\0\\1\end{bmatrix}+2\begin{bmatrix} -2\\3\\1\end{bmatrix}$$

\ref{item:linearcombofcols2b}   We begin by attempting to solve the matrix equation  $A\vec{x}=\vec{b}$.  The augmented matrix corresponding to this equation, together with its reduced row-echelon form are
$$\left[\begin{array}{ccc|c}  
 1&-1&3&4\\2&-1&1&-1\\0&1&-5&2
 \end{array}\right]\begin{array}{c}
 \\
 \rightsquigarrow\\
 \\
 \end{array}\left[\begin{array}{ccc|c}  
 1&0&-2&0\\0&1&-5&0\\0&0&0&1
 \end{array}\right]$$

This shows that this matrix equation has no solutions.  We conclude that $\vec{b}$ is not a linear combination of the columns of $A$. 

\end{explanation}
\end{example}

\section*{Practice Problems}
 \begin{problem}
Given a system of linear equations, write (a) the corresponding matrix equation, and (b) the corresponding linear combination equation.  DO NOT SOLVE.
\begin{center}
$\begin{array}{ccccccc}
       3x&- &y &- &2z&= &4 \\
     -x& & &+ &z&= &-1 \\
     & &-y &+ &5z&=&0

    \end{array}$
\end{center}
\end{problem}
\begin{problem}
Use an augmented matrix and elementary row operations to find coefficients $x_1$ and $x_2$ that make the expression true, or demonstrate that such coefficients do not exist.
 \begin{problem}

    $$ x_1\begin{bmatrix}
           1\\
           -2
         \end{bmatrix}+ x_2\begin{bmatrix}
           1\\
           3
         \end{bmatrix}=\begin{bmatrix}
           1\\
           8
         \end{bmatrix}$$
         \end{problem}



\begin{problem}

 $$ x_1\begin{bmatrix}
           4\\
           -1
         \end{bmatrix}+ x_2\begin{bmatrix}
           -8\\
           2
         \end{bmatrix}=\begin{bmatrix}
           0\\
           3
         \end{bmatrix}$$

  \end{problem}

\end{problem}
 
 \begin{problem}
 In each problem below determine whether vector $\vec{b}$ is in the span of the given set of vectors.
 \begin{problem}
 $\vec{b}=\begin{bmatrix}2\\14\\7\end{bmatrix}$ and $\left\{\begin{bmatrix}2\\-1\\1\end{bmatrix}, \begin{bmatrix}-3\\4\\1\end{bmatrix}, \begin{bmatrix}1\\-3\\-2\end{bmatrix}\right\}$
 \begin{multipleChoice}
 \choice{$\vec{b}$ is in the span of the vectors.}
 \choice[correct]{$\vec{b}$ is NOT in the span of the vectors.}
 \end{multipleChoice}
 \end{problem}
 
 \begin{problem}
 $\vec{b}=\begin{bmatrix}5\\2\\4\end{bmatrix}$ and $\left\{\begin{bmatrix}4\\2\\4\end{bmatrix}, \begin{bmatrix}4\\5\\1\end{bmatrix}, \begin{bmatrix}3\\2\\2\end{bmatrix}, \begin{bmatrix}1\\0\\2\end{bmatrix}\right\}$
 \begin{multipleChoice}
 \choice[correct]{$\vec{b}$ is in the span of the vectors.}
 \choice {$\vec{b}$ is NOT in the span of the vectors.}
 \end{multipleChoice}
 \end{problem}
 
 \begin{problem}
 $\vec{b}=\begin{bmatrix}2\\4\\-7\\-5\end{bmatrix}$ and $\left\{\begin{bmatrix}1\\-1\\2\\3\end{bmatrix}, \begin{bmatrix}4\\1\\-3\\1\end{bmatrix}\right\}$
 \begin{multipleChoice}
 \choice[correct]{$\vec{b}$ is in the span of the vectors.}
 \choice{$\vec{b}$ is NOT in the span of the vectors.}
 \end{multipleChoice}
 \end{problem}
 
 \end{problem}


\end{document} 
