\documentclass{ximera}
%% You can put user macros here
%% However, you cannot make new environments

\graphicspath{{./}{firstExample/}{secondExample/}}

\usepackage{tikz}
\usepackage{tkz-euclide}
\usetkzobj{all}
\pgfplotsset{compat=1.7} % prevents compile error.

\tikzstyle geometryDiagrams=[ultra thick,color=blue!50!black]


\author{Anna Davis \and Rosemarie Emanuele \and Paul Zachlin \and Paul Bender} \title{MAT-0050:  The Inverse of a Matrix} \license{CC-BY 4.0}

\begin{document}
\begin{abstract}
 We develop a method for finding the inverse of a square matrix, discuss when the inverse does not exist, and use matrix inverses to solve matrix equations.
\end{abstract}
\maketitle
\section*{MAT-0050:  The Inverse of a Matrix}
\subsection*{Definition and Properties of Matrix Inverses}
Consider the equation $2x=6$.  It takes little time to recognize that the solution to this equation is $x=3$.  In fact, the solution is so obvious that we do not think about the algebraic steps necessary to find it.  Let's take a look at these steps in detail.
$$2x=6$$
$$\frac{1}{2}\times (2x)=\frac{1}{2}\times 6$$
$$(\frac{1}{2}\times 2)x=3$$
$$1x=3$$
$$x=3$$
This process utilizes many properties of real-number multiplication.  In particular, we make use of the existence of \dfn{multiplicative inverses}.  Every non-zero real number $a$  has a multiplicative inverse $a^{-1}=\frac{1}{a}$ with the property that $\frac{1}{a}\times a=a\times \frac{1}{a}=1$.  We say that $1$ is the \dfn{multiplicative identity} because $a\times 1=1\times a=a$.

Given a matrix equation $A\vec{x}=\vec{b}$, we would like follow a process similar to the one above to solve this matrix equation for $\vec{x}$.  

Given the set of all $n\times n$ matrices, with standard matrix multiplication, the role of the multiplicative identity is filled by $I_n$ because $AI_n=I_nA=A$. 

Given an $n\times n$ matrix $A$, a multiplicative inverse of $A$ would have to be some $n\times n$ matrix $B$ that satisfies the following property:
$$BA=AB=I_n$$

Assuming that such an inverse $B$ exists, this is what the process of solving the equation $A\vec{x}=\vec{b}$ would look like:
$$A\vec{x}=\vec{b}$$
$$B(A\vec{x})=B\vec{b}$$
$$(BA)\vec{x}=B\vec{b}$$
$$I\vec{x}=B\vec{b}$$
$$\vec{x}=B\vec{b}$$

\begin{definition}\label{def:matinverse}
Let $A$ be an $n\times n$ matrix.  An $n\times n$ matrix $B$ is called an \dfn{inverse} of $A$ if 
$$AB=BA=I$$
where $I$ is an $n\times n$ identity matrix.  If such an inverse matrix exists, we say that $A$ is \dfn{invertible}.  If an inverse does not exist, we say that $A$ is not invertible.
\end{definition}
It follows directly from the way the definition is stated that if $B$ is an inverse of $A$, then $A$ is an inverse of $B$. We say that $A$ and $B$ are inverses of each other.

The following theorem shows that matrix inverses are unique.

\begin{theorem}\label{th:matinverseunique} Suppose $A$ is an invertible matrix, and $B$ is an inverse of $A$.  Then $B$ is unique.
\end{theorem}
\begin{proof}
Because $B$ is an inverse of $A$, we have:
$$AB=BA=I$$
Suppose there exists another $n\times n$ matrix $C$ such that 
$$AC=CA=I$$
Then
$$C(AB)=CI$$
$$(CA)B=C$$
$$IB=C$$
$$B=C$$
\end{proof}
Now that we know that a matrix $A$ cannot have more than one inverse, we can safely refer to the inverse of $A$ as $A^{-1}$.

\begin{example}\label{ex:inverse1} Let $$A=\begin{bmatrix}1 & -1\\-5 & 2\end{bmatrix}\quad\text{and}\quad B=\begin{bmatrix}-2/3 & -1/3\\-5/3 & -1/3\end{bmatrix}$$ Verify that $A$ and $B$ are inverses of each other.
\begin{explanation} We will show that $AB=BA=I$.

$$
AB=\begin{bmatrix}1 & -1\\-5 & 2\end{bmatrix}\begin{bmatrix}-2/3 & -1/3\\-5/3 & -1/3\end{bmatrix}
=\begin{bmatrix}-2/3+5/3 & -1/3+1/3\\10/3-10/3 & 5/3-2/3\end{bmatrix}=\begin{bmatrix}1 & 0\\0 & 1\end{bmatrix}
$$

$$
BA=\begin{bmatrix}-2/3 & -1/3\\-5/3 & -1/3\end{bmatrix}\begin{bmatrix}1 & -1\\-5 & 2\end{bmatrix}
=\begin{bmatrix}-2/3+5/3 & 2/3-2/3\\-5/3+5/3 & 5/3-2/3\end{bmatrix}=\begin{bmatrix}1 & 0\\0 & 1\end{bmatrix}
$$
\end{explanation}
\end{example}

\begin{example}\label{ex:inverse2} Use what we found in Example \ref{ex:inverse1} to solve the matrix equation:
$$\begin{bmatrix}1 & -1\\-5 & 2\end{bmatrix}\vec{x}=\begin{bmatrix}-3\\1\end{bmatrix}$$
\begin{explanation}
We multiply both sides of the equation by the inverse of $\begin{bmatrix}1 & -1\\-5 & 2\end{bmatrix}$.
$$\begin{bmatrix}-2/3 & -1/3\\-5/3 & -1/3\end{bmatrix}\begin{bmatrix}1 & -1\\-5 & 2\end{bmatrix}\vec{x}=\begin{bmatrix}-2/3 & -1/3\\-5/3 & -1/3\end{bmatrix}\begin{bmatrix}-3\\1\end{bmatrix}$$
$$\vec{x}=\begin{bmatrix}5/3\\14/3\end{bmatrix}$$
\end{explanation}
\end{example}

We now prove several useful properties of matrix inverses.

\begin{theorem}\label{th:invprop} The following properties are stated for square matrices of appropriate sizes.
\begin{enumerate}
\item \label{item:inverseofid} $I^{-1}=I$
\item \label{item:inverseofinverse} For an invertible matrix $A$, $(A^{-1})^{-1}=A$.
\item \label{item:inverseofproduct} If $A$ and $B$ are invertible matrices, then $(AB)$ is invertible and $(AB)^{-1}=B^{-1}A^{-1}$.
\item \label{item:inversekA} If $A$ is an invertible matrix and $k$ is a non-zero number, then $(kA)^{-1}=\frac{1}{k}A^{-1}$.
\item \label{item:inversetranspose} If $A$ is an invertible matrix, then $A^T$ is also invertible, and $(A^T)^{-1}=(A^{-1})^T$.
\end{enumerate}
\end{theorem}
We will prove \ref{item:inverseofproduct}.  The remaining properties are left as exercises.
\begin{proof}[Proof of Property~\ref{item:inverseofproduct}:]
We will check to see if $B^{-1}A^{-1}$ is the inverse of $AB$.
$$(B^{-1}A^{-1})(AB)=B^{-1}(A^{-1}A)B=B^{-1}IB=B^{-1}B=I$$
$$(AB)(B^{-1}A^{-1})=A(BB^{-1})A^{-1}=AIA^{-1}=AA^{-1}=I$$
Thus $(AB)$ is invertible and $(AB)^{-1}=B^{-1}A^{-1}$.
\end{proof}

\subsection*{Computing the Inverse}
We now turn to the question of how to find the inverse of a square matrix, or determine that the inverse does not exist.  

Given a square matrix $A$, we are looking for a square matrix $B$ such that $$AB=I\quad\text{and}\quad BA=I$$
We will start by attempting to satisfy $AB=I$.
Let $\vec{v}_1,\ldots,\vec{v}_n$ be the columns of $B$, then
$$A\begin{bmatrix}
           | & |& &|\\
		\vec{v}_1 & \vec{v}_2 &\dots &\vec{v}_n\\
		|&| & &|
         \end{bmatrix}=\begin{bmatrix}
           | & |& &|\\
		\vec{e}_1 & \vec{e}_2 &\dots &\vec{e}_n\\
		|&| & &|
         \end{bmatrix}$$
where each $\vec{e}_i$ is a standard unit vector of $\RR^n$.   This gives us a system of equations $A\vec{v}_i=\vec{e}_i$ for each $1\leq i\leq n$.  If each $A\vec{v}_i=\vec{e}_i$ has a unique solution, then finding these solutions will give us the columns of the desired matrix $B$.  

First, suppose that $\mbox{rref}(A)=I$, then we can use elementary row operations to carry each $[A|\vec{e}_i]$ to its reduced row-echelon form.
$$[A|\vec{e}_i]\rightsquigarrow [I|\vec{v}_i]$$
Observe that the row operations that carry $A$ to $I$ will be the same for each $A\vec{v}_i=\vec{e}_i$.  We can, therefore, combine the process of solving $n$ systems of equations into a single process
$$[A|I]\rightsquigarrow [I|B]$$
Each $\vec{v}_i$ is a unique solution of $A\vec{v}_i=\vec{e}_i$, and we conclude that $$B=\begin{bmatrix}
           | & |& &|\\
		\vec{v}_1 & \vec{v}_2 &\dots &\vec{v}_n\\
		|&| & &|
         \end{bmatrix}$$ is a solution to $AB=I$.  
 
Now we make a key observation: the fact that any elementary row operation can be reversed, as shown in \ref{}.  We can reverse the elementary row operations to obtain
$$[I|B]\rightsquigarrow [A|I]$$
But the same row operations would also give us
$$[B|I]\rightsquigarrow [I|A]$$
We conclude that $BA=I$, and $B=A^{-1}$.
         
Next, suppose that $\mbox{rref}(A)\neq I$.  Then $\mbox{rref}(A)$ must contain a row of zeros.  Because one of the rows of $A$ was completely wiped out by elementary row operations, one of the rows of $A$ must be a linear combination of the other rows.  Suppose row $p$ is a linear combination of the other rows.  Then row $p$ can be carried to a row of zeros. But then the system $A\vec{v}_p=\vec{e}_p$ is inconsistent.  This is because $\vec{e}_p$ has a $1$ as the $p^{th}$ entry and zeros everywhere else.  The $1$ in the $p^{th}$ spot will not be affected by elementary row operations, and the $p^{th}$ row will eventually look like this
$$[0\ldots 0|1]$$
This shows that a matrix $B$ such that $AB=I$ does not exist, and $A$ does not have an inverse.

We have just proved the following theorem.

\begin{theorem}[Row-reduction Method for Computing the Inverse of a Matrix]\label{th:matrixinverse}
Let $A$ be a square matrix.  If it is possible to use elementary row operations to carry the augmented matrix $[A|I]$ to $[I|B]$, then $B=A^{-1}$.  If such a reduction is not possible, then $A$ does not have an inverse.
\end{theorem}

\begin{corollary}\label{cor:rrefI}
A square matrix $A$ has an inverse if and only if $\mbox{rref}(A)=I$.
\end{corollary}

\begin{example}\label{ex:inverse3} Find $A^{-1}$ or demonstrate that $A^{-1}$ does not exist.
$$A=\begin{bmatrix}1&-1&2\\1&1&1\\1&3&-1\end{bmatrix}$$
\begin{explanation}
We start with the augmented matrix
$$\left[\begin{array}{ccc|ccc}  
 1&-1&2&1&0&0\\1&1&1&0&1&0\\1&3&-1&0&0&1
 \end{array}\right]
  \begin{array}{c}
 \\
 \xrightarrow{R_2-R_1}\\
 \xrightarrow{R_3-R_1}
 \end{array}$$
 
  $$\left[\begin{array}{ccc|ccc}  
 1&-1&2&1&0&0\\0&2&-1&-1&1&0\\0&4&-3&-1&0&1
 \end{array}\right]
  \begin{array}{c}
 \\
 \\
 \xrightarrow{R_3-2R_2}
 \end{array}$$
 
 $$\left[\begin{array}{ccc|ccc}  
 1&-1&2&1&0&0\\0&2&-1&-1&1&0\\0&0&-1&1&-2&1
 \end{array}\right]
 \begin{array}{c}
 \\
 \\
 \xrightarrow{(-1)R_3}
 \end{array}$$
 
 $$\left[\begin{array}{ccc|ccc}  
 1&-1&2&1&0&0\\0&2&-1&-1&1&0\\0&0&1&-1&2&-1
 \end{array}\right]
 \begin{array}{c}
 \xrightarrow{R_1-2R_3}\\
 \xrightarrow{R_2+R_3}\\
\\
 \end{array}$$
 
 $$\left[\begin{array}{ccc|ccc}  
 1&-1&0&3&-4&2\\0&2&0&-2&3&-1\\0&0&1&-1&2&-1
 \end{array}\right]
  \begin{array}{c}
 \\
 \xrightarrow{\frac{1}{2}R_2}\\
\\
 \end{array}$$
 
 $$\left[\begin{array}{ccc|ccc}  
 1&-1&0&3&-4&2\\0&1&0&-1&3/2&-1/2\\0&0&1&-1&2&-1
 \end{array}\right]
 \begin{array}{c}
 \xrightarrow{R_1+R_2}\\
 \\
\\
 \end{array}$$
 
 $$\left[\begin{array}{ccc|ccc}  
 1&0&0&2&-5/2&3/2\\0&1&0&-1&3/2&-1/2\\0&0&1&-1&2&-1
 \end{array}\right]$$
 
 We conclude that
 
$$A^{-1}=\begin{bmatrix}2&-5/2&3/2\\-1&3/2&-1/2\\-1&2&-1\end{bmatrix}$$
\end{explanation}

\end{example}

\begin{example}\label{ex:findinverse2} Find $A^{-1}$ or demonstrate that $A$ is not invertible.
$$A=\begin{bmatrix}2&3&-1\\0&2&1\\4&4&-3\end{bmatrix}$$
\begin{explanation}
We start with the augmented matrix
$$\left[\begin{array}{ccc|ccc}  
 2&3&-1&1&0&0\\0&2&1&0&1&0\\4&4&-3&0&0&1
 \end{array}\right]
 \begin{array}{c}
 \xrightarrow{\frac{1}{2}R_1}\\
 \xrightarrow{\frac{1}{2}R_2}\\
\\
 \end{array}$$

 $$\left[\begin{array}{ccc|ccc}  
 1&3/2&-1/2&1/2&0&0\\0&1&1/2&0&1/2&0\\4&4&-3&0&0&1
 \end{array}\right]
 \begin{array}{c}
 \\
 \\
\xrightarrow{R_3-4R_1}
 \end{array}$$

 $$\left[\begin{array}{ccc|ccc}  
 1&3/2&-1/2&1/2&0&0\\0&1&1/2&0&1/2&0\\0&-2&-1&-2&0&1
 \end{array}\right]
 \begin{array}{c}
 \\
 \\
\xrightarrow{R_3+2R_2}
 \end{array}$$
 
 $$\left[\begin{array}{ccc|ccc}  
 1&3/2&-1/2&1/2&0&0\\0&1&1/2&0&1/2&0\\0&0&0&-2&1&1
 \end{array}\right]$$
 At this point we see that the left-hand side cannot be turned into $I$ through elementary row operations. We conclude that $A^{-1}$ does not exist.
\end{explanation}
\end{example}
\begin{remark}
Recall that a square matrix whose reduced row-echelon form is the identity matrix is called \dfn{nonsingular}. (Definition \ref{def:nonsingularmatrix} of SYS-0030)
According to Corollary \ref{cor:rrefI}, a matrix is invertible if and only if it is nonsingular.  For this reason many linear algebra texts use the terms \dfn{invertible} and \dfn{nonsingular} as synonyms.
\end{remark}

\subsection*{Inverse of a $2\times 2$ Matrix}
We will conclude this section by discussing the inverse of a nonsingular $2\times 2$ matrix.   
Let $A=\begin{bmatrix}a&b\\c&d\end{bmatrix}$ be a non-singular matrix.  We can find $A^{-1}$ by using the row reduction method described above, that is, by computing the reduced row-echelon form of $[A|I]$.  Row reduction yields the following:

$$\left[\begin{array}{cc|cc}  
 a&b&1&0\\c&d&0&1
 \end{array}\right]\rightsquigarrow\left[\begin{array}{cc|cc}  
 1&0&d/(ad-bc)&-b/(ad-bc)\\0&1&-c/(ad-bc)&a/(ad-bc)
 \end{array}\right]$$
 Note that the denominator of each term in the inverse matrix is the same.  Factoring it out, gives us the following formula for $A^{-1}$.

\begin{formula}\label{form:detinverse}
$$A^{-1}=\frac{1}{ad-bc}\begin{bmatrix}d&-b\\-c&a\end{bmatrix}$$
\end{formula}

Clearly, the expression for $A^{-1}$ is defined, if and only if $ad-bc\neq 0$.  So, what happens when $ad-bc=0$?  In Practice Problem \ref{prob:inverseformula} you will be asked to fill in the steps of the row reduction procedure that produces this formula, and show that if $ad-bc=0$ then $A$ does not have an inverse. 


\section*{Practice Problems}

\begin{problem}\label{prob:owninverse}
Verify that the matrix $\begin{bmatrix} 2 & -1\\3 & -2\end{bmatrix}$ is its own inverse.
\end{problem}

\begin{problem}\label{prob:inverseofsingularmatrix}
Use the row-reduction method for computing matrix inverses to explain why the given matrix does not have an inverse.
$$\begin{bmatrix}1&1&1\\2&2&2\\3&3&3\end{bmatrix}$$
\end{problem}

\begin{problem}\label{prob:inversesofeachother} Let $$A=\begin{bmatrix}1&-1&2\\2&1&1\\-3&1&0\end{bmatrix}\quad\text{and}\quad B=\begin{bmatrix}-1/12&1/6&-1/4\\-1/4&1/2&1/4\\5/12&1/6&1/4\end{bmatrix}$$ Are $A$ and $B$ inverses of each other?

\begin{multipleChoice}
  \choice[correct]{Yes, $A$ and $B$ are inverses.}
  \choice{No, $A$ and $B$ are not inverses.}
\end{multipleChoice}
\end{problem}



\begin{problem}\label{prob:oroveinverseofid}
Prove Properties \ref{item:inverseofid} and \ref{item:inverseofinverse} of Theorem \ref{th:invprop}.
\end{problem}

\begin{problem}\label{prob:proveinverseka}
Prove Properties \ref{item:inversekA} and \ref{item:inversetranspose} of Theorem \ref{th:invprop}.
\end{problem}

\begin{problem}\label{prob:illustratematinverse} Use the following invertible matrices to illustrate Properties \ref{item:inverseofinverse}, \ref{item:inverseofproduct}, \ref{item:inversekA} and \ref{item:inversetranspose} of Theorem \ref{th:invprop}.
$$A=\begin{bmatrix}2&0\\4&-3\end{bmatrix}\quad\text{and}\quad B=\begin{bmatrix}2&-1\\1&5\end{bmatrix}$$
\end{problem}

\begin{problem}\label{prob:solvesysbyinverses}
In Example \ref{ex:inverse3} we found the inverse of $$A=\begin{bmatrix}1&-1&2\\1&1&1\\1&3&-1\end{bmatrix}$$

Use $A^{-1}$ to solve the equation
$$A\vec{x}=\begin{bmatrix}3\\-2\\4\end{bmatrix}$$
Answer:
$$\vec{x}=\begin{bmatrix}\answer{17}\\\answer{-8}\\\answer{-11}\end{bmatrix}$$

\end{problem}

\begin{problem}
If possible, find the inverse of each matrix by using the row-reduction procedure.  If you find that $A$ does not have an inverse, leave the answer matrix blank.
  \begin{problem}\label{prob:findinverse1}
  $$A=\begin{bmatrix}1&-1&2\\-1&0&0\\1&-2&3\end{bmatrix}$$
  $$A^{-1}=\begin{bmatrix}\answer{0} & \answer{-1} & \answer{0} \\ \answer{3} & \answer{1} & \answer{-2}\\ \answer{2} & \answer{1} & \answer{-1}\end{bmatrix}$$
  
 
 \begin{multipleChoice}
  \choice[correct]{$A$ is invertible}
  \choice{$A$ is not invertible}
\end{multipleChoice}
  \end{problem}
  

  \begin{problem}\label{prob:findinverse2}
  $$A=\begin{bmatrix}1&3&5\\-2&5&1\\1&-1&1\end{bmatrix}$$
  $$A^{-1}=\begin{bmatrix}\answer{} &\answer{}&\answer{}\\\answer{}&\answer{}&\answer{}\\\answer{}&\answer{}&\answer{}\end{bmatrix}$$
  
 
 \begin{multipleChoice}
  \choice{$A$ is invertible}
  \choice[correct]{$A$ is not invertible}
\end{multipleChoice}
  \end{problem}  
  
\begin{problem}\label{prob:findinverse3}
  $$A=\begin{bmatrix}2&-1&-2\\1&1&-1\\1&1&1\end{bmatrix}$$
  $$A^{-1}=\begin{bmatrix}\answer{1/3} & \answer{-1/6} & \answer{1/2} \\  \answer{-1/3} & \answer{2/3} & \answer{0}\\\answer{0} & \answer{-1/2} & \answer{1/2}\end{bmatrix}$$
  
 
 
 \begin{multipleChoice}
  \choice[correct]{$A$ is invertible}
  \choice{$A$ is not invertible}
\end{multipleChoice}
  \end{problem}   
  
\end{problem}


\begin{problem}\label{prob:inverseformula}
Use the row-reduction method to prove Formula \ref{form:detinverse} for a nonsingular matrix.  Show that if $ad-bc=0$ then $A$ does not have an inverse.
\begin{hint}After going through the row reduction, try it again, considering the possibility that $a=0$.
\end{hint}
\end{problem}


\begin{problem}\label{prob:useformforinv}
Use Formula \ref{form:detinverse} to find the inverse of $$A=\begin{bmatrix}2&-3\\1&5\end{bmatrix}$$
Use $A^{-1}$ to solve the equation
$$A\vec{x}=\begin{bmatrix}-3\\2\end{bmatrix}$$
Answer:
$$\vec{x}=\begin{bmatrix}\answer{-9/13}\\\answer{7/13}\end{bmatrix}$$

\end{problem}

\begin{problem}
For each matrix below refer to  Formula \ref{form:detinverse} to find the value of $x$ for which the matrix is not invertible.  
\begin{problem}\label{prob:notinv1}
$$\begin{bmatrix}1&2\\4&x\end{bmatrix}$$
Answer: $x=\answer{8}$
\end{problem}
\begin{problem}\label{prob:notinv2}
$$\begin{bmatrix}3&2\\3&x\end{bmatrix}$$
Answer: $x=\answer{2}$
\end{problem}
\begin{problem}\label{prob:notinv3}
$$\begin{bmatrix}5&4\\-5&x\end{bmatrix}$$
Answer: $x=\answer{-4}$
\end{problem}
\end{problem}



\begin{problem}\label{prob:cancelprop}
Suppose $AB=AC$ and $A$ is an invertible $n\times n$ matrix. Does it
follow that $B=C?$ Explain why or why not.
\end{problem}

\begin{problem}\label{prob:cancelpropsingular}
Suppose $AB=AC$ and $A$ is a non-invertible $n\times n$ matrix. Does it
follow that $B=C?$ Explain why or why not.
\end{problem}


\begin{problem}\label{prob:Asquaredid}
Give an example of a matrix $A$ such that $A^{2}=I$ and yet $A\neq I$
and $A\neq -I.$
\end{problem}

\begin{problem}\label{prob:invofsymm}
Suppose $A$ is a symmetric, invertible matrix.  Does it follow that $A^{-1}$ is symmetric?  What if we change ``symmetric" to ``skew symmetric"?  (See Definition \ref{def:symmetricandskewsymmetric} of MAT-0025.)
\end{problem}

\begin{problem}\label{prob:sumofinvertible} Suppose $A$ and $B$ are invertible $n\times n$ matrices.  Does it follow that $(A+B)$ is invertible?
\end{problem}

\end{document}
