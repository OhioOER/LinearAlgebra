\documentclass{ximera}
%% You can put user macros here
%% However, you cannot make new environments

\graphicspath{{./}{firstExample/}{secondExample/}}

\usepackage{tikz}
\usepackage{tkz-euclide}
\usetkzobj{all}
\pgfplotsset{compat=1.7} % prevents compile error.

\tikzstyle geometryDiagrams=[ultra thick,color=blue!50!black]


\author{Anna Davis \and Paul Zachlin} \title{VEC-0090: Span} \license{CC-BY 4.0}

\begin{document}

\begin{abstract}
 We define the span of a collection of vectors and explore the concept algebraically and geometrically.
\end{abstract}
\maketitle

\section*{VEC-0090: Span}

\subsection*{Linear Combinations Revisited}
Recall that a vector $\vec{v}$ is said to be a linear combination of vectors $\vec{v}_1, \vec{v}_2,\ldots, \vec{v}_n$ if 
$$\vec{v}=a_1\vec{v}_1+ a_2\vec{v}_2+\ldots + a_n\vec{v}_n$$
for some scalars $a_1, a_2, \ldots ,a_n$.


\begin{example}\label{ex:spanintro}
If possible, express the given vector as a linear combination of $$\vec{v}_1=\begin{bmatrix}-2\\-3\\4\end{bmatrix},\quad\vec{v}_2=\begin{bmatrix}2\\3\\2\end{bmatrix}$$  Interpret your results geometrically.

  \begin{enumerate}
  \item \label{item:spanintro1} 
  $$\vec{u}=\begin{bmatrix}2\\3\\5\end{bmatrix}$$
  
  
  \item \label{item:spanintro2}
  $$\vec{w}=\begin{bmatrix}5\\5\\1\end{bmatrix}$$
  \end{enumerate}
  
  \begin{explanation}
  \ref{item:spanintro1} We need to find coefficients $a_1$ and $a_2$ such that $\vec{u}=a_1\vec{v}_1+a_2\vec{v}_2$. To do this we need to solve the vector equation:
  $$a_1\begin{bmatrix}-2\\-3\\4\end{bmatrix}+a_2\begin{bmatrix}2\\3\\2\end{bmatrix}=\begin{bmatrix}2\\3\\5\end{bmatrix}$$
  This equation translates into the following system:
  
  $$\begin{array}{ccccc}
      -2a_1 & +&2a_2&= &2 \\
        -3a_1& +&3a_2&= &3 \\
      4a_1 &+ &2a_2&= &5\\
	     \end{array}$$
  We write the system in augmented matrix form and apply elementary row operations to bring it to reduced row-echelon form.
  $$\left[\begin{array}{cc|c}  
 -2&2&2\\-3&3&3\\4&2&5
 \end{array}\right]\rightsquigarrow\left[\begin{array}{cc|c}  
 1&0&1/2\\0&1&3/2\\0&0&0
 \end{array}\right]$$
  
  This shows that $a_1=\frac{1}{2}$ and $a_2=\frac{3}{2}$, and we can express $\vec{u}$ as a linear combination of $\vec{v}_1$ and $\vec{v}_2$ as follows:
  $$\frac{1}{2}\begin{bmatrix}-2\\-3\\4\end{bmatrix}+\frac{3}{2}\begin{bmatrix}2\\3\\2\end{bmatrix}=\begin{bmatrix}2\\3\\5\end{bmatrix}$$
    
  Observe that because vector $\vec{u}$ is a linear combination of $\vec{v}_1$ and $\vec{v}_2$, $\vec{u}$ is the diagonal of a parallelogram whose sides are scalar multiples of $\vec{v}_1$ and $\vec{v}_2$. As such, $\vec{u}$ lies in the same plane as $\vec{v}_1$ and $\vec{v}_2$, as illustrated below.
  
  \begin{center}
\tdplotsetmaincoords{70}{130}
\begin{tikzpicture}[scale=0.6]
	\draw[->](-6,0,0)--(6,0,0) node[below left]{$y$};
    \draw[->](0,-2,0)--(0,8,0) node[below left]{$z$};
    \draw[->](0,0,-2)--(0,0,6) node[below left]{$x$};
    
    \filldraw[blue, opacity=0.3](0,0,0)--(4.5,3,3)--(3,5,2)--(-1.5,2,-1)--cycle;
    \draw[->, line width=2pt,blue, -stealth](0,0,0)--(-3,4,-2)node[above left]{$\vec{v}_1=\begin{bmatrix}-2\\-3\\4\end{bmatrix}$};
    \draw[->, line width=2pt,red, -stealth](0,0,0)--(3,2,2)node[right]{$\vec{v}_2=\begin{bmatrix}2\\3\\2\end{bmatrix}$};
    \draw[->, line width=2pt, -stealth](0,0,0)--(3,5,2)node[above]{$\vec{u}=\begin{bmatrix}2\\3\\5\end{bmatrix}$};
    
    \end{tikzpicture}
\end{center}
  
  
%We say that $\vec{u}$ is \dfn{in the span of} $\vec{v}_1$ and $\vec{v}_2$.
  
  
\ref{item:spanintro2}  
We need to solve the following vector equation:   $$a_1\begin{bmatrix}-2\\-3\\4\end{bmatrix}+a_2\begin{bmatrix}2\\3\\2\end{bmatrix}=\begin{bmatrix}5\\5\\1\end{bmatrix}$$
  This equation corresponds to the system:
  
  $$\begin{array}{ccccc}
      -2a_1 & +&2a_2&= &5 \\
        -3a_1& +&3a_2&= &5 \\
      4a_1 &+ &2a_2&= &1\\
	     \end{array}$$
  Writing the system in augmented matrix form and applying elementary row operations gives us the following reduced row-echelon form:
  $$\left[\begin{array}{cc|c}  
 -2&2&5\\-3&3&5\\4&2&1
 \end{array}\right]\rightsquigarrow\left[\begin{array}{cc|c}  
 1&0&0\\0&1&0\\0&0&1
 \end{array}\right]$$
 We conclude that there are no solutions, and $\vec{w}$ is not a linear combination of $\vec{v}_1$ and $\vec{v}_2$.
 
 Geometrically, this means that $\vec{w}$ is not the diagonal of any parallelogram whose sides are scalar multiples of $\vec{v}_1$ and $\vec{v}_2$.  Thus, $\vec{w}$ does not lie in the plane determined by $\vec{v}_1$ and $\vec{v}_2$.  
 %We say that $\vec{w}$ is not \dfn{in the span} of $\vec{v}_1$ and $\vec{v}_2$.
 
  \begin{center}
\tdplotsetmaincoords{70}{130}
\begin{tikzpicture}[scale=0.6]
	\draw[->](-6,0,0)--(6,0,0) node[below left]{$y$};
    \draw[->](0,-2,0)--(0,8,0) node[below left]{$z$};
    \draw[->](0,0,-2)--(0,0,6) node[below left]{$x$};
    
    
    \draw[->, line width=2pt,blue, -stealth](0,0,0)--(-3,4,-2)node[above left]{$\vec{v}_1=\begin{bmatrix}-2\\-3\\4\end{bmatrix}$};
   
   \draw[->, line width=2pt,red, -stealth](0,0,0)--(3,2,2)node[right]{$\vec{v}_2=\begin{bmatrix}2\\3\\2\end{bmatrix}$};
    
    \draw[->, line width=2pt, -stealth](0,0,0)--(5,1,5)node[below]{$\vec{w}=\begin{bmatrix}5\\5\\1\end{bmatrix}$};

\end{tikzpicture}
\end{center}
  
  \end{explanation}
\end{example}

In part \ref{item:spanintro1} of Example \ref{ex:spanintro} we expressed $\vec{u}$ as a linear combination of $\vec{v}_1$ and $\vec{v}_2$, and concluded that $\vec{u}$ lies in the plane determined by $\vec{v}_1$ and $\vec{v}_2$.  We will now introduce a new vocabulary term and say that $\vec{u}$ is \dfn{in the span} of $\vec{v}_1$ and $\vec{v}_2$.  In fact, every vector in the plane determined by $\vec{v}_1$ and $\vec{v}_2$ is in the span of $\vec{v}_1$ and $\vec{v}_2$.  We say that $\vec{v}_1$ and $\vec{v}_2$ \dfn{span the plane}.

In contrast, vector $\vec{w}$ of part \ref{item:spanintro2} of Example \ref{ex:spanintro} is not a linear combination of $\vec{v}_1$ and $\vec{v}_2$.  We say that $\vec{w}$ is not in the span of $\vec{v}_1$ and $\vec{v}_2$.

The following video takes another look at Example \ref{ex:spanintro} using our new vocabulary.

\youtube{Sum3qoOEXhY}

\subsection*{Definition of Span}

\begin{definition}\label{def:span} Let $\vec{v}_1, \vec{v}_2,\ldots ,\vec{v}_p$ be vectors in $\RR^n$.  The set $S$ of all linear combinations of $\vec{v}_1, \vec{v}_2,\ldots ,\vec{v}_p$ is called the \dfn{span} of $\vec{v}_1, \vec{v}_2,\ldots ,\vec{v}_p$.  We write 
$$S=\mbox{span}(\vec{v}_1, \vec{v}_2,\ldots ,\vec{v}_p)$$
and we say that vectors $\vec{v}_1, \vec{v}_2,\ldots ,\vec{v}_p$ \dfn{span} $S$.  Any vector in $S$ is said to be \dfn{in the span} of $\vec{v}_1, \vec{v}_2,\ldots ,\vec{v}_p$.  The set $\{\vec{v}_1, \vec{v}_2,\ldots ,\vec{v}_p\}$ is called a \dfn{spanning set} for $S$.
\end{definition}

\begin{example}\label{ex:describespan}
Describe $\mbox{span}\left(\begin{bmatrix}-3\\1\end{bmatrix}\right)$.
\begin{explanation}
The span of $\begin{bmatrix}-3\\1\end{bmatrix}$ is the set of all linear combinations of $\begin{bmatrix}-3\\1\end{bmatrix}$.  Since we are looking for linear combinations of only one vector, we are really looking for all of its scalar multiples.  So, the span will be the set of all vectors of the form $\vec{v}=a\begin{bmatrix}-3\\1\end{bmatrix}$.  All such vectors lie on the line determined by $\begin{bmatrix}-3\\1\end{bmatrix}$.

\begin{center}
\begin{tikzpicture}[scale=0.8]
\draw[thin,gray!40] (-6,-4) grid (6,4);
  \draw[<->] (-6,0)--(6,0);
  \draw[<->] (0,-4)--(0,4);
  
 \draw[line width=1pt,-stealth](0,0)--(-6,2);
 \draw[line width=1pt,-stealth](0,0)--(6,-2);
 \draw[line width=1pt,-stealth](0,0)--(3,-1);
 \draw[line width=1pt,-stealth](0,0)--(1.5,-0.5);
 \draw[line width=1pt,-stealth](0,0)--(4.5,-1.5);
 \draw[line width=1pt,-stealth](0,0)--(-4.5,1.5);

 \draw[line width=2pt,red,-stealth](0,0)--(-3,1) node[above right]{$\begin{bmatrix}-3\\1\end{bmatrix}$};

 \end{tikzpicture}
\end{center}


\end{explanation}
\end{example}

\begin{example}\label{ex:spanr2}
Describe $\mbox{span}\left(\begin{bmatrix}2\\2\end{bmatrix}, \begin{bmatrix}-1\\0\end{bmatrix}\right)$.
\begin{explanation}
First, observe that $\begin{bmatrix}2\\2\end{bmatrix}$ and $\begin{bmatrix}-1\\0\end{bmatrix}$ are not scalar multiples of each other.  

\begin{center}
\begin{tikzpicture}[scale=0.8]
\draw[thin,gray!40] (-4,-1) grid (4,4);
  \draw[<->] (-4,0)--(4,0);
  \draw[<->] (0,-1)--(0,4);
  
  \draw[line width=2pt,blue,-stealth](0,0)--(2,2) node[above right]{$\begin{bmatrix}2\\2\end{bmatrix}$};
  \draw[line width=2pt,red,-stealth](0,0)--(-1,0) node[above left]{$\begin{bmatrix}-1\\0\end{bmatrix}$};

 \end{tikzpicture}
\end{center}


%If $\vec{w}$ is a scalar multiple of one of $\begin{bmatrix}2\\2\end{bmatrix}, \begin{bmatrix}-1\\0\end{bmatrix}$, then $\vec{w}$ is in $\mbox{span}\left(\begin{bmatrix}2\\2\end{bmatrix}, \begin{bmatrix}-1\\0\end{bmatrix}\right)$.  
Geometrically, we can use Procedure \ref{pro:lincombgeo} of VEC-0040 to express any vector of $\RR^2$ as a linear combination of $\begin{bmatrix}2\\2\end{bmatrix}$ and  $\begin{bmatrix}-1\\0\end{bmatrix}$.  So, intuitively it makes sense that the two vectors span all of $\RR^2$.

To verify this claim algebraically we will show that an arbitrary vector $\begin{bmatrix}s\\t\end{bmatrix}$ of $\RR^2$ can be written as a linear combination of $\begin{bmatrix}2\\2\end{bmatrix}$ and  $\begin{bmatrix}-1\\0\end{bmatrix}$.  

Consider the vector equation:

$$a_1\begin{bmatrix}2\\2\end{bmatrix}+a_2\begin{bmatrix}-1\\0\end{bmatrix}=\begin{bmatrix}s\\t\end{bmatrix}$$
  This corresponds to the system:
  
  $$\begin{array}{ccccc}
      2a_1 & -&a_2&= &s \\
        2a_1& &&= &t \\
	 \end{array}$$
  Writing the system in augmented matrix form and applying elementary row operations gives us the following reduced row-echelon form:
  $$\left[\begin{array}{cc|c}  
 2&-1&s\\2&0&t
 \end{array}\right]\rightsquigarrow\left[\begin{array}{cc|c}  
 1&0&t/2\\0&1&t-s
 \end{array}\right]$$
This shows that every vector of $\RR^2$ can be written as a linear combination of $\begin{bmatrix}2\\2\end{bmatrix}$ and  $\begin{bmatrix}-1\\0\end{bmatrix}$: 

$$(t/2)\begin{bmatrix}2\\2\end{bmatrix}+(t-s)\begin{bmatrix}-1\\0\end{bmatrix}=\begin{bmatrix}s\\t\end{bmatrix}$$

We conclude that $$\mbox{span}\left(\begin{bmatrix}2\\2\end{bmatrix}, \begin{bmatrix}-1\\0\end{bmatrix}\right)=\RR^2$$
\end{explanation}
\end{example}

\begin{example}\label{ex:spanoftwovectors}
Describe $\mbox{span}\left(\begin{bmatrix}5\\0\\4\end{bmatrix}, \begin{bmatrix}0\\4\\2\end{bmatrix}\right)$.
\begin{explanation}
First, observe that $\begin{bmatrix}5\\0\\4\end{bmatrix}, \begin{bmatrix}0\\4\\2\end{bmatrix}$ are not scalar multiples of each other.  

\begin{center}
\tdplotsetmaincoords{70}{130}
\begin{tikzpicture}
	\draw[->](-2,0,0)--(5,0,0) node[below left]{$y$};
    \draw[->](0,-2,0)--(0,5,0) node[below left]{$z$};
    \draw[->](0,0,-2)--(0,0,5) node[below left]{$x$};
    
    \draw[->, line width=2pt,blue, -stealth](0,0,0)--(0,4,5)node[above right]{$\begin{bmatrix}5\\0\\4\end{bmatrix}$};
    \draw[->, line width=2pt,red, -stealth](0,0,0)--(4,2,0)node[above left]{$\begin{bmatrix}0\\4\\2\end{bmatrix}$};
    
\end{tikzpicture}
\end{center}

The span of $\begin{bmatrix}5\\0\\4\end{bmatrix}$ and $\begin{bmatrix}0\\4\\2\end{bmatrix}$ consists of elements of the form
$$a_1\begin{bmatrix}5\\0\\4\end{bmatrix}+a_2\begin{bmatrix}0\\4\\2\end{bmatrix}$$

Geometrically, we can interpret all such linear combinations as diagonals of parallelograms determined by scalar multiples of $\begin{bmatrix}5\\0\\4\end{bmatrix}$ and $\begin{bmatrix}0\\4\\2\end{bmatrix}$.  All such diagonals will lie in the plane determined by $\begin{bmatrix}5\\0\\4\end{bmatrix}$ and $\begin{bmatrix}0\\4\\2\end{bmatrix}$.  Let this plane be called $p$.  A portion of $p$ is shown below.
\begin{center}
\tdplotsetmaincoords{70}{130}
\begin{tikzpicture}
	\draw[->](-2,0,0)--(5,0,0) node[below left]{$y$};
    \draw[->](0,-2,0)--(0,5,0) node[below left]{$z$};
    \draw[->](0,0,-2)--(0,0,5) node[below left]{$x$};
    \filldraw[blue, opacity=0.3] (0,0,0)--(4, 2, 0)--(4,6,5)--(0, 4, 5)--cycle;
    \draw[->, line width=2pt,blue, -stealth](0,0,0)--(0,4,5);
    \draw[->, line width=2pt,red, -stealth](0,0,0)--(4,2,0);
    
\end{tikzpicture}
\end{center}

Because Procedure \ref{pro:lincombgeo} of VEC-0040 can be applied to vectors that lie in $p$ just as easily as it can be applied to vectors of $\RR^2$, we conclude that every vector in $p$ can be expressed as a linear combination of $\begin{bmatrix}5\\0\\4\end{bmatrix}$ and  $\begin{bmatrix}0\\4\\2\end{bmatrix}$.  Thus, 

$$\mbox{span}\left(\begin{bmatrix}5\\0\\4\end{bmatrix}, \begin{bmatrix}0\\4\\2\end{bmatrix}\right)=p$$

\end{explanation}
\end{example}

\section*{Practice Problems}

\begin{problem}
Choose the best description for each set below.
  \begin{problem}\label{prob:describespan1}
  $$\mbox{span}\left(\begin{bmatrix}1\\1\\-2\end{bmatrix}, \begin{bmatrix}2\\2\\-4\end{bmatrix}\right)$$
  
  \begin{multipleChoice}
 \choice{Plane in $\RR^3$}
 \choice{Line in $\RR^2$}
     \choice[correct]{Line in $\RR^3$}
 \choice{$\RR^3$}
  \choice{$\RR^2$ }
 \end{multipleChoice}
  \end{problem}
  
  \begin{problem}\label{prob:describespan2}
  $$\mbox{span}\left(\begin{bmatrix}1\\-2\end{bmatrix}, \begin{bmatrix}1\\0\end{bmatrix}\right)$$
  
   \begin{multipleChoice}
 \choice{Plane in $\RR^3$}
 \choice{Line in $\RR^2$}
     \choice{Line in $\RR^3$}
 \choice{$\RR^3$}
  \choice[correct]{$\RR^2$ }
  
\end{multipleChoice}
  \end{problem}
  \begin{problem}\label{prob:describespan3}
  $$\mbox{span}\left(\begin{bmatrix}-3\\1\end{bmatrix}, \begin{bmatrix}6\\-2\end{bmatrix}, \begin{bmatrix}3\\-1\end{bmatrix}\right)$$
  
  \begin{multipleChoice}
 \choice{Plane in $\RR^3$}
 \choice[correct]{Line in $\RR^2$}
     \choice{Line in $\RR^3$}
 \choice{$\RR^3$}
  \choice{$\RR^2$ }
\end{multipleChoice}
  \end{problem}
\end{problem}

\begin{problem}\label{prob:equalsets}
Which of the following pairs of sets are equal?  

\begin{selectAll}
  \choice[correct]{$V=\mbox{span}\left(\begin{bmatrix}5\\0\\0\end{bmatrix},\begin{bmatrix}10\\0\\0\end{bmatrix},\begin{bmatrix}0\\0\\-4\end{bmatrix}\right)\quad\text{and}\quad W=\mbox{span}\left(\begin{bmatrix}-2\\0\\0\end{bmatrix},\begin{bmatrix}0\\0\\1\end{bmatrix}\right)$}
 \choice[correct]{$V=\mbox{span}\left(\begin{bmatrix}5\\3\end{bmatrix},\begin{bmatrix}10\\-1\end{bmatrix},\begin{bmatrix}0\\2\end{bmatrix}\right)\quad\text{and}\quad W=\mbox{span}\left(\begin{bmatrix}-2\\0\end{bmatrix},\begin{bmatrix}0\\1\end{bmatrix}\right)$}
 \choice{$V=\mbox{span}\left(\begin{bmatrix}1\\-2\\4\end{bmatrix}\right)\quad\text{and}\quad W=\mbox{span}\left(\begin{bmatrix}-1\\2\\-4\end{bmatrix},\begin{bmatrix}0\\0\\1\end{bmatrix}\right)$}
 \choice[correct]{$V=\mbox{span}\left(\begin{bmatrix}5\\0\end{bmatrix}\right)\quad\text{and}\quad W=\mbox{span}\left(\begin{bmatrix}-2\\0\end{bmatrix},\begin{bmatrix}1\\0\end{bmatrix},\begin{bmatrix}-4\\0\end{bmatrix},\begin{bmatrix}0\\0\end{bmatrix}\right)$}
\end{selectAll}
\end{problem}

\begin{problem}\label{prob:notinspan}
Let $\vec{v}=\begin{bmatrix}3\\4\\5\end{bmatrix}$.  Give an example of at least one vector $\vec{w}$ such that $\vec{v}$, $\vec{w}$ do NOT span a plane in $\RR^3$.  Describe $\mbox{span}(\vec{v}, \vec{w})$.
\end{problem}

\begin{problem}\label{prob:zeroinspan}
Prove or disprove.  The zero vector of $\RR^n$ is contained in the span of any collection of vectors of $\RR^n$.
\end{problem}


\end{document} 