\documentclass{ximera}
%% You can put user macros here
%% However, you cannot make new environments

\graphicspath{{./}{firstExample/}{secondExample/}}

\usepackage{tikz}
\usepackage{tkz-euclide}
\usetkzobj{all}
\pgfplotsset{compat=1.7} % prevents compile error.

\tikzstyle geometryDiagrams=[ultra thick,color=blue!50!black]


\author{Anna Davis \and Paul Zachlin \and Rosemarie Emanuele} \title{DET-0010: Definition of the Determinant -- Expansion Along the First Row} \license{CC-BY 4.0}

\begin{document}

\begin{abstract}
 We define the determinant of a square matrix in terms of cofactor expansion along the first row.
\end{abstract}
\maketitle

\section*{DET-0010: Definition of the Determinant -- Expansion Along the First Row}
In this module we will define a function that assigns to each square matrix $A$ a scalar output called the \dfn{determinant of $A$}.  For matrices with real number entries, the outputs of the determinant function will be real numbers.  We will denote the determinant of $A$ by $\det{A}$.  

One important property of the determinant is its connection to matrix inverses.  We will find that a matrix $A$ is singular if and only if $\det{A}=0$.  For nonsingular matrices, we will establish a formula that gives the inverse of a matrix exclusively in terms of determinants.

\subsection*{Determinants of Square Matrices ($n=1,2,3$)}
To start from the beginning, let us define the determinant of a $1\times 1$ matrix.
\begin{definition}\label{def:onebyonedet} Let
$A=\begin{bmatrix}a\end{bmatrix}$.  Define the \dfn{determinant} of $A$ by $\det{A}=a$.
\end{definition}
It is important to note that this definition is consistent with our goal of making a connection between determinants and invertibility.  Observe that $A^{-1}=\begin{bmatrix}a^{-1}\end{bmatrix}$ exists if and only if $a\neq 0$.

Now we proceed to $2\times 2$ matrices.  According to Formula \ref{form:detinverse} of {\color{red} MAT-M-0050}, the inverse of a nonsingular matrix $A=\begin{bmatrix}a&b\\c&d\end{bmatrix}$ is given by
$$A^{-1}=\frac{1}{ad-bc}\begin{bmatrix}d&-b\\-c&a\end{bmatrix}$$
Observe that $A^{-1}$ exists if and only if $ad-bc\neq 0$.  We will call the number $ad-bc$ the \dfn{determinant} of $A$.

\begin{definition}\label{def:twobytwodet}
Let $A=\begin{bmatrix}a&b\\c&d\end{bmatrix}$.  The determinant of $A$ is defined by 
\begin{equation}\label{eq:twobytwodet}\det{A}=\det{\begin{bmatrix}a&b\\c&d\end{bmatrix}}=\begin{vmatrix}a&b\\c&d\end{vmatrix}=ad-bc\end{equation}
\end{definition}
Note the distinction between the square bracket notation associated with the matrix $\begin{bmatrix}a&b\\c&d\end{bmatrix}$ and the vertical bar notation $\begin{vmatrix}a&b\\c&d\end{vmatrix}$ used to denote  the determinant in expression (\ref{eq:twobytwodet}).

\begin{example}
$$\det{\begin{bmatrix}1&2\\3&4\end{bmatrix}}=\begin{vmatrix}1&2\\3&4\end{vmatrix}=(1)(4)-(2)(3)=-2$$
\end{example}

Our next goal is to define  the determinant for a $3\times 3$ matrix.  Let 
$$A=\begin{bmatrix}a&b&c\\d&e&f\\g&h&i\end{bmatrix}$$

Our definition will require three \dfn{minor} matrices associated with $A$.  
\begin{itemize}
\item $A_{11}$ is obtained from $A$ by deleting the first row and the first column of $A$.
\begin{center}
\begin{tikzpicture}
  \matrix (m)[
    matrix of math nodes,
    nodes in empty cells,
    left delimiter={[},right delimiter={]},minimum width=width("a"),minimum height=height("b")] {
    a    & b  & c  \\
    d & e   & f   \\
    g   & h    & i     \\
  } ;

  \draw (m-3-2.south west) rectangle (m-2-3.north east);
  %\draw[blue](m-1-2.west) -- (m-1-3.east);
  %\draw[blue](m-2-1.north) -- (m-3-1.south);
 \end{tikzpicture}
 \end{center} 
 $$A_{11}=\begin{bmatrix}e&f\\h&i\end{bmatrix}$$
 \item $A_{12}$ is obtained from $A$ by deleting the first row and the second column of $A$.
 \begin{center}
\begin{tikzpicture}
  \matrix (m)[
    matrix of math nodes,
    nodes in empty cells,
    left delimiter={[},right delimiter={]},minimum width=width("a")] {
    a    & b  & c  \\
    d & e   & f   \\
    g   & h    & i     \\
  } ;
\draw (m-3-1.south west) rectangle (m-2-1.north east);
  \draw (m-3-3.south west) rectangle (m-2-3.north east);
 \end{tikzpicture}
 \end{center} 
 $$A_{12}=\begin{bmatrix}d&f\\g&i\end{bmatrix}$$
 \item $A_{13}$ is obtained from $A$ by deleting the first row and the third column of $A$.
 \begin{center}
\begin{tikzpicture}
  \matrix (m)[
    matrix of math nodes,
    nodes in empty cells,
    left delimiter={[},right delimiter={]},minimum width=width("a")] {
    a    & b  & c  \\
    d & e   & f   \\
    g   & h    & i     \\
  } ;
\draw (m-3-1.south west) rectangle (m-2-2.north east);
  
 \end{tikzpicture}
 \end{center} 
 $$A_{13}=\begin{bmatrix}d&e\\g&h\end{bmatrix}$$
\end{itemize}
We are now ready to define the determinant of a $3\times 3$ matrix.
\begin{definition}\label{def:threebythreedet}
Let
$$A=\begin{bmatrix}a&b&c\\d&e&f\\g&h&i\end{bmatrix}$$
The \dfn{determinant} of $A$ is given by
\begin{align*}\det{A}=|A|&=a\begin{vmatrix}e&f\\h&i\end{vmatrix}-b\begin{vmatrix}d&f\\g&i\end{vmatrix}+c\begin{vmatrix}d&e\\g&h\end{vmatrix}\\
&=a\big(\det{A_{11}}\big)-b\big(\det{A_{12}}\big)+c\big(\det{A_{13}}\big)
\end{align*}
\end{definition}

We would like to point out several important features of this definition:
\begin{itemize}
\item The coefficients $a$, $b$ and $c$ in front of determinants of minor matrices are the entries of the first row of matrix $A$.  
\item To remember which minor matrix is associated with each first row entry, cross out the row and column that the entry is in.  For example, minor matrix $A_{12}$ is found by crossing out the row and column that $b$ is in.
\begin{center}
\begin{tikzpicture}
  \matrix (m)[
    matrix of math nodes,
    nodes in empty cells,
    left delimiter={[},right delimiter={]},minimum width=width("a"),minimum height=height("b")] {
    a    & {\color{red}b}  & c  \\
    d & e   & f   \\
    g   & h    & i     \\
  } ;

  %\draw (m-3-2.south west) rectangle (m-2-3.north east);
  \draw[blue](m-1-1.west) -- (m-1-1.east);
   \draw[blue](m-1-3.west) -- (m-1-3.east);
  \draw[blue](m-2-2.north) -- (m-3-2.south);
 \end{tikzpicture}
 \end{center} 
\item Coefficients follow an alternating sign pattern: $+a$, $-b$, $+c$.
\end{itemize}
\begin{example}\label{ex:threebythreedet1}
Find $\det{A}$ if 
$$A=\begin{bmatrix}3&-2&1\\5&-1&2\\1&4&1\end{bmatrix}$$
\begin{explanation}
\begin{align*}
\det{A}&=(3)\begin{vmatrix}-1&2\\4&1\end{vmatrix}-(-2)\begin{vmatrix}5&2\\1&1\end{vmatrix}+(1)\begin{vmatrix}5&-1\\1&4\end{vmatrix}\\
&=(3)(-1-8)-(-2)(5-2)+(1)(20+1)\\
&=-27+6+21\\
&=0
\end{align*}
\end{explanation}
\end{example}

\begin{example}
Find $\det{A}$ if 
$$A=\begin{bmatrix}4&3&-2\\1&-5&3\\-4&1&1\end{bmatrix}$$
\begin{explanation}
$$
\det{A}=(4)\begin{vmatrix}\answer{-5}&\answer{3}\\\answer{1}&\answer{1}\end{vmatrix}-(3)\begin{vmatrix}\answer{1}&\answer{3}\\\answer{-4}&\answer{1}\end{vmatrix}+(-2)\begin{vmatrix}\answer{1}&\answer{-5}\\\answer{-4}&\answer{1}\end{vmatrix}
=\answer{-33}
$$
\end{explanation}
\end{example}

\subsection*{Definition of the Determinant}
Let $A$ be an $n\times n$ matrix. 
$$A=\begin{bmatrix}a_{11} & a_{12} & \dots & a_{1j-1} & a_{1j} & a_{1j+1} & \dots & a_{1n}  \\
    a_{21} & a_{22} & \dots & a_{2j-1} & a_{2j} & a_{2j+1} & \dots & a_{2n}  \\
   \vdots & \vdots &  & \vdots & \vdots & \vdots &  & \vdots  \\
   a_{n1} & a_{n2} & \dots & a_{nj-1} & a_{nj} & a_{nj+1} & \dots & a_{nn}\end{bmatrix}$$
   Define $A_{1j}$ to be an $(n-1)\times (n-1)$ matrix obtained from $A$ by deleting the first row and the $j^{th}$ column of $A$.  We say that $A_{1j}$ is the \dfn{$(1, j)$-minor} of $A$.
\begin{center}
\begin{tikzpicture}
  \matrix (m)[
    matrix of math nodes,
    nodes in empty cells,
    left delimiter={[},right delimiter={]},minimum width=width("a22")] {
    a_{11} & a_{12} & \dots & a_{1j-1} & a_{1j} & a_{1j+1} & \dots & a_{1n}  \\
    a_{21} & a_{22} & \dots & a_{2j-1} & a_{2j} & a_{2j+1} & \dots & a_{2n}  \\
   \vdots & \vdots &  & \vdots & \vdots & \vdots &  & \vdots  \\
   a_{n1} & a_{n2} & \dots & a_{nj-1} & a_{nj} & a_{nj+1} & \dots & a_{nn}  \\
  } ;
\draw (m-4-1.south west) rectangle (m-2-4.north east);
\draw (m-4-6.south west) rectangle (m-2-8.north east);
 
 %\draw[blue](m-4-5.south) -- (m-2-5.north); 
 %\draw[blue](m-1-1.west) -- (m-1-4.east);
 %\draw[blue](m-1-6.west) -- (m-1-8.east);
 \end{tikzpicture}
 \end{center} 
Recall that for a $3\times 3$ matrix $A$ we have
$$\det{A}=a_{11}\big(\det{A_{11}}\big)-a_{12}\big(\det{A_{12}}\big)+a_{13}\big(\det{A_{13}}\big)$$
We want to follow the same pattern  to define the determinant of a larger matrix.  A distinct feature of this expression is the alternating sign pattern.  We want to preserve this feature as we increase matrix size. Before we present a formal definition, let's take a look at an example.
\begin{example}\label{ex:expansiontoprow}
Find $\det{A}$ if 
$$A=\begin{bmatrix}4&-1&2&1\\3&0&1&-2\\
2&1&5&1\\-2&1&3&-1\end{bmatrix}$$
\begin{explanation}
\begin{align*}
\det{A}&=(4)\det{A_{11}}-(-1)\det{A_{12}}+(2)\det{A_{13}}-(1)\text{det}(A_{14})\\
&=4\begin{vmatrix}0&1&-2\\1&5&1\\1&3&-1\end{vmatrix}+\begin{vmatrix}3&1&-2\\2&5&1\\-2&3&-1\end{vmatrix}+2\begin{vmatrix}3&0&-2\\2&1&1\\-2&1&-1\end{vmatrix}-\begin{vmatrix}3&0&1\\2&1&5\\-2&1&3\end{vmatrix}\\&=4(\answer{6})+(\answer{-56})+2(\answer{-14})-(\answer{-2})\\
&=\answer{-58}
\end{align*}
\end{explanation}
\end{example}


Now we are ready to give the general formula. Pay close attention to how we will use subscripts to capture the alternating sign pattern.

\begin{definition}\label{def:toprowexpansion}  Let $A=\begin{bmatrix}a_{ij}\end{bmatrix}$ be an $n\times n$ matrix.  Define the \dfn{determinant} of $A$ by
\begin{align*}\det{A}=(-1)^{1+1}a_{11}\det{A_{11}}&+(-1)^{1+2}a_{12}\det{A_{12}}+\ldots \\
\ldots &+(-1)^{1+j}a_{1j}\det{A_{1j}}+\ldots \\
\ldots &+(-1)^{1+n}a_{1n}\det{A_{1n}}\\
=\sum_{j=1}^n(-1)^{1+j}a_{1j}\det{A_{1j}}
\end{align*}
\end{definition}
Naturally, we would like to condense this formula.  To accomplish this, let
$$C_{1j}=(-1)^{1+j}\det{A_{1j}}$$
We will refer to $C_{1j}$ as the \dfn{$(1,j)$-cofactor of $A$}.  When we use the cofactor notation, the expression in Definition \ref{def:toprowexpansion} turns into the following:
$$\det{A}=a_{11}C_{11}+a_{12}C_{12}+\ldots +a_{1n}C_{1n}=\sum_{j=1}^n a_{1j}C_{1j}$$

The process of computing the determinant given by Definition \ref{def:toprowexpansion} is called the \dfn{cofactor expansion along the first row}.  We will later show that we can expand along any row or column of a matrix and obtain the same value.  This surprising result, known as the Laplace Expansion Theorem, will be the subject of {\color{red} DET-M-0050}.

We conclude this section with a simple, but useful lemma.

\begin{lemma}\label{lemma:detofid} Let $I$ be the identity matrix, then
$$\det{I}=1$$
\end{lemma}
\begin{proof} Left to the reader.
\end{proof}

%\begin{lemma}\label{lemma:triangulardet}
%{\color{red}NEED LEMMA ON TRIANGULAR MATRICES}
%\end{lemma}
\section*{Practice Problems}

\begin{problem}
Find the determinant of each matrix.
  \begin{problem}
  $$A=\begin{bmatrix}4&-2\\3&7\end{bmatrix}$$
  Answer:
  $$\det{A}=\answer{34}$$
  \end{problem}
  
  \begin{problem}
  $$B=\begin{bmatrix}5&-1&0\\0&3&-2\\1&-1&2\end{bmatrix}$$
  Answer:
  $$\text{det}(B)=\answer{22}$$
  \end{problem}
\end{problem}

\begin{problem}
Show that Definition \ref{def:toprowexpansion} is consistent with Definition \ref{def:twobytwodet} by verifying that both produce the same result when applied to a $2\times 2$ matrix.
\end{problem}

\begin{problem}
Let $B'$ be a matrix obtained from $B$ of Problem {\color{red} reference} by switching the first and the second row of $B$.  Compute the determinant of $B'$.  What do you observe?
\end{problem}

\begin{problem}
Make a conjecture about what happens to the determinant of a matrix if two rows of a matrix are switched.  Prove your conjecture for a $2\times 2$ matrix.
\end{problem}

\begin{problem}Let $B'$ be a matrix obtained from $B$ of Problem {\color{red} reference} by multiplying the middle row by $-3$.  Compute the determinant of $B'$.  What do you observe?
\end{problem}

\begin{problem}
Make a conjecture about what happens to the determinant of a matrix if one of the rows is multiplied by a constant.  Prove your conjecture for a $2\times 2$ matrix.
\end{problem}

\begin{problem}Let $B'$ be a matrix obtained from $B$ of Problem {\color{red} reference} by multiplying $B$ by $2$.  Compute the determinant of $B'$.  What do you observe?
\end{problem}

\begin{problem}
Make a conjecture about what happens to the determinant of a matrix if the matrix is multiplied by a constant.  Prove your conjecture for a $2\times 2$ matrix.
\end{problem}

\begin{problem}
Let $B'$ be a matrix obtained from $B$ of Problem {\color{red} reference} by adding twice the third row to the first.  Compute the determinant of $B'$.  What do you observe?
\end{problem}

\begin{problem}
Make a conjecture about what happens to the determinant of a matrix if a multiple of one row is added to another row.  Prove your conjecture for a $2\times 2$ matrix.
\end{problem}

\begin{problem}
Is it true that 
$\det{(A+B)}=\det{A}+\det{B}$?
\end{problem}

\begin{problem}
Use mathematical induction to prove Lemma \ref{lemma:detofid}.
\end{problem}
\end{document} 