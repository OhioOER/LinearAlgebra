\documentclass{ximera}


\author{Anna Davis \and Rosemarie Emanuele \and Paul Bender} \title{VEC-0020:  Length of a Vector} \license{CC-BY 4.0}
%% You can put user macros here
%% However, you cannot make new environments

\graphicspath{{./}{firstExample/}{secondExample/}}

\usepackage{tikz}
\usepackage{tkz-euclide}
\usetkzobj{all}
\pgfplotsset{compat=1.7} % prevents compile error.

\tikzstyle geometryDiagrams=[ultra thick,color=blue!50!black]

\begin{document}

\begin{abstract}
 We find vector magnitude.
\end{abstract}
\maketitle

\section*{VEC-0020:  Length of a Vector}

Vector quantities, such as velocity and force, have magnitude and direction.  The magnitude of a vector quantity is the length of the vector.  For example, if a force of 10 Newtons is applied to an object, we would represent the force by a 10-unit-long vector.

 \begin{center}
\begin{tikzpicture}[scale=1]
\draw[line width=2pt,-stealth](0,0)--(5,1)node[right]{$\vec{F}$} ;
\node[] at (2.5, 0.8)   {$10$};
\end{tikzpicture}
\end{center}

The magnitude of a vector is denoted by double absolute value brackets.  In the case of force $\vec{F}$, we write $$\norm{\vec{F}}=10$$

To find the length of a vector, we need to find the distance between the tail of the vector and its head.  Recall that in $\RR^2$, the distance between $A(a_1, a_2)$ and $B(b_1, b_2)$ is given by 
\begin{equation*}
d_{AB}=\sqrt{(a_1-b_1)^2+(a_2-b_2)^2}
\end{equation*}
A vector $\vec{v}=\begin{bmatrix}v_1\\ v_2\end{bmatrix}$ has the length of  the vector in standard position with its head at $(v_1, v_2)$ and tail at $(0, 0)$. We find the length of $\vec{v}$ using the distance formula

\begin{equation}\label{eq:normr2}
\norm{\vec{v}}=\sqrt{(v_1-0)^2+(v_2-0)^2}=\sqrt{v_1^2+v_2^2}
\end{equation}

\begin{example}\label{ex:findmaginr2} Find the magnitude of $\vec{u}=\begin{bmatrix}-3\\4\end{bmatrix}$.  
\begin{center}
\begin{tikzpicture}[scale=0.8]
\draw[thin,gray!40] (-4,-1) grid (2,5);
  \draw[<->] (-4,0)--(2,0);
  \draw[<->] (0,-1)--(0,5);
  \draw[line width=1pt,-stealth](0,0)--(-3,4) node[above right]{$\vec{u}=[-3,4]$};
 \end{tikzpicture}
\end{center}
\begin{explanation}

 $$
\norm{\vec{u}}=\sqrt{(-3)^2+(4)^2}=5
$$
\end{explanation}
\end{example}


The distance formula for points in $\RR^3$ is analogous to the distance formula in $\RR^2$.  Given two points $A(a_1, a_2, a_3)$ and $B(b_1, b_2, b_3)$, the distance between them is given by 

\begin{equation*}
d_{AB}=\sqrt{(a_1-b_1)^2+(a_2-b_2)^2+(a_3-b_3)^2}
\end{equation*}

To find the length of vector $\vec{v}=\begin{bmatrix}v_1\\ v_2\\ v_3\end{bmatrix}$, we find the distance between $(v_1, v_2, v_3)$ and $(0, 0, 0)$.
\begin{equation}\label{eq:normr3}
\norm{\vec{v}}=\sqrt{(v_1-0)^2+(v_2-0)^2+(v_3-0)^2}=\sqrt{v_1^2+v_2^2+v_3^2}
\end{equation}

\begin{example}\label{ex:findmaginr3}
The magnitude of $\vec{v}=\begin{bmatrix}6\\-2\\3\end{bmatrix}$ is given by
$$
\norm{\vec{v}}=\sqrt{(6)^2+(-2)^2+(3)^2}=7
$$
\end{example}


Distance formulas for $\RR^2$ and $\RR^3$ motivate the following definition of distance between two points in $\RR^n$.

  \begin{definition}\label{def:distrn} Let $A(a_1, a_2,\ldots, a_n)$ and $B(b_1, b_2,\ldots, b_n)$ be points in $\RR^n$.  The \dfn{distance} between $A$ and $B$ is defined to be
\begin{equation*}
d_{AB}=\sqrt{(a_1-b_1)^2+(a_2-b_2)^2+\ldots +(a_n-b_n)^2}
\end{equation*}
\end{definition}

The following definition follows directly from the distance formula for $\RR^n$ in the same way that expressions (\ref{eq:normr2}) and (\ref{eq:normr3}) followed from distance formulas in $\RR^2$ and $\RR^3$.  
\begin{definition}\label{def:normrn}
Let $\vec{v}=\begin{bmatrix}v_1\\ v_2\\ \vdots \\v_n\end{bmatrix}$ be a vector in $\RR^n$, then the \dfn{length}, or the \dfn{magnitude}, of $\vec{v}$ is given by
\begin{equation} \label{eq:normrn} 
\norm{\vec{v}}=\sqrt{v_1^2+v_2^2+\ldots +v_n^2}
\end{equation}
\end{definition}

\begin{example}\label{ex:findnorminr4}
The magnitude of $\vec{w}=\begin{bmatrix}5\\ -1\\ -2\\ 4\end{bmatrix}$ is given by
 $$\norm{\vec{w}}=\sqrt{5^2+(-1)^2+(-2)^2+4^2}=\sqrt{46}$$ 
\end{example}

\section*{Practice Problems}
\begin{problem}%\label{prob:magnitude}
Find the length of the following vectors.  Enter your answers in exact, simplified form. (e.g. $5\sqrt{3}$)
\begin{problem}\label{prob:magnitude1}
 $$\vec{u}=\begin{bmatrix}1\\3\end{bmatrix}$$ 
 Answer:
 $$\norm{\vec{u}}=\sqrt{\answer{10}}$$
 \end{problem}
 
 \begin{problem}\label{prob:magnitude2}
 $$\vec{v}=\begin{bmatrix}-4\\-6\end{bmatrix}$$
 Answer: Enter your answer in the most simplified exact form. (e.g. $a\sqrt{b}$)
 $$\norm{\vec{v}}=\answer{2}\sqrt{\answer{13}}$$
 \end{problem}
 
 \begin{problem}\label{prob:magnitude3}
 $$\vec{w}=\begin{bmatrix}2\\-4\\6\end{bmatrix}$$
 Answer: Enter your answer in the most simplified exact form. (e.g. $a\sqrt{b}$)
 $$\norm{\vec{w}}=\answer{2}\sqrt{\answer{14}}$$
 \end{problem}
 
 \begin{problem}\label{prob:magnitude4}
 $$\vec{q}=\begin{bmatrix}-10\\ 4\\ -7\\ 5\\ 3\end{bmatrix}$$
 Answer:
 $$\norm{\vec{q}}=\sqrt{\answer{199}}$$
 \end{problem}
 \end{problem}
 
\begin{problem}\label{prob:compformgivenmag}
Find the component form of vector $\vec{v}$ in $\RR^2$ if we know that $\norm{\vec{v}}=15$, the $x$ component of $\vec{v}$ is $-9$ and the vector is located in the third quadrant.

Answer:
$$\vec{v}=\begin{bmatrix}-9\\\answer{-12}\end{bmatrix}$$
\end{problem}

\begin{problem}\label{prob:comppossibilities}
For a vector in $\RR^2$ with a length of 26 and the $y$ component of 10, what are the possibilities for the $x$ component?  List the possibilities in an increasing order.

Answer:
$$\answer{-24},\quad\text{and}\quad\answer{24} $$
\end{problem}

\begin{problem}\label{prob:yvaluesgivennorm}
Let $\vec{v}=\begin{bmatrix}1\\ y\\ 4\end{bmatrix}$.  Find all possible values of $y$ if $\norm{\vec{v}}=9$.  List the possibilities in an increasing order.

Answer:
$$\answer{-8},\quad\text{and}\quad\answer{8}$$
\end{problem}

\begin{problem}\label{prob:comppossibilitiesinr4}
For a vector in $\RR^4$, what are the possibilities for the fourth component if the length of the vector is 14, and the $x$, $y$ and $z$ components are 1, 5 and 13, respectively? List the possibilities in an increasing order.

Answer:
$$\answer{-1},\quad\text{and}\quad\answer{1}$$
\end{problem}
 
 



\end{document} 