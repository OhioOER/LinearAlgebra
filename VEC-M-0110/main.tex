\documentclass{ximera}


\author{Anna Davis \and Paul Zachlin} \title{VEC-M-0110: Linear Independence and Matrices} \license{CC-BY 4.0}
%% You can put user macros here
%% However, you cannot make new environments

\graphicspath{{./}{firstExample/}{secondExample/}}

\usepackage{tikz}
\usepackage{tkz-euclide}
\usetkzobj{all}
\pgfplotsset{compat=1.7} % prevents compile error.

\tikzstyle geometryDiagrams=[ultra thick,color=blue!50!black]

\begin{document}

\begin{abstract}
We prove several results concerning linear independence of rows and columns of a matrix.
\end{abstract}
\maketitle


\section*{VEC-M-0110: Linear Independence and Matrices}
\subsection{Results Concerning Row-Echelon Forms of a Matrix}

Recall that a matrix (or augmented matrix) is in \dfn{row-echelon form} if:
\begin{itemize}
\item All entries {\it below} each leading entry are $0$.
\item Each leading entry is in a column to the right of the leading entries in the rows above it.
\item All rows of zeros, if there are any, are located below non-zero rows.
\end{itemize}

A matrix in row-echelon form is said to be in \dfn{reduced row-echelon form} if it has the following additional properties
\begin{itemize}
\item Each leading entry is a $1$
\item All entries {\it above} each leading $1$ are $0$
\end{itemize}

Every matrix can be brought to reduced row-echelon form using the Gauss-Jordan Algorithm (See Module SYS-M-0030).

\begin{theorem}\label{th:rowsrreflinind}
Let $A$ be a matrix.  The nonzero rows of $\mbox{rref}(A)$ are linearly independent.
\end{theorem}
\begin{proof}
Every nonzero row of $\mbox{rref}(A)$ contains a leading $1$.  All entries above and below the leading $1's$ are $0$.  Thus, no nonzero row of $\mbox{rref}(A)$ can be written as a linear combination of the other rows.  Therefore, by Theorem \ref{th:redundantifflindep} of VEC-M-0100, the nonzero rows of $\mbox{rref}(A)$ are linearly independent.
\end{proof}

\begin{theorem}\label{th:rowsofreflinind}
Let $A$ be a matrix.  The nonzero rows of a row-echelon form of $A$ are linearly independent.
\end{theorem}
\begin{proof}
See Practice Problem \ref{prob:proofofrowsofreflinind}
\end{proof}

\begin{theorem}\label{th:pivotcolslinind}
Pivot columns of a matrix in row-echelon form are linearly independent.
\end{theorem}
\begin{proof}
See Practice Problem \ref{prob:proofofpivotcolslinind}.
\end{proof}

\begin{theorem}\label{th:linindandrank}
Let $A$ be an $m\times n$ matrix.
\begin{enumerate}
    \item Columns of $A$ are linearly independent if and only if $\mbox{rank}(A)=n$.
    \item Rows of $A$ are linearly independent if and only if  $\mbox{rank}(A)=m$.
\end{enumerate}
\end{theorem}
\begin{proof}
See Practice Problem \ref{prob:proofoflinindandrank}
\end{proof}


\subsection{Results Concerning Nonsingular Matrices}
Recall that a square matrix is said to be \dfn{nonsingular} provided that its reduced row-echelon form is equal to the identity matrix.  In MAT-M-0050 we showed that a square matrix is nonsingular if and only if it is invertible. (See Corollary \ref{cor:rrefI} of MAT-M-0050)

\begin{theorem}\label{th:linindcolsnonsingular}
An $n\times n$ matrix $A$ is nonsingular if and only if the columns (rows) of $A$ are linearly independent.
\end{theorem}
\begin{proof}
By Theorem \ref{th:linindandrank}, a square matrix has linearly independent columns and linearly independent rows if and only if its rank is equal to the number of columns (rows).  This means that a square matrix has linearly independent columns and linearly independent rows if and only if the matrix is nonsingular.
\end{proof}


\section*{Practice Problems}
\begin{problem}\label{prob:proofofrowsofreflinind}
Prove Theorem \ref{th:rowsofreflinind}.
\end{problem}

\begin{problem}\label{prob:proofofpivotcolslinind}
Prove Theorem \ref{th:pivotcolslinind}.
\end{problem}

\begin{problem}\label{prob:proofoflinindandrank}
Prove Theorem \ref{th:linindandrank}.
\end{problem}

\end{document} 