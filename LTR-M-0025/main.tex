\documentclass{ximera}

\author{Anna Davis \and Paul Zachlin} \title{Linear Transformations and Basis} \license{CC-BY 4.0}

\renewcommand{\vec}[1]{{\bf #1}}
\newcommand{\RR}{\mathbb{R}}
\newcommand{\dfn}{\textit}
\newcommand{\dotp}{\cdot}

\newtheorem{general}{Generalization}
\newtheorem{initprob}{Exploration Problem}
\usepackage{tikz-cd}
\pgfplotsset{compat=1.14}



\begin{document}
\begin{abstract}
We define linear transformation for vector spaces, and  establish that a linear transformation of a vector space is completely determined by its effect on a basis. 
\end{abstract}
\maketitle

\section*{Linear Transformations and Bases}

Recall that a transformation $T:\mathbb{R}^n\rightarrow \mathbb{R}^m$ is called a {\it linear transformation} if the following are true for all vectors ${\bf u}$ and ${\bf v}$ in $\mathbb{R}^n$, and scalars $k$.
\begin{equation*}
T(k{\bf u})= kT({\bf u})
\end{equation*}
\begin{equation*}
T({\bf u}+{\bf v})= T({\bf u})+T({\bf v})
\end{equation*}

We generalize this definition as follows.

\begin{definition}\label{def:lintransgeneral}
Let $V$ and $W$ be vector spaces. A transformation $T:V\rightarrow W$ is called a {\it linear transformation} if the following are true for all vectors ${\bf u}$ and ${\bf v}$ in $V$, and scalars $k$.
\begin{equation*}
T(k{\bf u})= kT({\bf u})
\end{equation*}
\begin{equation*}
T({\bf u}+{\bf v})= T({\bf u})+T({\bf v})
\end{equation*}
\end{definition}



\begin{initprob}\label{init:tij}  Suppose we want to define a linear transformation $T:\RR^2\rightarrow \RR^2$ by $$T(\vec{i})=\begin{bmatrix}3\\-2\end{bmatrix}\quad\text{and}\quad T(\vec{j})=\begin{bmatrix}-1\\2\end{bmatrix}$$  
Is this information sufficient to define $T$?  
To answer this question we will try to determine what $T$ does to an arbitrary vector of $\RR^2$.  If $\vec{v}$ is a vector in $\RR^2$, then $\vec{v}$ can be uniquely expressed as a linear combination of $\vec{i}$ and $\vec{j}$
$$\vec{v}=a\vec{i}+b\vec{j}$$  By linearity of $T$ we have $$T(\vec{v})=T(a\vec{i}+b\vec{j})=aT(\vec{i})+bT(\vec{j})=a\begin{bmatrix}3\\-2\end{bmatrix}+b\begin{bmatrix}-1\\2\end{bmatrix}$$
This shows that the image of any vector of $\RR^2$ under $T$ is completely determined by the action of $T$ on the standard unit vectors $\vec{i}$ and $\vec{j}$.  

Vectors $\vec{i}$ and $\vec{j}$ form a standard basis of $\RR^2$.  What if we want to use a different basis?  

Let $\mathcal{B}=\left\{\begin{bmatrix}1\\1\end{bmatrix},\begin{bmatrix}2\\-1\end{bmatrix}\right\}$ be our basis of choice for $\RR^2$. (How would you verify that $\mathcal{B}$ is a basis of $\RR^2$?) And suppose we want to define a linear transformation $S:\RR^2\rightarrow \RR^2$ by $$S\left(\begin{bmatrix}1\\1\end{bmatrix}\right)=\begin{bmatrix}0\\-1\end{bmatrix}\quad\text{and}\quad S\left(\begin{bmatrix}2\\-1\end{bmatrix}\right)=\begin{bmatrix}2\\0\end{bmatrix}$$
Is this enough information to define $S$?

Because $\begin{bmatrix}1\\1\end{bmatrix},\begin{bmatrix}2\\-1\end{bmatrix}$ span $\RR^2$, every element $\vec{v}$ of $\RR^2$ can be written as a unique linear combination $$\vec{v}=a\begin{bmatrix}1\\1\end{bmatrix}+b\begin{bmatrix}2\\-1\end{bmatrix}$$
We can find $S(\vec{v})$ as follows:
$$S(\vec{v})=S\left(a\begin{bmatrix}1\\1\end{bmatrix}+b\begin{bmatrix}2\\-1\end{bmatrix}\right)=a\begin{bmatrix}0\\-1\end{bmatrix}+b\begin{bmatrix}2\\0\end{bmatrix}$$

Again, we see how a linear transformation is completely determined by its action on a basis.

\end{initprob}



Our discussion thus far hinged on the fact that the representation of a vector in terms of the given basis elements is unique.  Imagine what would happen if this were not the case.  In Exploration Problem \ref{init:tij}, for instance, we might have been able to represent $\vec{v}$ as $a\vec{i}+b\vec{j}$ and $c\vec{i}+d\vec{j}$ ($a\neq c$ or $b\neq d$).  This would have potentially resulted in $\vec{v}$ mapping to two different elements: $aT(\vec{i})+bT(\vec{j})$ and $cT(\vec{i})+dT(\vec{j})$, implying that $T$ is not even a function.  Fortunately, Theorem ({\color{red} reference}) assures us that in $\RR^n$, vector representation in terms of the elements of a basis is unique.  
Lemma \ref{lemma:uniquerep} duplicates this result for all vector spaces.

\begin{lemma}\label{lemma:uniquerep}
Let $V$ be a vector space, and let $\mathcal{B}=\{\vec{v}_1, \vec{v}_2,\ldots,\vec{v}_n\}$ be a basis for $V$.  Then every element $\vec{v}$ of $V$ has a unique representation as linear combination of the elements of $\mathcal{B}$.
\end{lemma}
\begin{proof}
By the definition of a basis, we know that $\vec{v}$ can be written as a linear combination of $\vec{v}_1, \vec{v}_2,\ldots,\vec{v}_n$.  Suppose there are two such representations.  Then,
$$\vec{v}=a_1\vec{v}_1+ a_2\vec{v}_2+\ldots+a_n\vec{v}_n$$
$$\vec{v}=b_1\vec{v}_1+ b_2\vec{v}_2+\ldots+b_n\vec{v}_n$$
But then we have:
\begin{align*}
a_1\vec{v}_1+ a_2\vec{v}_2+\ldots+a_n\vec{v}_n&=b_1\vec{v}_1+ b_2\vec{v}_2+\ldots+b_n\vec{v}_n\\
a_1\vec{v}_1+ a_2\vec{v}_2+\ldots+a_n\vec{v}_n-(b_1\vec{v}_1+ b_2\vec{v}_2+\ldots+b_n\vec{v}_n)&=\vec{0}\\
(a_1-b_1)\vec{v}_1+ (a_2-b_2)\vec{v}_2+\ldots+(a_n-b_n)\vec{v}_n&=\vec{0}
\end{align*}
Because $\vec{v}_1, \vec{v}_2,\ldots,\vec{v}_n$ are linearly independent, we have $a_i-b_i=0$ for $1\leq i\leq n$. Consequently $a_i=b_i$ for $1\leq i\leq n$.
\end{proof}

\begin{general}
Suppose we want to define a linear transformation $T:V\rightarrow W$.  Let $\mathcal{B}=\{\vec{v}_1,\ldots,\vec{v}_p\}$ be a basis of $V$.  Would it be sufficient to specify the image of each $\vec{v}_i$ in order to define $T$?
Given any vector $\vec{v}$ of $V$, we can uniquely express $\vec{v}$ as a linear combination of $\vec{v}_1,\ldots,\vec{v}_p$.  Thus,
$$\vec{v}=a_1\vec{v}_1+\ldots+a_p\vec{v}_p$$ for some scalar coefficients $a_1,\ldots,a_p$.
We find $T(\vec{v})$ as follows:
$$T(\vec{v})=T(a_1\vec{v}_1+\ldots+a_p\vec{v}_p)=a_1T(\vec{v}_1)+\ldots+a_pT(\vec{v}_p)$$
%Each element of $V$ has a unique representation (up to the order) in terms of the elements of $\mathcal{B}$.  (Why?)  Thus, the image of each vector $\vec{v}$ of $V$ is determined by the images of the elements of the basis.  So, $T$ is determined by its action on a basis.
\end{general}

In our final example, we will consider $T$ in the context of a basis of the codomain, as well as a basis of the domain.  This will later help us tackle the question of the matrix of $T$ associated with bases other than the standard basis of $\RR^n$.

\begin{example}\label{ex:subtosub1}
Let
$$\vec{v}_1=\begin{bmatrix}1\\2\\0\end{bmatrix}\quad\text{and}\quad\vec{v}_2=\begin{bmatrix}0\\1\\1\end{bmatrix}$$
$$\vec{w}_1=\begin{bmatrix}1\\0\\1\end{bmatrix}\quad\text{and}\quad\vec{w}_2=\begin{bmatrix}1\\0\\0\end{bmatrix}$$
Let $$V=\text{span}(\vec{v}_1, \vec{v}_2)\quad\text{and}\quad W=\text{span}(\vec{w}_1, \vec{w}_2)$$

Because each of $\{\vec{v}_1, \vec{v}_2\}$ and $\{\vec{w}_1, \vec{w}_2\}$ is linearly independent, let 
$$\mathcal{B}=\{\vec{v}_1, \vec{v}_2\}\quad\text{and}\quad\mathcal{C}=\{\vec{w}_1, \vec{w}_2\}$$
be bases of $V$ and $W$, respectively.


Define a linear transformation $T:V\rightarrow W$ by 
$$T(\vec{v}_1)=2\vec{w}_1-3\vec{w}_2\quad\text{and} \quad T(\vec{v}_2)=-\vec{w}_1+4\vec{w}_2$$

\begin{enumerate}
\item \label{item:subtosub1a}
Verify that $\vec{v}=\begin{bmatrix}2\\5\\1\end{bmatrix}$ is in $V$.
\item\label{item:subtosub1b}
Find $T(\vec{v})$.
\end{enumerate}
\begin{explanation}
\ref{item:subtosub1a} We need to express $\vec{v}$ as a linear combination of $\vec{v}_1$ and $\vec{v}_2$.  This can be done by observation or by solving the equation
$$\begin{bmatrix}1&0\\2&1\\0&1\end{bmatrix}\begin{bmatrix}a\\b\end{bmatrix}=\begin{bmatrix}2\\5\\1\end{bmatrix}$$
We find that $a=2$ and $b=1$, so $\vec{v}=2\vec{v}_1+\vec{v}_2$.  Thus $\vec{v}$ is in $V$.

\ref{item:subtosub1b} By linearity of $T$ we have \begin{align*}T(\vec{v})=T(2\vec{v}_1+\vec{v}_2)&=2T(\vec{v}_1)+T(\vec{v}_2)\\&=2(2\vec{w}_1-3\vec{w}_2)+(-\vec{w}_1+4\vec{w}_2)\\&=3\vec{w}_1-2\vec{w}_2=\begin{bmatrix}1\\0\\3\end{bmatrix}
\end{align*}

The important observation here is that given a linear transformation defined on the basis elements of $V$ in terms of the basis elements of $W$, we are able to find the image of any $\vec{v}$ in $V$ in terms of the basis elements of $W$. We will visit this idea again in Exploration Problem \ref{init:matlintransgeneral}.

% {\color{green} Paul, please check the blue wording}

% {\color{blue} The key take-away here is that given a linear transformation defined on the basis elements of $V$ in terms of the basis elements of $W$, we were able to express an element of $V$ in terms of the basis elements of $V$, and find the image of $\vec{v}$ in terms of the basis elements of $W$. We will visit this problem again in Exploration Problem} {\color{red} reference}
\end{explanation}

\end{example}

 

\subsection*{Practice Problems}
\begin{problem}
Find the image of $\vec{v}=\begin{bmatrix}0\\-3\end{bmatrix}$ under $S$ in Exploration Problem \ref{init:tij}.

$$S(\vec{v})=\begin{bmatrix}\answer{2}\\\answer{2}\end{bmatrix}$$
\end{problem}

\begin{problem}

Suppose $T:\RR^3\rightarrow \RR^3$ is a linear transformation such that 
$$T(\vec{i})=\begin{bmatrix}1\\1\\-2\end{bmatrix},\quad T(\vec{j})=\begin{bmatrix}4\\-1\\0\end{bmatrix},\quad T(\vec{k})=\begin{bmatrix}-1\\0\\2\end{bmatrix}$$

Find $T\left(\begin{bmatrix}2\\2\\-1\end{bmatrix}\right)$.

$$T\left(\begin{bmatrix}2\\2\\-1\end{bmatrix}\right)=\begin{bmatrix}\answer{11}\\\answer{0}\\\answer{-2}\end{bmatrix}$$
\end{problem}

  \begin{problem}
Let $$\vec{v}_1=\begin{bmatrix}1\\3\\0\end{bmatrix},\quad \vec{v}_2=\begin{bmatrix}0\\1\\-2\end{bmatrix}$$
$$\vec{w}_1=\begin{bmatrix}1\\-1\\4\end{bmatrix},\quad \vec{w}_2=\begin{bmatrix}0\\2\\-1\end{bmatrix}$$

Let $V=\text{span}(\vec{v}_1, \vec{v}_2)$ and $W=\text{span}(\vec{w}_1, \vec{w}_2)$.

Suppose $T:V\rightarrow W$ is a linear transformation such that 
$$T(\vec{v}_1)=\begin{bmatrix}2\\0\\7\end{bmatrix},\quad T(\vec{v}_2)=\begin{bmatrix}-1\\7\\1\end{bmatrix}$$
  \begin{problem}
  Verify that vectors $\begin{bmatrix}2\\0\\7\end{bmatrix}$ and $\begin{bmatrix}-1\\7\\1\end{bmatrix}$ are in $W$ by expressing each as a linear combination of $\vec{w}_1$ and $\vec{w}_2$.
  $$\begin{bmatrix}2\\0\\7\end{bmatrix}=\answer{2}\vec{w}_1+\answer{1}\vec{w}_2$$
  $$\begin{bmatrix}-1\\7\\1\end{bmatrix}=\answer{-1}\vec{w}_1+\answer{3}\vec{w}_2$$
  \end{problem}
  
  \begin{problem}
  Show that $\vec{v}=\begin{bmatrix}1\\2\\2\end{bmatrix}$ is in $V$ by expressing it as a linear combination of $\vec{v}_1$ and $\vec{v}_2$.
  $$\vec{v}=\answer{1}\vec{v}_1+\answer{-1}\vec{v}_2$$
  \end{problem}
  
  \begin{problem}
  Find $T(\vec{v})$ and express it as a linear combination of $\vec{w}_1$ and $\vec{w}_2$.
  $$T(\vec{v})=\answer{3}\vec{w}_1+\answer{-2}\vec{w}_2$$
  \end{problem}
  \end{problem}


\begin{problem} Let $V$ and $W$ be vector spaces, and let $\mathcal{B}_V=\{\vec{v}_1, \vec{v}_2, \vec{v}_3, \vec{v}_4\}$ and $\mathcal{B}_W=\{\vec{w}_1,\vec{w}_2, \vec{w}_3\}$ be bases of $V$ and $W$, respectively.  Suppose $T:V\rightarrow W$ is a linear transformation such that: $$T(\vec{v}_1)=\vec{w}_2$$ $$T(\vec{v}_2)=2\vec{w}_1-3\vec{w}_2$$
$$T(\vec{v}_3)=\vec{w}_2+\vec{w}_3$$
$$T(\vec{v}_4)=-\vec{w}_1$$
If $\vec{v}=-2\vec{v}_1+3\vec{v}_2-\vec{v}_4$, express $T(\vec{v})$ as a linear combination of vectors of $\mathcal{B}_W$.
$$T(\vec{v})=\answer{7}\vec{w}_1-\answer{11}\vec{w}_2+\answer{0}\vec{w}_3$$
\end{problem}




\end{document}
